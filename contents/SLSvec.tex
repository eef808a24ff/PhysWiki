% 简单的矢量和矩阵类
% C++|矩阵|容器

虽然我们可以直接使用 \verb|std::vector|, 但是通过创建一个简单的矢量类我们可以了解写一个类的基本思路, 在这之后就可以按照同样的思路创建各种容器了.

\begin{lstlisting}[language=cpp]
// Base Class for vector/matrix
template <class T>
class Vector
{
protected:
	T *m_p; // pointer to the first element
	Long m_N; // number of elements
public:
	// constructors
	Vector();
	explicit Vector(Long_I N);
	Vector(const Vector<T> &rhs); // copy constructor

	// member functions
	T* ptr(); // get pointer
	const T* ptr() const;
	Long size() const;
	void resize(Long_I N);
	T & operator[](Long_I i);
	const T & operator[](Long_I i) const;
	T& end(Long_I i = 1); // i = 1 for the last, i = 2 for the second last...
	const T& end(Long_I i = 1) const;
	Vector & operator=(const Vector &rhs);
	Vector & operator=(const T &rhs); // for scalar
	~Vector();
};
\end{lstlisting}

首先这是一个类模板, 而不是类本身, 例如如果用户声明了一个变量如 \verb|Vector<Doub>|, 那么编译器会根据模板生成一个定义, 将模板中的所有 \verb|T| 替换为 \verb|Doub|. 如果声明了多种类型的 \verb|Vector<T>|, 那么编译器也会相应生成多个不同版本的 \verb|Vector| 的定义.

注意 Vector 只有两个很简单的数据成员, 分别是类型 \verb|T| 的指针以及矢量的长度. 事实上矩阵元数据并不在 Vector 内部, 而是在 constructor 或 \verb|resize()| 函数中通过 \verb|new| 命令来动态分配的, 所以如果我们在程序中用 \verb|sizeof(Vector<Doub>)| 来测试占用内存, 得到的字节数将会是 \verb|m_p| 和 \verb|m_N| 的字节数相加, 而不包括矩阵元占用的空间. 如果在程序中直接用 \verb|Vector<Doub> a| 来声明变量, \verb|m_p| 和 \verb|m_N| 将会储存在 stack 中, 而矩阵元(以后如果有的话)将会在储存在 heap 中.
