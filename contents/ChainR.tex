% 复合函数求导(链式法则)

\pentry{微分\upref{Diff}}
若有一元函数 $f(x)$ 和 $g(x)$, 我们可以用 $f[g(x)]$ 表示其\bb{复合函数},即把 $g$ 的因变量作为 $f$ 的自变量.现在假设 $f$ 和 $g$ 在某定义域可导,且我们已知导函数 $f'$ 和 $g'$, 如何求复合函数 $f[g(x)]$ 的导数呢?

我们不妨先引入中间变量 $u$ 作为 $f$ 的自变量和 $g$ 的因变量,即 $y = f(u)$,  $u = g(x)$. 现在可以写出微分关系\upref{Diff}
\begin{equation}
\dd{y} = f'(u) \dd{u}  \qquad \dd{u} = g'(x) \dd{x}
\end{equation}
即 $y$ 的微小变化由 $u$ 的微小变化引起,而 $u$ 的微小变化又由 $x$ 引起.代入消去 $\dd{u}$,得
\begin{equation}
\dd{y} = f'(u) g'(x) \dd{x} = f'[g(x)] g'(x) \dd{x}
\end{equation}
或记为
\begin{equation}
\dd{f} = \dv{f}{g} \dv{g}{x} \dd{x}
\end{equation}
写成导数的形式为
\begin{equation}\label{ChainR_eq4}
\dv{y}{x} = f'[g(x)] g'(x)
\quad\text{或}\quad
\dv{f}{x} = \dv{f}{g} \dv{g}{x}
\end{equation}
这就是一元复合函数求导的\bb{链式法则}.

对于多重嵌套的情况如 $f[g(h(x))]$, 可以先对 $g[h(x)]$ 求导得 $g'[h(x)]h'(x)$ 再得到
\begin{equation}
\dv{x} f[g(h(x))] = f'[g(h(x))]g'[h(x)]h'(x)
\end{equation}
\phantom{=}

\begin{exam}{对函数求导}
\begin{equation}
\frac{1}{\sqrt{x^2+a^2}}
\end{equation}
首先令 $f(x) = 1/\sqrt{x}$ 再令 $g(x) = x^2+a^2$, 上式等于 $f[g(x)]$. 由基本初等函数的导数\upref{FunDer},
\begin{equation}
f'(x) = -\frac{1}{2\sqrt{x^3}}  \qquad g'(x) = 2x
\end{equation}
代入\autoref{ChainR_eq4}, 得
\begin{equation}
\dv{x} \frac{1}{\sqrt{x^2+a^2}} =  f'[g(x)] g'(x) = -\frac{x}{\sqrt{(x^2+a^2)^3}}
\end{equation}
\end{exam}

\begin{exam}{对函数求导}
\begin{equation}
\sin^2 x
\end{equation}
类似地,令 $f(x) = x^2$ 再令 $g(x) = \sin(x)$, 上式等于 $f[g(x)]$. 由基本初等函数的导数,%未完成
\begin{equation}
f'(x) =2x  \qquad g'(x) = \cos(x)
\end{equation}
代入\autoref{ChainR_eq4}, 得
\begin{equation}
\dv{x} \frac{1}{\sqrt{x^2+a^2}} =  f'[g(x)] g'(x) = 2\sin x\cos x
\end{equation}
\end{exam}

一种较灵活的情况是,当三个变量只有一个自由度\footnote{即任何一个变量值确定后,另外两个变量也随之确定}时,任何一个变量都可以看做任何另外一个变量的函数\footnote{姑且假设不会出现一个自变量对应两个因变量的情况},这时可以根据需要灵活运用链式法则,如\autoref{ChainR_ex3}.

\begin{exam}{加速运动公式}\label{ChainR_ex3}
假设质点做一维运动,位移,速度和加速度分别为 $x(t)$,  $v(t) = \dv*{x}{t}$,  $a(t) = \dv*{v}{t}$, 但若把速度 $v$ 看做复合函数 $v[x(t)]$, 根据链式法则有
\begin{equation}
a = \dv{v}{t} = \dv{v}{x}\dv{x}{t} = \dv{v}{x}v
\end{equation}
写成微分表达式,有 $a\dd{x} = v\dd{v}$. 注意到 $\dd (v^2) = 2v\dd{v}$, 代入得
\begin{equation}
\dd(v^2) = 2a \dd{x}
\end{equation}
若质点做匀加速运动,该式的物理意义是在任何一段微小时间内,速度平方的增量正比于这段时间内的位移增量.在一段时间 $[t_1,t_2]$ 内把这些增量累加起来,就得到高中熟悉的运动学公式
\begin{equation}
v_2^2-v_1^2 = 2a(x_2-x_1)
\end{equation}
其中 $x_1,v_1$ 和 $x_1,v_1$ 分别是 $t_1,t_2$ 时刻的位置和速度.
\end{exam}
