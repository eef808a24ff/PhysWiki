% 超定线性方程组

令 $A$ 为 $M\times N$ 的复数矩阵, $\vec x$ 和 $\vec y$ 为复数列矢量, 当 $M > N$ 时, 以下方程组称为超定方程组
\begin{equation}\label{OvrDet_eq1}
\vec y = \mat A \vec x
\end{equation}

我们把 $\mat y$ 和 $\mat A$ 拼接成一个 $M\times(N+1)$ 的矩阵, 当这个矩阵的 $M$ 个行矢量中只有 $N$ 个线性无关时, 我们只需取上式中 $N$ 个线性无关的行即可得到普通的线性方程组.

如果有大于 $N$ 个线性无关的行, 那么超定方程无解. 但我们仍然可以寻找一个最优的 $\vec x$, 使以下误差函数取最小值
\begin{equation}
\abs{\mat A\vec x - \vec y}^2 =  \sum_k  \qty(\sum_j A_{kj} x_j - y_k) \qty(\sum_j A_{kj} x_j - y_k)^*
\end{equation}

所以这时一个最小二乘法问题. 令误差函数分别对每个 $\Re[x_i]$ 和 $\Im[x_i]$ 求导等于 0, 得
\begin{equation}
\sum_j \qty(\sum_i A\Her_{ik} A_{kj}) x_j = \sum_k A\Her_{ik} y_k
\end{equation}
即
\begin{equation}\label{OvrDet_eq4}
\mat A\Her \mat A \vec x = \mat A\Her \vec y
\end{equation}
通常情况下该方程只有一个解, 也就是最小二乘法的解.
% 到底会不会有多个解或者无解呢?

对比\autoref{OvrDet_eq1} 可以发现\autoref{OvrDet_eq4} 只是在左右两侧同时乘以 $\mat A$ 的厄米共轭. 所以任何能满足\autoref{OvrDet_eq1} 的解也可以通过\autoref{OvrDet_eq4} 解得.
