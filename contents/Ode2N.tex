% 二阶常系数非齐次微分方程
% 微积分|微分方程|常微分方程|二阶常系数非齐次微分方程

\pentry{二阶常系数齐次微分方程\upref{Ode2}}

% 未完成:公式 ref
\subsection{结论}

在二阶常系数齐次微分方程的右端加上一个函数 $f(x)$, 就得到了二阶常系数非齐次微分方程
\begin{equation}\label{Ode2N_eq1}
y'' + by' + cy = f(x)
\end{equation}
这就是\textbf{二阶常系数非齐次微分方程}.其解为
\begin{equation}
y(x) = C_1 y_1 + C_2 y_2 - y_1\int \frac{y_2 f}{W} \dd{x} + y_2\int \frac{y_1 f}{W} \dd{x}
\end{equation}
其中 $W$ 可以写成二阶行列式
\begin{equation}
W = 
\begin{vmatrix}
y_1 & y_2\\
y'_1 & y'_2
\end{vmatrix} = y_1 y'_2 - y'_1 y_2
\end{equation}
其中 $y_1, y_2, W, f$ 都是 $x$ 的函数,后面的括号和自变量被省略.$y_1(x)$ 和 $y_2(x)$ 是对应齐次方程 
\begin{equation}\label{Ode2N_eq4}
y'' + by' + cy = 0
\end{equation}
的两个线性无关的解.

\subsection{应用} % 未完成

\subsection{推导}

下面介绍的方法叫常数变易法,其主要思想可参考一阶线性非齐次微分方程的通解%未完成

设通解的形式为
\begin{equation}\label{Ode2N_eq5}
y = v_1 y_1 + v_2 y_2
\end{equation}
其中,$v_i$ 也是关于 $x$ 的函数. 对该式两边求导,得
\begin{equation}\label{Ode2N_eq6}
y' = v'_1 y_1 + v'_2 y_2 + v_1 y'_1 + v_2 y'_2
\end{equation}
为了接下来计算方便,我们规定 $v_1$, $v_2$ 满足关系\footnote{这么规定会不会丢失一部分解呢?或许会,但是由于我们已经有了\autoref{Ode2N_eq1} 对应的齐次解 $y_1$ 和 $y_2$, 根据线性微分方程解的结构(见同济大学的《高等数学》),只需要找到\autoref{Ode2N_eq1} 的任意一个解,就可以找到它的通解.}
\begin{equation}\label{Ode2N_eq7}
v'_1 y_1 + v'_2 y_2 = 0
\end{equation}
把\autoref{Ode2N_eq7} 代入\autoref{Ode2N_eq6}, 得到
\begin{equation}\label{Ode2N_eq8}
y' = v_1 y'_1 + v_2 y'_2
\end{equation}
继续对求导,得到
\begin{equation}\label{Ode2N_eq9}
y'' = v'_1 y'_1 + v'_2 y'_2 + v_1 y''_1 + v_2 y''_2
\end{equation}
把\autoref{Ode2N_eq5} \autoref{Ode2N_eq8} \autoref{Ode2N_eq9} 代回原方程\autoref{Ode2N_eq1} 得
\begin{equation}
(v'_1 y'_1 + v'_2 y'_2 + v_1 y''_1 + v_2 y''_2) + b (v_1 y'_1 + v_2 y'_2) + c(v_1 y_1 + v_2 y_2) = f
\end{equation}
化简,得
\begin{equation}
(v'_1 y'_1 + v'_2 y'_2) + v_1 (a y''_1 + b y'_1 + c y_1) + v_2 (a y''_2 + b y'_2 + c y_2) = f \end{equation}
由于 $y_1$ 和 $y_2$ 都是\autoref{Ode2N_eq4} 的解,式(9)化为 % 未完成:公式编号有问题
\begin{equation} v'_1 y'_1 + v'_2 y'_2 = f
\end{equation}
总结一下,刚刚的推导说明,和在(5)的假设条件下,只要满足(10)即可满足(1)式.联立(5)和(10)式,得到关于 $v_1'$ 和 $v_2'$ 的方程组
\begin{equation}
\leftgroup{
y_1 v'_1 + y_2 v'_2 = 0\\
y'_1 v'_1 + y'_2 v'_2 = f
}\end{equation}
解得
\begin{equation}
\leftgroup{v_1' &= -y_2f/W\\
v_2' &= y_1 f/W}
\end{equation}
其中
\begin{equation}
W = y_1 y'_2 - y_2 y'_1 = 
\begin{vmatrix}
y_1 & y_2\\
y'_1 & y'_2
\end{vmatrix}
\end{equation}
对(13)% 未完成,公式链接
的两条式子积分,即可得到
\begin{equation}
v_1 =  - \int \frac{y_2 f}{W} \dd{x}  + C_1
\end{equation}
\begin{equation}
v_2 = \int \frac{y_1 f}{W} \dd{x}  + C_2
\end{equation}
(15)(16)代入(5)式,得到方程(1)的解为 %未完成,公式链接.
\begin{equation}
y(x) = C_1 y_1 + C_2 y_2 - y_1 \int \frac{y_2 f}{W} \dd{x} + y_2 \int \frac{y_1 f}{W} \dd{x}
\end{equation}
由于上式满足线性微分方程解的结构,所这已经是通解了.但是必须注意,根据常数变易法,我们只能在没有零点的区间内找到方程\autoref{Ode2N_eq1} 的通解.
