% 理想气体(巨正则系综法)

\subsection{理想气体的巨配分函数}
\begin{equation}\label{IdMCE_eq1}
\Xi  = \exp(z Q_1) = \exp(\frac{zV}{\lambda^3})
\end{equation}
\subsection{推导}
\begin{equation}\label{IdMCE_eq2}
\begin{aligned}
\Xi & = \sum_{N = 0}^\infty  \sum_{i = 1}^\infty  \E^{(N\mu - E_i)\beta}  = \sum_{N = 0}^\infty  z^N Q
 = \sum_{N = 0}^\infty  z^N \frac{1}{N!}Q_1^N \\
& = \sum_{N=0}^\infty  \frac{1}{N!} (z Q_1)^N
= \exp(z Q_1) = \exp(\frac{zV}{\lambda^3})
\end{aligned}
\end{equation}
其中用到了指数函数的泰勒展开(\autoref{Taylor_eq11}\upref{Taylor})。

\subsection{状态方程推导}
首先求出理想气体的巨势
\begin{equation}\label{IdMCE_eq4}
\Phi  =  - kT\ln \Xi  =  - kT\frac{zV}{\lambda ^3}
\end{equation}
由巨正则系综法\upref{MCEsb}
\begin{equation}\label{IdMCE_eq5}
\dd{\Phi} =  - P\dd{V} - S\dd{T} - N\dd{\mu}
\end{equation}
\begin{equation}\label{IdMCE_eq6}
P = - \qty(\pdv{\Phi}{V})_{T, \mu} = kT\frac{z}{\lambda ^3}
\end{equation}
注意 $z$ 是 $\mu $ 和 $T$ 的函数( $z = \E^{\mu/(kT)}$ ), $\lambda $ 是 $T$ 的函数, 所以 $z$ 和 $\lambda $ 在该偏微分中都看做常数.
\begin{equation}\label{IdMCE_eq7}
N = - \qty(\pdv{\Phi}{\mu})_{V,T} = kT\frac{V}{\lambda^3} \qty(\pdv{z}{\mu})_{V,T} = \frac{Vz}{\lambda ^3} 
\qquad ( = z{Q_1})
\end{equation}
若用上面两式消去 $z/\lambda^3$ 因子, 得到理想气体状态方程 $PV = NkT$.
  
另外, 想象在巨正则系综的物理情景中, 变化 $T$ 和 $\mu $,  从而使\autoref{IdMCE_eq7} 中的粒子数保持不变, 则 $N$ 不变时 可以看成 $T$ 的函数(而这个函数应该与正则系宗所得到的一样).由\autoref{IdMCE_eq7} 得
\begin{equation}\label{IdMCE_eq8}
\mu  = kT\ln \frac{N\lambda^3}{V}
\end{equation}
再测试一下状态方程 $PV =  - \Phi$,  得到 $PV = kTzV/\lambda^3$,  这与上面的压强公式(编号)重复, 没有新的信息. 若把粒子数公式 $N = Vz/\lambda^3$ (编号)代入理想气体的巨配分函数 $\Xi  = \exp(zV/\lambda^3)$(编号)以及巨势 $\Phi = - kTzV/\lambda^3$(编号), 得到两个个相当简洁的表达式, 可以方便记忆
\begin{equation}\label{IdMCE_eq9}
\Xi = \exp(N)
\end{equation}
\begin{equation}\label{IdMCE_eq10}
\Phi = - NkT
\end{equation}
理想气体的熵为
\begin{equation}\label{IdMCE_eq11}
\ali{
S &=  - \qty(\pdv{\Phi}{T})_{V, \mu}  = Vk\frac{T}{\lambda^3}\pdv{z}{T} + kTz\pdv{T} \qty(\frac{T}{\lambda^3}) \\
& = Vk\frac{T}{\lambda ^3} \qty(-\frac{\mu z}{kT^2}) + kTz\pdv{T} \qty(\frac{(2\pi mk)^{3/2}T^{5/2}}{h^3})\\
& = - \frac{\mu zV}{T\lambda^3} + kTz\frac52 \frac{(2\pi mkT)^{3/2}}{h^3}
=  - \frac{\mu zV}{T\lambda^3} + \frac52 \frac{kTz}{\lambda^3}\\
&= Nk \qty(\frac52 - \frac{\mu}{kT})
}\end{equation}
这里得出的熵是 $\mu $ 和 $T$ 的函数(从巨正则系综的物理情景来看, 得出的所有结果都应该是预先设定的参数 $\mu $ 和 $T$ 的函数).

为了和巨正则系综比较, 把\autoref{IdMCE_eq8} 代入\autoref{IdMCE_eq11},  即把粒子数人为保持不变, 一切看成温度的函数. 果然得到了理想气体的熵(Sackur-Tetrode公式)
\begin{equation}\label{IdMCE_eq12}
S = Nk \qty(\ln \frac{V}{N \lambda^3} + \frac52)
\end{equation}

\subsection{理解}

巨正则系综法的物理情景是: 让系统(体积 $V$ )与粒子源(化学势 $\mu $ )和热源(温度 $T$ ) 保持平热平衡, 由 $\mu $ 和 $T$ 决定粒子数 $N$,  压强 $P$,  能量 $E$ 等等. 这与微正则系综或正则系宗的物理情景不一样. 但是得到的结论却是一样的.
