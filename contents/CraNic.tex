% Crank-Nicolson 算法(一维)
% 算法|薛定谔方程|数值解|偏微分方程

薛定谔方程为
\begin{equation}
-\frac12 \pdv[2]{\psi}{x} + V\psi = \I \pdv{\psi}{t}
\end{equation}
用 Crank-Nicolson 或 Caley scheme\footnote{二者是一回事, 见 Numerical Recipes 19.2} 得到的结果是
\begin{equation}\label{CraNic_eq2}
\qty(1+\frac{\I}{2}\mat H\Delta t)\psi^{n+1} = \qty(1-\frac{\I}{2}\mat H\Delta t)\psi^n
\end{equation}
其中二阶导数用三点差分计算, 得
\begin{equation}
\ali{
\psi_i^{n+1} - \psi_i^n &= \frac{\I\Delta t}{4\Delta x^2} (\psi_{i-1}^n - 2\psi_i^n + \psi_{i+1}^n + \psi_{i-1}^{n+1} - 2\psi_i^{n+1} + \psi_{i+1}^{n+1})\\
&\qquad\qquad - \frac{\I\Delta t}{2}(V_i^n\psi_i^n + V_i^{n+1}\psi_i^{n+1})
}\end{equation}
令 $\alpha = \I\Delta t/(4\Delta x^2), \beta = \I\Delta t/2$, 整理可得
\begin{equation}\label{CraNic_eq4}
\ali{
&\quad -\alpha\psi_{i-1}^{n+1} + (1+2\alpha + \beta V_i^{n+1})\psi_i^{n+1} - \alpha \psi_{i+1}^{n+1}\\
&= \alpha\psi_{i-1}^n + (1 - 2\alpha - \beta V_i^n)\psi_i^n + \alpha \psi_{i+1}^n
}\end{equation}
其中 $\psi_i^n = \psi(x_i, t_n)$, $V_i^n = V(x_i, t_n)$.

我们把一个区间划分成 $N_x - 1$ 段等长的区间, 并令 $N_x$ 个格点为 $x_1\dots x_{N_x}$. 最简单的边界条件是取 $\psi(x_1) = \psi(x_{N_x}) = 0$. 这样\autoref{CraNic_eq4} 中的 $i$ 可以取 $i = 2\dots N_x - 1$, 得到 $N_x - 2$ 条式子, 其中只有 $\psi_2^{n+1}\dots \psi_{N_x-1}^{n+1}$ 这 $N_x - 2$ 个未知量, 每条式子最多包含连续 3 个未知量. 将线性方程用矩阵表示, 就可以得到一个三对角矩阵(第一行和最后一行只有两个系数).

但事实上, 还可以继续减少计算量. 将\autoref{CraNic_eq2} 整理后得
\begin{equation}\label{CraNic_eq5}
\qty(\frac12 + \frac{\I}{4}\mat H\Delta t)\qty(\psi^{n+1}+\psi^n) = \psi^n
\end{equation}
解这个方程, 再减去 $\psi^n$ 即可.

\subsection{虚时间}
使用虚时间后, \autoref{CraNic_eq2} 和\autoref{CraNic_eq5} 分别变为
\begin{equation}
\qty(1+\frac12\mat H\Delta t)\psi^{n+1} = \qty(1-\frac12\mat H\Delta t)\psi^n
\end{equation}
\begin{equation}
\qty(\frac12 + \frac14\mat H\Delta t)\qty(\psi^{n+1}+\psi^n) = \psi^n
\end{equation}

(代码未完成)
