% 四象限 Arctan 函数
% atan2|反三角函数|反正切|arctan

我们经常会遇到这样一个问题: 已知平面直角坐标系上一点 $P$, 坐标为 $(x, y)$, 求射线 $OP$ 与 $x$ 轴正方向的夹角 $\theta$.%图未完成
首先我们要给这个夹角取一个范围, 一般来说既可以取 $[0, 2\pi)$ 也可以取 $(-\pi, \pi]$, 但如无特殊说明, 我们统一使用后者.

一些教材中通常直接用 $\theta = \arctan(y/x)$ 来表示这一关系, 但这仅适用于 $x > 0$ 的情况, 当 $x < 0$ 时 $\theta$ 的范围仍然只能取 $(-\pi/2, \pi/2)$.

所以我们来定义一个符合要求的新函数, 记为 $\Arctan(y, x)$ \footnote{在许多编程语言中 $\arctan$ 被记为 \lstinline|atan|, $\Arctan$ 被记为 \lstinline|atan2|. 也有一些教材将 $\Arctan$ 记为 $\opn{Tan}^{-1}$}.

\begin{equation}\label{Arctan_eq1}
\Arctan(y,x) \equiv 
\begin{cases}
\arctan (y/x) \quad &(x > 0)\\
\arctan (y/x) + \pi  &(x < 0,\,y \geqslant 0)\\
\arctan (y/x) - \pi  &(x < 0,\,y < 0)\\
\pi /2  &(x = 0, \,y > 0)\\
 -\pi /2  &(x = 0, \,y < 0)\\
0 & (x=0,\,y=0)
\end{cases}
\end{equation}

注意这个函数在除了在原点和 $x$ 轴的负半轴, 其它地方都是连续且光滑的.
