% 角动量

角动量问题包含一般的轨道角动量,和更有趣点的自旋,两者能产生磁矩,却有着不一样的产生系数(g因子). 当然,如果不考虑磁效应的话,角动量有自己的一般理论;但是如果考虑磁效应,利用角动量来解决很多磁学问题是很常见的,一种极端简化的模型就是Ising Model和Heisenberg Model,会在讨论磁学问题的时候详细分析.

\subsection{一般理论 General Ideas}

角动量的问题其实相对来说更独立一点. 它有自己的一套体系. 本部分将从基本的数学结构开始建立角动量理论,其中会逐步的引入高级一点的数学描述(群描述),虽然这部分知识并不难,但是却很多时候并不容易理解.

\subsubsection{经典角动量的直观推广}

很容易证明,经典的角动量 $\bvec L = \bvec r\times \bvec p$ 对应的算符是一个很个性的算符:它在直角坐标中的三个分量都不对易,而且它们三个都是平权的. 我们知道,能够同时确定的物理量必须满足:它们有共同的本征态,换句话说,它们彼此对易. 因此,我们在这三个分量里面只能有一个能确定. 既然平权,我们不妨取为 $z$ 方向分量. 计算仍然给出这三个方向的角动量分量算符,和总角动量的平方这个标量算符也是对易的\footnote{这一段里的结论的验算可以参考后面的练习.}.

写出来就是
\begin{align}
[\hat L_z, \hat{\bvec L}^2] &= 0 \label{QMAM_eq1}\\
[\hat L_a, \hat L_b] &= \I\hbar\epsilon_{abc}L_c \label{QMAM_eq2}
\end{align}

其中,后者是个标量算符. 本章在后面很可能习惯性的把 $\hbar$ 略去,请注意.
\begin{equation}
\hat{\bvec L}^2 = \hat L_x^2+\hat L_y^2+\hat L_z^2
\end{equation}

于是,如果仅在角动量框架内的话,我们可以取可以同时确定的本征态 $|l,m\rangle$,其中
\begin{align}
\hat{\bvec L}^2|l,m\rangle &= l(l+1)|\hbar^2l,m\rangle\\
\hat L_z|l,m\rangle &= m\hbar|l,m\rangle
\end{align}

其中呢,$l(l+1)$ 只是写起来比较带感而已,至此我们没给出任何更具体的关于本征值的结论.

\begin{exer}{}
考虑到角动量算符的定义,
\begin{equation}
{\hat{\bvec L}} = \displaystyle\frac{1}{2}\left({\hat{\bvec p}}\times {\hat{\bvec r}} - {\hat{\bvec r}}\times {\hat{\bvec p}}\right)
\end{equation}

(1) 请验证\autoref{QMAM_eq1} 和\autoref{QMAM_eq2},其中需要计算
\begin{equation}
[\hat L_a, \hat r_b] = ? \quad [\hat L_a, \hat p_z] = ? 
\end{equation}

(2) 计算矢量算符
\begin{equation}
\hat{\bvec L}\times \hat{\bvec L} = ?
\end{equation}
\end{exer}

可以像前面处理一维谐振子那样处理体系,我们定义\bb{阶梯算符}为
\begin{equation}
\hat L_{\pm} = \hat L_x \pm i\hat L_y
\end{equation}

这个算符有什么好处呢?我们来做一些简单的计算. 一个重要的关系就是
\begin{equation}\label{QMAM_eq10}
[\hat L_z, \hat L_{\pm}] = \pm \hat L_{\pm}
\end{equation}

这个的推导很简单,可以自己完成. 我们要讲的,是 $L_{\pm}|l,m\rangle$ 这个东西,它也是 $L_z, {\bvec L}^2$ 的本征态. 前者很简单,我们只需要计算下面这个式子:
\begin{equation}
L_z L_{\pm}|l,m\rangle = (L_{\pm}L_z \pm L_{\pm})|l,m\rangle = (m\pm 1)L_{\pm}|l,m\rangle
\end{equation}

由此我们可以看出,很有可能
\begin{equation}
L_{\pm}|l,m\rangle \overset{?}{=} c|l,m\pm1\rangle
\end{equation}

其中 $c$ 是一个待确定的数,可以明确的知道它和 $l,m$ 有关. 只需要确定它是不是 ${\bvec L}^2$ 的本征值为 $l$ 的本征态就行了. 我们可以去计算
\begin{equation}
{\bvec L}^2L_{\pm}|l,m\rangle
\end{equation}

这次,我们得先做点工作:我们直接重新写 ${\bvec L}^2$ 的表达形式:
\begin{equation}
{\bvec L}^2 = L_z(L_z+1)+L_-L_+
\end{equation}

由\autoref{QMAM_eq10},很容易证明这个写法满足
\begin{equation}\ali{
{\bvec L}^2L_{\pm}|l,m\rangle &=  L_z(L_z+1)L_{\pm}|l,m\rangle + L_-L_+L_{\pm}|l,m\rangle\\
&= (m\pm1)(m+1\pm1)L_{\pm}|l,m\rangle + L_-L_+L_{\pm}|l,m\rangle
}\end{equation}

验算得到
\begin{equation}
[\hat L_+,\hat L_-] = 2\hat L_z
\end{equation}

由此,我们分别验算 $L_+,L_-$. 先看 $L_+$
\begin{equation}
L_-L_+L_+ = (L_+L_- + 2L_z) L_+ = L_+ (L_-L_+) - 2L_zL_+
\end{equation}

从而
\begin{equation}\ali{
&\quad\,\, {\bvec L}^2L_{+}|l,m\rangle \\
&= (m+1)(m+2)L_{+}|l,m\rangle +(L_+ (L_-L_+) - 2L_zL_+)|l,m\rangle \\
&= (m+1)(m+2)L_{+}|l,m\rangle +\left(L_+ ({\bvec L}^2 - L_z(L_z+1)) -2(m+1)L_+\right)|l,m\rangle \\
&= \left[(m+1)(m+2) + l(l+1) - m(m+1) - 2(m+1)\right]L_+|l,m\rangle \\
&= l(l+1)L_+|l,m\rangle
}\end{equation}

另一方面,$L_-$ 满足
\begin{equation}
L_-L_+L_- = L_-(L_-L_+ -2L_z) = L_-(L_-L_+) - 2L_-L_z
\end{equation}

从而
\begin{equation}\ali{
&\quad\,\, {\bvec L}^2L_{-}|l,m\rangle\\
&= m(m-1)L_{-}|l,m\rangle +(L_- (L_-L_+) + 2L_-L_z)|l,m\rangle \\
&= m(m-1)L_{-}|l,m\rangle +\left(L_- ({\bvec L}^2 - L_z(L_z+1)) + 2mL_-\right)|l,m\rangle \\
&= \left[m(m-1) + l(l+1) - m(m+1) + 2m\right]L_-|l,m\rangle \\
&= l(l+1)L_-|l,m\rangle
}\end{equation}

啊,总之,我们得到了我们喜欢的式子
\begin{equation}\label{QMAM_eq21}
L_{\pm}|l,m\rangle = c|l,m\pm1\rangle
\end{equation}

至于求 $c$ 如何求,这个还是很有技巧的,我们作为练习引导大家一步步求出来. 请参考\autoref{QMAM_exe2},结论为
\begin{equation}
L_{\pm}|l,m\rangle = \sqrt{l(l+1) - m(m\pm 1)}|l,m\pm1\rangle
\end{equation}

接下来我们看一个实际的例子,理解一下我们干了什么.

\begin{exam}{}
有一个体系,哈密顿量为
\begin{equation}
\hat H = \hat L_x^2 + \hat L_y^2 -\hat  L_z = \sum_{l,m} E_0 \hat L_-\hat L_+
\end{equation}

$E_0$ 是常数,它的本征态显然为各个角动量态 $|l,m\rangle$,假设每个态的能量为 $E_{l,m}$,可以写成
\begin{equation}
\hat H = \sum_{l,m}E_{l,m}|l,m\rangle\langle l,m|
\end{equation}

而通过插入resolution of identity,哈密顿量可以写成
\begin{equation}
\begin{split}
\hat H &= \sum_{l,m} E_0 \hat L_-\hat L_+ = \sum_{l,m} E_0 \hat L_-\hat L_+ |l,m\rangle \langle l,m|\\ &= \sum_{l,m} E_0 \left[l(l+1)-m(m+1)\right]|l,m\rangle \langle l,m|
\end{split}
\end{equation}

从而轻易得出
\begin{equation}
E_{l,m} = E_0 \left[l(l+1)-m(m+1)\right]
\end{equation}
\end{exam}

\subsubsection{角动量叠加原理}

这一部分的内容实际上可以说是不涉及高深数学知识的角动量的最复杂的知识了.

\begin{exer}{}\label{QMAM_exe2}
如何求\autoref{QMAM_eq21} 中的系数 $c$?换句话说,如何求
\begin{equation}
\langle l,m\pm1|L_{\pm}|l,m\rangle
\end{equation}
这实际上还包含一个被称为Condon-Shortley convention的事情在影响这个 $c$. 它叫convention是说明它确实有一些待确定的自由度,就是这个 $c$ 的相位. 我们后面会看到.

注意到,由算符的Hermitianity,
\begin{equation}
\langle l,m|L_{\pm}\Her|l,m\pm1\rangle = \langle l,m|L_{\mp}|l,m\pm1\rangle = c^*
\end{equation}
即
\begin{equation}
L_{\mp}|l,m\pm1\rangle = c^*|l,m\rangle
\end{equation}
或者说
\begin{equation}
L_{\mp}L_{\pm}|l,m\rangle = c^*c|l,m\rangle
\end{equation}
这一下就简单了,因为我们注意到,在 ${\bvec L}^2$ 里面也有类似的项. 分开来看:
\begin{equation}
L_-L_+ = {\bvec L}^2 - L_z(L_z+1)
\end{equation}
于是
\begin{equation}
L_+|l,m\rangle = c|l,m+\rangle, |c|^2 = l(l+1)-m(m+1)
\end{equation}
类似的,
\begin{equation}
L_+L_- = L_-L_+ +2L_z = {\bvec L^2} -L_z(L_z-1)
\end{equation}
于是
\begin{equation}
L_-|l,m\rangle = c|l,m-1\rangle, |c|^2 = l(l+1)-m(m-1) 
\end{equation}
著名的Condon-Shortley convention规定,
\begin{equation}
c = \sqrt{l(l+1) - m(m\pm1)}
\end{equation}
从而
\begin{equation}
L_{\pm}|l,m\rangle = \sqrt{l(l+1) - m(m\pm 1)}|l,m\pm1\rangle
\end{equation}
\end{exer}

\subsection{磁学问题 Magnetic Problems}

这部分的问题,一定程度上参考了Auerbach的书\cite{Auerbach}.