% 线性变换
% 线性代数|平面旋转变换|线性变换|逆变换|矩阵

% 未完成: 应该重新写, 从几何矢量出发, 给出 “线性” 的定义, 给出一些平面线性变换的例子(包括投影变换), 然后再总结出一般的代数形式, 给出矩阵表示

\begin{issues}
\issueDraft
\issueOther{本词条需要重新创作和整合,融入章节逻辑体系.}
\end{issues}

%应该整合进入线性映射词条?

%\pentry{平面旋转变换\upref{Rot2DT}, 映射\upref{map}}
\pentry{线性映射\upref{LinMap}}


线性空间$V$上的一个线性变换$T$,是指把$V$中的任意向量$\bvec{v}$映射为另一个向量$T\bvec{v}$的\textbf{操作}.这个操作满足线性性,因此被称作线性变换;线性性的好处在于我们只需要讨论基向量被映射到哪里,就知道了任何向量会被映射到哪里.

用线性映射\upref{LinMap}的语言来说,线性变换就是$V$到自身的一个线性映射.

\begin{definition}{线性变换}
给定线性空间$V$,如果$T$是$V$到$V$上的线性映射,那么称$T$是一个$V$上的\textbf{线性变换(linear transformation)}.$T$将$\bvec{v}\in V$映射为$T\bvec{v}$.
\end{definition}




