% 经典力学及其他物理理论

%\subsection{牛顿力学}
%牛顿力学主要由牛顿的运动三定律和万有引力定律构成.
% 未完成,什么是牛顿力学
% 介绍已知的四种作用力

\subsection{物理学理论的可证伪性}

著名的奥地利哲学家波普尔(Popper)对科学的划界是: 一个命题是科学的, 当且仅当它是\bb{可证伪}的. 如果有人提出一个物理理论,那么既可以尝试用它来计算已有的实验结果,也可以用它来预言一些没有做过的实验结果.如果在实验误差范围内,所有实验与理论计算得到的结果一致,那么就还没有证据表明这个理论是错误的,但也不能说它是绝对正确的.毕竟我们不可能把一个理论的每一种实验,每一套参数都做一遍.然而一旦有一个实验与该理论的计算结果不相符,那么就可以证明这个理论是错误的\footnote{当然首先要考虑是否存在计算错误,实验操作失误,或者存在未考虑到的因素}, 这就是物理学理论的\bb{可证伪性}.

然而可惜的是, 在物理学中目前还没有一个理论可以在任意范围内解释实验或观测结果, 所有的理论(如经典力学, 相对论, 量子力学,量子场论) 都只在一定的范围内成立. 我们能做的仅仅是不断创造与实验符合得更精确, 且适用范围更广的理论. 这样一来, 给一个曾经普遍接受的理论打上“错误”的标签似乎有些不妥, 于是我们一般称其为\bb{在适用范围内成立}.


\subsection{物理理论的适用范围\ 经典力学的价值}

经典力学在\bb{宏观低速}的范围内适用.粗略而言,“宏观” 要求物体的质量远大于原子的质量,“低速” 要求物体的运动速度远小于光速. 事实上还有一个条件是 “弱引力场”,例如由于水星离太阳较近,引力场较强,导致其轨道与经典力学的计算出现偏差(轨道进动).所以严格来说,经典力学是一个被证伪的理论.

若上述中只有“低速” 条件不满足,我们就需要使用狭义相对论,若“弱引力场” 条件不满足,就需要广义相对论(狭义相对论是广义相对论的一部分),若“宏观” 条件不满足,就需要量子力学,若都不满足,那么现在还没有非常完善的理论可以计算(叫做\bb{量子场论,Quantum Field Theory}).

以相对论(狭义和广义的统称)为例,它所适用的范围既包含了经典力学适用的范围,又包含了 “高速” 和 “强引力场”,所以原则上相对论可以完全取代经典力学.由于经典力学在适用的范围内已经得到几百年来大量的实验验证,那么如果相对论是正确的,在经典力学适用的范围内,用相对论计算问题就应该得到同样的结果\footnote{准确来说,二者计算结果的误差需要在实验的测量误差范围内.}.值得注意的是,相对论提出的一些物理概念与经典力学大相径庭.经典的万有引力定律提出任何两个物体之间都存在万有引力,而相对论却指出并不存在引力,而是有质量的物体扭曲了周围的时空,使周围物体的运动方式不同.既然相对论的适用范围更广,那么至少从目前看来相对论对物理现象的解释才是更可信的,而经典力学的解释是错误的. 

既然如此,为什么我们还要先学习经典力学呢? 首先无论在概念上还是数学上,经典力学比相对论简单得多. 其次在日常生活或生产中我们接触的绝大部分运动都在经典力学的适用范围内. 第三, 相对论中同样会出现 “参考系”,“速度”,“能量”,“动量”,等概念,这些概念只有先学习经典力学才会有一个初步的认识,才能继续学习相对论.最后,通过学习经典力学可以了解物理中常见的数学工具,包括一些基础的微积分,矢量分析,线性代数等,这些数学在物理的其他领域更是无处不在.

以上论述同样适用于量子力学与经典力学的关系. 量子力学除了在经典力学的范围适用, 还描述了微观粒子的运动. 总而言之经典力学在现代的物理学中只是一个简单的近似模型,提出的一些原理并不正确, 公式也只是一种近似.一些 “民间科学家” 时常企图推翻牛顿定律, 显然是还不了解这点.

另一方面,即使是相对论和量子力学也并非完美无瑕,通常所说的量子力学是指 “非相对论量子力学”,即同样要求 “低速” 和 “弱引力场”.目前,“相对论量子力学” 的理论还并不完善,是许多理论物理学家努力的方向.
