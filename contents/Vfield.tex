% 矢量场

\pentry{球坐标系的定义\upref{Sph},矢量的求导法则\upref{DerV}}

对空间中指定范围的每一点 $P$ 赋予一个矢量 $\vec v$, 就在该空间中形成了一个\bb{矢量场}.例如,电荷附近的任意一点都存在一个电场矢量,这就构成了一个矢量场.管道中任意一点的水流都存在一个速度矢量,它们也构成一个矢量场.

矢量场在不同的参考系中有不同的表示方法.在空间直角坐标系中,矢量场可以用矢量的三个分量关于 $x,y,z$ 三个坐标的函数表示.点 $P(x,y,z)$ 处的矢量分量为
 \begin{equation}
\leftgroup{
v_x(x,y,z) &= \vec v \vdot \uvec x\\
v_y(x,y,z) &= \vec v \vdot \uvec y\\
v_z(x,y,z) &= \vec v \vdot \uvec z
}\end{equation}
也可以作为单位正交基\upref{OrNrB} 的线性组合写成一个整体
\begin{equation}
\ali{
\vec v &= (\vec v \vdot \uvec x)\,\uvec x + (\vec v \vdot \uvec y)\,\uvec y + (\vec v \vdot \uvec z)\,\uvec z\\
&= v_x(x,y,z)\,\uvec x + v_y(x,y,z)\,\uvec y + v_z(x,y,z)\,\uvec z
}\end{equation}
 
在球坐标系\upref{Sph}中,也可以把每个点的矢量根据该点处的三个单位矢量 $\uvec r$,  $\uvec \theta$,  $\uvec \phi$ 分解为三个分量. 基底的线性组合为
\begin{equation}
\vec v = v_r(r,\theta ,\phi)\,\uvec r + v_\theta(r,\theta ,\phi) \,\uvec\theta  + v_\phi(r,\theta ,\phi)\,\uvec \phi  
\end{equation} 

需要特别注意,$\uvec r$,  $\uvec \theta$,  $\uvec \phi$ 也是关于 $(r,\theta ,\phi )$ 的函数,所以对 $\vec v$ 求导(或偏导)时必须根据矢量的求导法则\upref{DerV} 进行.

% 未完成: 这个例子命名应该放到“力场”词条中!
\eentry{力场\upref{V}}


\rentry{梯度\upref{Grad}, 散度\upref{Divgnc}, 旋度%未完成:引用
}