% Python 数据类型
% Python|列表|元组|list|tuple|字典|dictionary

\subsection{常用数据类型}
\subsubsection{列表(List)}
List (列表) 是 Python 中使用最常用的的数据类型之一.列表中元素的类型可以不相同,它支持数字,字符串甚至可以包含列表.列表是写在方括号 \verb|[]| 之间、用逗号分隔开的元素列表.
\begin{lstlisting}[language=python]
list1 = [1,2,3,4]
list2 = ['a','b',1,3]
print (list1)            # 输出完整列表
print (list1[0])         # 输出列表第一个元素
print (list2[1:3])       # 从第二个开始输出到第三个元素
print (list2[2:])        # 输出从第三个元素开始的所有元素
print (list2 * 2)    # 输出两次列表
\end{lstlisting}
输出为:
\begin{lstlisting}[language=python]
[1, 2, 3, 4]
1
['b', 1]
[1, 3]
['a', 'b', 1, 3, 'a', 'b', 1, 3]
\end{lstlisting}

\subsubsection{元组(Tuple)}
元组(tuple)与列表类似,不同之处在于元组的元素\textbf{不能修改}.元组写在小括号 \verb|()| 里,元素之间用逗号隔开.元组中的元素类型也可以不相同:
\begin{lstlisting}[language=python]
tup1 = (1,2,3,4)
tup2 = ('a','b',1,3)
print(tup1)            # 输出完整元组
print (tup1[0])         # 输出元组第一个元素
print (tup2[1:3])       # 从第二个开始输出到第三个元素
print (tup2[2:])        # 输出从第三个元素开始的所有元素
print (tup2 * 2)
\end{lstlisting}
输出
\begin{lstlisting}[language=python]
(1, 2, 3, 4)
1
('b', 1)
(1, 3)
('a', 'b', 1, 3, 'a', 'b', 1, 3)
\end{lstlisting}

\subsubsection{字典(Dictionary)}
字典是无序的对象集合.字典当中的元素是通过键来存取的,用 \verb|{}| 标识,它是一个无序的 \verb|key:value| 的集合, \verb|key| 和 \verb|value| 分别译成\textbf{键}和\textbf{值}. \verb|key| 必须使用不可变类型. 在同一个字典中, \verb|key| 必须是\textbf{唯一}的.例如统计一个班学生的成绩可以使用字典表示.
\begin{lstlisting}[language=python]
scores = {'语文': 89, '数学': 92, '英语': 93}
print(scores['语文'])
\end{lstlisting}
输出:
\begin{lstlisting}[language=python]
89
\end{lstlisting}
