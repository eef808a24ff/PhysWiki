% 分离变量法与张量积空间
% 张量积空间|本征函数|分离变量法|正交完备性

我们可以从张量积空间和本征问题的角度来理解分离变量法.

第一我们假设考虑的方程是线性的. 即解可以表示为齐次解的线性组合加一个非齐次解.

以球坐标的拉普拉斯方程为例, 这是一个齐次方程
\begin{equation}
\laplacian u = \frac{1}{r^2} \pdv{r} \qty(r^2 \pdv{u}{r}) + \frac{1}{r^2 \sin\theta}\pdv{\theta} \qty(\sin \theta \pdv{u}{\theta}) + \frac{1}{r^2 \sin^2 \theta} \pdv[2]{u}{\phi} = 0
\end{equation}
我们可以将 $u$ 所在的三维函数空间看做是三个变量各自的函数空间的张量积空间, 张量积空间中的基底为三个单变量函数空间中的基底的张量积
\begin{equation}
u(r, \theta, \phi) = f(r) g(\theta) h(\phi)
\end{equation}

要分离变量, 可以从最简单的变量开始, 分离出该变量的算符, 然后求该算符的本征方程的一组本征函数(本征矢), 作为该变量的函数空间的基底. 这样, 该算符就可以被替换为常数. 替换后, 尝试分离下一个变量的算符, 以此类推.

拉普拉斯方程中, 首先最容易分离的是关于 $\phi$ 的算符
\begin{equation}
\pdv[2]{\phi}
\end{equation}
令该算符的本征函数 $h_m(\phi)$ 为 $\phi$ 函数空间的基底, 令本征值为常数 $-m^2$
\begin{equation}
\pdv[2]{h}{\phi} = -m^2 h
\end{equation}
 
将算符替换为本征值, 然后容易发现后两项中都有公共的 $1/r^2$ 因子, 提取出来就可以分离出只含有 $\theta$ 的算符
\begin{equation}
 \frac{1}{\sin\theta}\pdv{\theta} \qty(\sin \theta \pdv{\theta}) - \frac{m^2}{\sin^2 \theta}
\end{equation}
令该算符的本征函数 $g_l(\theta)$ 为 $\theta$ 函数空间的基底, 令本征值为常数 $-l(l+1)$.

最后, 我们剩下仅关于 $r$ 的方程, 同样可以表示为本征方程
\begin{equation}
\pdv{r} \qty(r^2 \pdv{f}{r}) = l(l+1) f
\end{equation}
令 $f_l(r)$ 为 $r$ 函数空间的基底.

根据施图姆—刘维尔定理, 三组本征基底都是正交且完备的, 所以张量积空间中的基底也是正交完备的
\begin{equation}
u_{l,m}(r, \theta, \phi) = f_l(r) g_l(\theta) h_m(\phi)
\end{equation}
