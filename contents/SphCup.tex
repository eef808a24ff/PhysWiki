% 球谐函数 CG 系数 角动量加法

Bransden 附录 A4 给出了一个很有用的公式
\begin{equation}\label{SphCup_eq1}
\ali{
&\quad \int Y_{l_1 m_1} (\uvec r) Y_{l_2 m_2} (\uvec r) Y_{l_3 m_3}(\uvec r) \dd{\Omega}\\
&= (-1)^{m_3} \sqrt{\frac{(2l_1+1)(2l_2+1)}{4\pi(2l_3+1)}} \braket{l_1, l_2, 0, 0}{l_3, 0}\braket{l_1, l_2, m_1, m_2}{l_3, -m_3}
}\end{equation}
%来源:bransden,写成 3j 符号的形式会更对称, 见 https://physics.stackexchange.com/questions/10039/integral-of-the-product-of-three-spherical-harmonics

任何算符的一组归一化本征基底各自乘以一个任意相位因子 $\E^{\I \phi_I}$ 仍然是一组归一化的本征基底(本征值不变). 所以基底变换矩阵(如 CG 矩阵), 每一行或每一列分别乘以一个相位因子, 仍然表示相同的基底变换. 这些相位怎么取被称为相位约定(Phase Convention). 一般的约定是使所有 CG 系数为实数, 且第一行和第一列的矩阵元大于零(Wikipedia,Griffiths).

CG 系数有解析表达式
\begin{equation}
\ali{
&\braket{l_1 m_1 l_2 m_2}{l_1 l_2 L M} =
\sqrt{\frac{(2L+1)(L+l_1-l_2)!(L-l_1+l_2)!(l_1+l_2-L)!}{(l_1+l_2+L+1)!}}\\
&\times\sqrt{(L+M)!(L-M)!(l_1-m_1)!(l_1+m_1)!(l_2-m_2)!(l_2+m_2)!}\\
&\times\sum_k \frac{(-1)^k}{k!(l_1+l_2-L-k)!(l_1-m_1-k)!(l_2+m_2-k)!}\\
&\times \frac{1}{(L-l_2+m_1+k)!(L-l_1-m_2+k)!}
}\end{equation}

CG 系数的对称性有
\begin{equation}
\ali{
\bmat{l_1 &l_2 &L\\ m_1 &m_2 &M}
&= (-1)^{l_1+l_2-L}\bmat{l_1 &l_2 &L\\ -m_1 &-m_2 &-M}\\
&= (-1)^{l_1+l_2-L}\bmat{l_2 &l_1 &L\\ m_2 &m_1 &M}
}\end{equation}

\subsection{广义球谐函数}
有了 CG 系数的相位约定和球谐函数的相位约定, 就可以定义广义球谐函数(Generalized Spherical Harmonics)
\begin{equation}
\mathcal{Y}_{l_1,l_2}^{L,M}(\uvec r_1, \uvec r_2) = \sum_{m_1, m_2} C_{l_1 m_1 l_2 m_2}^{L,M} Y_{l_1 m_1}(\uvec r_1) Y_{l_2 m_2} (\uvec r_2)
\end{equation}

\subsection{对称性}
宇称算符 $\Pi$ 的作用是把所有自变量乘以 $-1$ 得到的函数. 本征函数是所有中心对称或反对车的函数, 本征值为分别为 $\pm 1$.

球谐函数是宇称算符的本征矢, 本征值为 $(-1)^l$ (\autoref{SphHar_eq6}\upref{SphHar}), 易得广义球谐函数也是宇称算符的本征矢, 本征值为 $(-1)^{l_1+l_2}$
\begin{equation}
\Pi \mathcal{Y}_{l_1,l_2}^{L,M}(\uvec r_1, \uvec r_2) =  \mathcal{Y}_{l_1,l_2}^{L,M}(-\uvec r_1, -\uvec r_2) = (-1)^{l_1+l_2} \mathcal{Y}_{l_1,l_2}^{L,M}(\uvec r_1, \uvec r_2)
\end{equation}

