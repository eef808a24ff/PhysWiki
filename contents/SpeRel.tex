% 狭义相对论的基本假设
% 狭义相对论|迈克尔逊—莫雷实验|光速不变|爱因斯坦|洛伦兹|相对性原理

% 基本完成
\pentry{公理系统\upref{axioms},麦克斯韦方程组\upref{MWEq}}

“光速在任意参考系下都不变”这一理念,和广为流传的误解不同,并不是实验中得来的.部分书籍会简单粗暴地告诉你,Michelson-Morley(迈克尔逊-莫雷)实验(下称 \textbf{MM 实验})是为了寻找以太存在的证据而进行的,实验结果表明在误差范围内光速在任何参考系都是一样的,因此提出了“光速不变原理”,也就是狭义相对论的两个公理之一.然而,这是对历史的错误描述.一些物理课本使用这样的描述是有其教育意义的,因为用这样的误解很容易引入狭义相对论,而不需要学生有扎实的电动力学基础.

本书秉承准确、翔实的原则,将当代物理史学界对“光速不变原理”的由来阐释如下.

\subsection{关于Michelson-Morley实验的常见误解}

\subsubsection{误解1:MM实验的目的是寻找以太}

\footnote{本小节的内容主要来自 University of Minnesota 在 2019 秋季的 History of 20th Century Physics课程}不是的.

MM实验是在1887年进行的.科学家们曾经就“光到底是粒子还是波”争论不休,早年由于牛顿的绝对权威,几乎所有人都顺从他而坚持光的粒子说.在1800年至1850年之间,波动说才逐渐占据了上风.但是在当时的常识看来,波动是介质运动的结果,也就是说,有波动就必须有介质.光既然是波,那么它就一定有介质,物理学家们将其称为\textbf{以太(ether)}.

很明显,根据定义,以太会和光作用.但是以太会不会和其它物质作用呢?这个问题没有直接的答案.光学界的泰斗杨(Young)和菲涅尔(Fresnel)认为以太是不会和普通物质相互作用的;而斯托克斯(Stokes)认为以太应该和普通物质作用,从而会被普通物质拖拽而产生运动.

杨的观点来自于一个天文学现象:恒星的\textbf{像差(aberration)}\footnote{“Upon considering the phenomena
of the aberration of the stars I am disposed to believe, that the luminiferous ether pervades the substance of all material bodies with little or no resistance, as freely perhaps as the wind passes through a grove of trees.” Thomas Young于1804年所说.笔者翻译其原话为:“考虑恒星的像差现象,我自然而然地相信,传播光的以太在任何物质中弥散,只有很少甚至没有阻力,就好像风吹过树林一般.”}.而斯托克斯的观点是,由于光有偏振,意味着光是横波;即使是可见光也有极高的频率\footnote{比如说,589纳米波长的光在我们眼中是黄色的,其频率高达$\nu=c/\lambda=5\times10^{17}\opn{Hz}$.其中$c$为光速,$\lambda=5.89\times10^{-10}\opn{m}$为波长},因此以太必须坚如磐石,怎么可能“像风吹过树林一般”?所以以太必须和普通物质相互拖拽\footnote{笔者对此论述有疑义:如果以太和普通物质不相互作用,那它坚如磐石也和普通物质无关啊.那个时代的物理学家普遍有很多来自日常经验的先入为主思想,在今天看来逻辑不严谨其实并不奇怪.}.%秉承百科原则,可能需要解释恒星像差现象?

MM实验由此产生,其目的是检验以太究竟会不会和普通物质作用.\textbf{也就是说,MM实验默认了以太的存在.}实验的基本思想是,如果以太不和普通物质作用,那么以太可以看成是一种绝对静止的参考系,地球在公转、自转等运动中一定会相对以太而运动,否则就过于凑巧了.但是如果以太和普通物质作用,那么地球就会拖拽以太,在地面上以太应该近乎静止.如果我们在地面检测光在两个相互垂直的方向上的传播速度,那么杨的理论预言光速会有不同,而斯通克斯的理论则语言光速不会变化.

当然了,实验结果是光速在两个方向上的差别非常小,因此当时人们认为MM实验证明了以太和普通物质有相互作用.

\subsubsection{误解2:爱因斯坦是受了MM实验的启发而提出“光速不变”的}

不是的.

尽管MM实验不是为了寻找以太,但实验结果也可以解读为我们今天所说的“光速不变原理”.但事实上,爱因斯坦并没有拿MM实验当回事,在他1905年的论文里并没有提到MM实验——至少没有明确提到.在很长一段时间里,物理史学者甚至都不确定爱因斯坦那时知不知道MM实验(现在的结论是,他知道的).1905年之后,爱因斯坦倒是频繁提到MM实验和狭义相对论的关系,但是并没有说过实验支撑了光速不变原理.

1931年,爱因斯坦和迈克尔逊在加州帕萨迪纳会面.爱因斯坦很不解地问迈克尔逊为什么要做那么一个实验,后者回答说,"Because it's fun!(因为很有意思啊!)" 

\subsubsection{误解3:洛伦兹只不过提出了尺缩钟慢原理}

洛伦兹的贡献实际上更多.

斯托克斯显然认为,两个方向的光速不变意味着以太相对地面静止,从而说明地球很可能和以太相互作用.洛伦兹(Lorentz)和斐兹杰惹(FitzGerald)提出,这有可能是因为干涉仪的臂长在和以太运动方向平行的时候更短而造成的.也就是说,他们依然认为以太是存在的,只不过用尺缩假设来支持杨的想法.实验物理学家们此时有了三个研究方向:寻找以太风、探测以太和普通物质的拖拽以及测量尺缩效应.

但是洛伦兹想得更多.由于那时已经有了麦克斯韦理论,人们知道光是一种电磁波.如果以太是光的介质,那就意味着以太是电场和磁场的介质,电磁场分布在以太里而不是凭空存在的.如果普通物质不和以太作用,那么物体在以太中的运动完全不会造成扰动;但是带电物质会被电磁场所作用,也就是说以太反过来会对物质有作用.那么以太究竟算不算可以和普通物质作用呢?

另外,由于麦克斯韦理论中并没有指明是在哪个惯性系成立,因此如果我们假设它在和以太相对静止的参考系成立而地球和以太有相对运动的话,那么麦克斯韦理论应该在地球上不成立.但是大量广泛的应用表明,麦克斯韦理论在地球上是适用的.

为了解释上述矛盾,洛伦兹猜想,同一个电磁场在不同的惯性系中应该会有所不同.他的猜想解释了为什么任何至多有一阶近似的光学实验都不可能探测出以太的运动.他进一步完善了这个理论,给出了电磁场在不同惯性系中的变化.巧合的是,这种变化和物质的尺缩效应是完全一样的.洛伦兹认为这只是一个巧合,但这个结果极大地启发了闵可夫斯基时空.相对论思想其实是在洛伦兹工作的基础上更进一步,把巧合看成是必然的,解释为时空本身的变换.所以洛伦兹不只是提出了尺缩钟慢原理和洛伦兹变换,他对电磁理论的修正也启发了相对论的时空观.

\subsection{光速不变原理和相对性原理}

在洛伦兹的基础上更进一步,以太的存在已经没什么逻辑上的必要了.假设光速不会变化是由时空本身的属性造成的,而不是任何类似以太的物质的作用,那么我们就得到了光速不变原理:光速在任何参考系中都不变.今天的公制单位中,使用铯-133原子中两个超精细分裂能级的跃迁频率来定义秒,而使用光速来定义米:一米就是光在$1/299792458$秒内行进的距离.也就是说,光速被\textbf{定义}为$c=299792458\opn{m/s}$.

注意,由于单位是可以任意取的,如果你强行定义$\opn{\tau}=299792458\opn{s}$,那么在这个单位下也有$c=1\opn{m/\tau}$.因此理论物理学家通常不会使用冗长的$c$的表达,而是简单地令$c=1$.这样计算结果非常简洁,并且如果想要回归日常单位,只需要做量纲分析,在合适的地方重新把$c$加上就可以了.比如说,当你算出来某个东西的速度是$0.2$,考虑到速度的量纲,你自然能得出这个东西的速度应该为$0.2c$.

抛弃以太而选择光速不变原理,相当于赋予了任何惯性参考系以相同的地位,没有哪个更特殊.秉承相同的理念,爱因斯坦进而认为任何惯性参考系都应该遵循完全相同的物理规律,不应该有地位差别.

洛伦兹对电磁理论的修正说明我们从伽利略时代以来就坚守的时空观并不能保证物理规律的一致性.这意味着我们需要新的时空观.




