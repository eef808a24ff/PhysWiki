% 线性变换
% 线性代数|平面旋转变换|线性变换|逆变换|矩阵

\pentry{平面旋转变换\upref{Rot2DT}}

从代数的角度来说,对于给出几个数,把它们分别与一些常数相乘再把积相加,得到另外几个数的的过程就叫\textbf{线性变换}. 例如,在“平面旋转变换\upref{Rot2DT}” 中,直角坐标系中任意一点 $P$ 的坐标 $(x,y)$ 绕远点旋转角 $\alpha $ 以后的坐标为
\begin{equation}\label{LTrans_eq1}
\leftgroup{
x' &= (\cos\alpha) x + (-\sin\alpha)y\\
y' &= (\sin \alpha)x + (\cos\alpha)y
}\end{equation}
这就是一个常见的线性变换,任意给出两个实数 $x,y$, 通过与常数相乘再相加的方法得到两个新的实数  $x',y'$. 

有些线性变换是一一对应的,例如上面的例子中,任何一组 $x,y$, 有且仅有一组 $x',y'$ 与之对应,反之亦然.在这种情况下,这个变换存在\textbf{逆变换}.

\subsection{线性变换的矩阵表示}

由 $n$ 个数 $x_1 \ldots x_n$ 变换到 $m$ 个数 $y_1 \ldots y_n$ 的线性变换的一般形式为
\begin{equation}
\leftgroup{
y_1 &= a_{11} x_1 + a_{12} x_2 + \ldots + a_{1n} x_n\\
y_2 &= a_{21} x_1 + a_{22} x_2 + \ldots + a_{2n} x_n\\
&\;\;\vdots \\
y_m &= a_{m1} x_1 + a_{m2} x_2 + \ldots + a_{mn} x_n
}\end{equation} 
这里一共有 $m \times n$ 个系数,每个系数的下标由两个数组成, $a_{ij}$ 是计算 $y_i$ 时 $x_j$ 前面的系数.为了书写方便,把这些系数写成一个 $m$ 行 $n$ 列的数表,用圆括号括起来,就是表示该变换的\textbf{矩阵}.
\begin{equation}\begin{pmatrix}
a_{11} & a_{12} & \ldots & a_{1n}\\
a_{21} & a_{22} & \ldots & a_{2n}\\
 \vdots & \vdots & \ddots & \vdots \\
a_{m1} & a_{m2} & \ldots & a_{mn}
\end{pmatrix}\end{equation} 
