% 浮力
% 重力|液体|密度|浮力|等效法

\subsection{等效法}

我们先用一个简单易懂的方式解释浮力. 假设在重力加速度为 $g$ 的环境中, 容器中密度为 $\rho_0$ 的液体完全静止. 这时令液体内部有一任意形状的闭合曲面, 体积为 $V_0$. 把曲面内部的液体作为一个整体做受力分析, 其质量为 $m = \rho_0 V_0$, 所受重力为 $mg = \rho_0 V_0 g$. 由于曲面中液体保持静止, 说明曲面外的液体对曲面内的液体施加了相同大小的浮力. 现在我们如果把曲面内的液体替换为一块密度为 $\rho$ 的物体, 由于曲面形状不改变, 外界液体对该物体的浮力仍然为
\begin{equation}
F = \rho_0 V_0 g
\end{equation}
注意 $V_0$ 为物体在水中部分的体积, 如果物体只有部分在水中, $V_0$ 将小于物体的体积.

\subsection{散度法}
\pentry{散度定理\upref{Divgnc}}

现在我们用面积分的方法表示浮力. 令 $z$ 轴竖直向上, 且水面处 $z = 0$, 则水面下压强为
\begin{equation}
P = -\rho_0 g z
\end{equation}
现在把上述的闭合曲面划分为许多个微面元, 第 $i$ 个面元用矢量 $\Delta \bvec S_i$, 表示, 其中模长为面元的面积, 方向为从内向外的法向. 这个面元受到外界液体的压力为
\begin{equation}
\Delta \bvec F_i = -P\Delta \bvec S_i = \rho_0 g z \Delta \bvec S_i
\end{equation}
现在把所有面元所受的压力求和, 并用曲面积分\upref{SurInt}表示为
\begin{equation}
\bvec F = \oint \rho_0 g z \dd{\bvec S}
\end{equation}
这就是物体所受的浮力. 我们先计算其 $z$ 分量, 等式两边投影到 $\uvec z$ (内积)得
\begin{equation}
F_z = \oint (\rho_0 g z \uvec z)\vdot \dd{\bvec S}
\end{equation}
我们可以把括号内的矢量看做一个矢量场 $\bvec A$, 其散度为
\begin{equation}
\div \bvec A = \pdv{z}(\rho_0 g z) = \rho_0 g
\end{equation}
对上式应用散度定理\upref{Divgnc}, 得
\begin{equation}
F_z = \int \div\bvec A \dd{V} = \rho_0 g V_0
\end{equation}
再来计算 $x$ 和 $y$ 分量的浮力, 由于
\begin{equation}
\div (\rho_0 g z\uvec x) = \div (\rho_0 g z\uvec y) = 0
\end{equation}
两个水平分量为零. 可见该结论与“等效法”中得出的一致.


