%圆锥曲线的光学性质

%未完成,需要图片
\pentry{椭圆的三种定义\upref{Elips3},抛物线的三种定义\upref{Para3},双曲线的三种定义\upref{Hypb3}}

\subsection{椭圆的光学性质}
$F_1$,$F_2$ 是椭圆的两个焦点,$ P $ 是椭圆上一点,则 $\angle F_1PF_2 $ 的外角平分线 $ PT $ 是椭圆的切线.

\subsubsection{证明}
作点 $F_2$ 关于直线 $PT$ 的对称点 $F_2^1$.由于 $PT$ 是 $\angle F_1PF_2 $ 的外角平分线,因此 $F_1$,$P$,$F_2^1$ 三点在同一条直线上.

记 $Q$ 是直线 $PT$ 上的任意一点.于是
\begin{equation}\label{ConOpt_eq1}
F_1Q + F_2Q = F_1Q + F_2^1Q \ges F_1F_2^1 = F_1P + F_2P
\end{equation}
当且仅当 $P$,$Q$ 重合时,不等式(\autoref{ConOpt_eq1})为等式.
再由椭圆的定义:平面上到两焦点的距离之和为定值的点的集合,可知直线 $PT$ 上有且仅有点 $P$ 在椭圆上,即直线 $PT$ 是椭圆的切线.得证.

\subsubsection{等价的命题}
结合光在同一介质中直线传播的性质,以及光的反射定律:反射角等于入射角,不难推知,上述命题等价于“从椭圆一个焦点处射出的光线经过在椭圆曲线上的反射后,反射光线都汇聚于另一个焦点”.

\subsection{抛物线的光学性质}
$F$ 是抛物线的焦点,$l$ 是准线,$P$ 是抛物线上的一点,作 $PP' \perp l$,垂足为 $P'$,则 $\angle FPP' $ 的角平分线 $ PT $ 是抛物线的切线.

等价命题:从抛物线焦点射出的光线,经过抛物线曲线的反射后,反射光线平行于抛物线对称轴.

\subsection{双曲线的光学性质}
$F_1$,$F_2$ 是双曲线的两个焦点,$P$ 是双曲线上的一点,则 $\angle F_1PF_2 $ 的角平分线 $ PT $ 是双曲线的切线.

等价命题:从双曲线一个焦点射出的光线,经过双曲线的反射后,反射光的反向延长线汇聚于另一个焦点.

\begin{exer}{证明题}\label{ConOpt_exe1}
仿照椭圆光学性质的证明过程,证明抛物线和双曲线的光学性质.
\end{exer}