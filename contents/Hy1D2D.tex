% 一维和二维氢原子模型势能

如果在 1D 或 2D 中直接使用 $V = A/r$ 作为势能, 会发现解不出束缚态.

所以我们一般用 $V_a = A/\sqrt{r^2 + a^2}$ 作为势能, 并且调整 $V_0$ 和 $a$, 使基态等于氢原子的基态. 这个势能相当于把 $1/r$ 势能在原点的奇点变得有限且平滑了, 且 $r\to\infty$ 时有 $V \to A/r$. 当 $a$ 越小, $V_a(r)$ 就越趋近 $V(r)$.
