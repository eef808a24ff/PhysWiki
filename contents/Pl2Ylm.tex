% 平面波的球谐展开
% 复数|平面波|球谐波|归一化|球谐展开函数|傅里叶变换

\pentry{平面波\upref{PWave}, 球谐函数\upref{SphHar}}

复数形式的平面波可以展开为\footnote{由 $Y_{l,m}^* = (-1)^m Y_{l,-m}$ 易证这里的复共轭可以加在任意一个球谐函数上.}\footnote{由 $Y_{l,m}(-\uvec k) = (-1)^l Y_{l,m}(\uvec k)$ 易得 $\bra{\bvec k}$.}
\begin{equation}\label{Pl2Ylm_eq1}
\ket{\bvec k} = \frac{1}{(2\pi)^{3/2}}\E^{\I \bvec k \vdot \bvec r} = \sum_{l=0}^{\infty} \sum_{m=-l}^l \I^l Y_{lm}^*(\uvec k) \sqrt{\frac{2}{\pi}} j_l(kr) Y_{lm}(\uvec r)
\end{equation}
可以证明, 一组正交归一的基底为(见 “径向函数的归一化\upref{FrNorm}”)
\begin{equation}
\ket{s_{l,m}(k)} = s_{l,m}(k,\bvec r) = \frac{1}{r}\sqrt{\frac{2}{\pi}}kr j_l(kr) Y_{lm}(\uvec r)
\end{equation}
平面波\autoref{Pl2Ylm_eq1} 可以表示为相同能量球谐波的线性组合\footnote{这里不作证明}
\begin{equation}\label{Pl2Ylm_eq3}
\ket{\bvec k} = \sum_{l,m}\frac{\I^l}{k} Y_{lm}^*(\uvec k)\ket{s_{l,m}(k)} 
\end{equation}

% 未完成, 这个应该放到哪里?
$\ket{s_{l,m}(k)}$ 的完备性可以表示为
\begin{equation}
\int_0^\infty \dd{k} \sum_{l,m} \ket{s_{l,m}(k)}\bra{s_{l,m}(k)} = I
\end{equation}
就像一维傅里叶变换可以表示为% 未完成: 应该在哪里介绍一下
\begin{equation}
\ket{f} = \int_{-\infty}^{\infty} \dd{k} \ket{k}\braket{k}{f}
\end{equation}
其中 $\ket{k} = \exp(\I kx)/\sqrt{2\pi}$.

特殊地, 如果平面波的方向指向极轴($\theta = 0$), 由对称性, 我们只需要 $m = 0$ 的球谐波即可组成平面波. 将 $\theta = 0$ 代入\autoref{Pl2Ylm_eq1} 得
\begin{equation}
\E^{\I kz} = \sum_{l=0}^\infty (2l+1) \I^l j_l(kr) P_l(\cos\theta)
\end{equation}

\subsection{球谐展开函数的傅里叶变换}

若三维函数具有球谐展开的形式
\begin{equation}
f(\bvec r) = \frac{1}{r}\sum_{l,m} u_{lm}(r) Y_l^m(\uvec r)
\end{equation}
要做傅里叶变换
\begin{equation}
g(\bvec k) = \braket{\bvec k}{f} =  \frac{1}{(2\pi)^{3/2}} \int f(\bvec r) \E^{-\I \bvec k \bvec r} \dd[3]{r}
\end{equation}
将\autoref{Pl2Ylm_eq3} 代入上式得
\begin{equation}
g(\bvec k) = \frac{1}{k} \sum_{l,m} g_{l,m}(k)  Y_l^m(\uvec k) 
\end{equation}
其中
\begin{equation}
g_{l,m}(k) = \sqrt{\frac{2}{\pi}} \I^{-l} \int_0^{+\infty} u_{lm}(r) kr j_l(kr) \dd{r}
\end{equation}

\begin{example}{类氢原子基态的动量谱}\label{Pl2Ylm_ex1}
类氢原子基态的波函数为(见\autoref{HWF_eq3}~\upref{HWF}, 使用原子单位)
\begin{equation}
\psi(\bvec r) = \frac{Z^{3/2}}{\sqrt\pi} \E^{-Zr}
\end{equation}
显然只有 $l = 0, m = 0$ 球谐项. 而 $Y_{0,0} = 1/\sqrt{4\pi}$, 所以径向波函数为
\begin{equation}
R_{00}(r) = 2 Z^{3/2} \E^{-Zr}
\end{equation}
所以傅里叶变换为(注意 $j_0(x) = \sin x/x$)
\begin{equation}
g(\bvec k) = \frac{\sqrt{2}}{k\pi} \int_0^\infty \E^{-r} \sin(kr) r \dd{r} = \frac{2\sqrt{2}}{\pi(k^2+1)^2}
\end{equation}
\begin{equation}
g(\bvec k) = \qty(\frac{2}{Z})^{3/2} \frac{1}{\pi(k^2/Z^2 + 1)^2}
\end{equation}
我们也可以将沿 $z$ 轴正方向的三维平面波用球坐标表示(不使用球谐函数), 再在球坐标中与波函数积分, 结果相同.
\end{example}
