%拉普拉斯—龙格—楞次矢量

\pentry{开普勒问题\upref{CelBd}}

在开普勒问题% 未完成:什么是开普勒问题?
中, 我们定义\bb{拉普拉斯—龙格—楞次矢量(Laplace-Runge-Lenz Vector)} (通常简称为 LRL 矢量)为
\begin{equation}\label{LRLvec_eq1}
\vec A = \vec p \cross \vec L - mk \uvec r
\end{equation}
其中 $\vec p$ 为质点动量, $\vec L$ 为轨道角动量, $k = GMm$, $\uvec r$ 为质点位矢 $\vec r$ 的单位矢量. 在开普勒问题% 未完成:什么是开普勒问题?
中, 可以证明 $\vec A$ 是一个守恒量.

\subsection{守恒证明}
我们下面证明 $\dot{\vec A} = 0$. 对\autoref{LRLvec_eq1} 求时间导数, 考虑到中心力场中质点角动量 $\vec L$ 守恒, 有
\begin{equation}\label{LRLvec_eq2}
\dot{\vec A} = \dot{\vec p}\cross \vec L  - mk\dot{\uvec r}
\end{equation}
其中由牛顿第二定律\upref{New3} 和万有引力定律\upref{Gravty}, 有
\begin{equation}\label{LRLvec_eq3}
\dot{\vec p} = \vec F = - \frac{k}{r^2}\uvec r
\end{equation}
又由“极坐标中单位矢量的偏导\upref{Dpol1}” 得
\begin{equation}\label{LRLvec_eq4}
\dot{\uvec r} = \pdv{\uvec r}{\theta} \dv{\theta}{t} = \dot\theta\uvec \theta
\end{equation}
最后由\autoref{CenFrc_eq4}\upref{CenFrc}, 极坐标系中的角动量等于
\begin{equation}\label{LRLvec_eq5}
\vec L = mr^2\dot \theta \uvec z
\end{equation}
将\autoref{LRLvec_eq3} 至\autoref{LRLvec_eq5} 代入\autoref{LRLvec_eq2} 得
\begin{equation}
\dot{\vec A} = -\frac{k}{r^2}\uvec r \cross (mr^2\dot\theta\uvec z) - mk\dot\theta\uvec\theta
=-mk\dot\theta (\uvec r\cross \uvec z + \uvec \theta)
= \vec 0
\end{equation}
最后一个等号成立是因为 $\uvec r\cross\uvec z = -\uvec\theta$, 可以类比直角坐标系中的 $\uvec x\cross\uvec z = -\uvec y$. 证毕.

% 未完成: LRL 矢量的模长和方向?