% 库仑波函数

\pentry{薛定谔径向方程\upref{RadSE}}
本词条使用原子单位. 参考资料一个是 Wikipedia, 另一个是 “F Morales et al 2016 J. Phys. B: At. Mol. Opt. Phys. 49 245001” 的附录. 据说 Merzbacher 的量子力学也有.

库仑波函数常用的边界条件有两种, 分别对应以 $\vec k$ 平面波入射和出射的散射态, 记为 $\psi_{\vec k}^{(+)}(\vec r)$ 和 $\psi_{\vec k}^{(-)}(\vec r)$. 两种边界条件分别为
\begin{equation}
\psi_{\vec k}^{(\pm)}(\vec r) \to \frac{1}{(2\pi)^{3/2}} \E^{\I\vec k\vdot\vec r + \eta\ln (kr - \vec k \vdot\vec r)}
\qquad
(\vec r\vdot \vec k \to \mp\infty)
\end{equation}
两种波函数为
\begin{equation}
\psi_{\vec k}^{(\pm)}(\vec r) = \frac{1}{(2\pi)^{3/2}} \Gamma(1\pm \I\eta)\E^{-\pi\eta/2} \E^{\I\vec k\vdot\vec r} {_1F_1}(\mp\I\eta; 1; \pm\I kr - \I \vec k\vdot \vec r)
\end{equation}
其中 $\rho = kr$, $\eta = Z/k$. 上式满足
\begin{equation}
\psi_{\vec k}^{(+)}(\vec r) = \psi_{-\vec k}^{(-)}(r)^*
\end{equation}
\begin{equation}
\braket*{\psi_{\vec k}^{(\pm)}}{\psi_{\vec k'}^{(\pm)}} = \delta(\vec k - \vec k')
\end{equation}

\subsection{球坐标}
库仑势的薛定谔径向方程为
\begin{equation}
-\frac12 \dv[2]{u}{r} + \qty[\frac{Z}{r} + \frac{l(l+1)}{2 r^2}]u = \frac{k^2}{2}u
\end{equation}
其中 $u(r)$ 是 Scaled 波函数, $k$ 是能量为 $E = k^2/2$ 的平面波的波矢, $Z$ 是原子核和电子的电荷之积, $l$ 是角量子数.

令 $\rho = kr$, $\eta = Z/k$, 则上式变为
\begin{equation}
\dv[2]{u}{\rho} + \qty[1 - \frac{2\eta}{\rho} - \frac{l(l+1)}{\rho^2}]u = 0
\end{equation}
两个线性无关解为\bb{第一类库仑函数} $F_l(\eta, \rho)$ 和\bb{第二类库仑函数} $G_l(\eta, \rho)$\upref{CulmF}.

所以完备正交归一的\bb{库仑球面波}为
\begin{equation}
\ket{C_{l,m}(k)} = \frac{1}{r} \sqrt{\frac{2}{\pi}} F_l(\eta, kr) Y_{l,m}(\uvec r)
\end{equation}
满足
\begin{equation}
\braket{C_{l',m'}(k')}{C_{l,m}(k)} = \delta_{l,l'}\delta_{m,m'}\delta(k-k')
\end{equation}

现在可以将库仑波函数展开为
\begin{equation}
\psi_{\vec k}^{(\pm)}(\vec r) =  \sum_{l,m} a_{l,m}^{(\pm)}(k) \ket{C_{l,m}(k)}
\end{equation}
其中
\begin{equation}% (-) 有待验证
a_{l,m}^{(\pm)}(k) = \frac{\I^l}{k} \exp[\pm\I\sigma_l(Z/k)] Y_{l,m}^* (\uvec k)
\end{equation}
注意 $a_{l,m}$ 并不是 $\braket*{C_{l,m}(k)}{\psi_{\vec k}^{(\pm)}}$, 后者模长为无穷大, 也可记为 $a_{l,m}\delta(0)$。

与平面波的情况(傅里叶变换)类似, 要将一个球谐展开的波函数 $\ket{f}$ 投影到库仑波函数上, 就先投影到库仑球面波上, 然后进行幺正变换
\begin{equation}
\braket*{\psi_{\vec k}^{(\pm)}}{f} = \sum_{l,m}  a_{l,m}^{(\pm)}(k)^* \braket{C_{l,m}(k)}{f}
\end{equation}
若 $\ket{f}$ 的 scaled 径向波函数为 $u_{l,m}(r)$, 则
\begin{equation}
\braket*{\psi_{\vec k}^{(\pm)}}{f} = \frac{1}{k} \sum_{l,m} g_{l,m}(k) Y_l^m(\uvec k)
\end{equation}
其中
\begin{equation}
g_{l,m}(k) = \sqrt{\frac{2}{\pi}} \I^{-l} \E^{\mp\I\sigma_l} \int_0^\infty F_l(\eta, kr) u_{l,m}(r) \dd{r}
\end{equation}

将波函数做傅里叶变换和投影到库仑波函数有什么区别呢? 例如做氢原子电离的 TDSE, 若想求 $t = +\infty$ 时的动量分布, 理论上只要在 $t$ 足够大时做傅里叶变换即可, 但如果 $t$ 不够大(电场已消失), 电离波包所受的库仑力还不可忽略, 那么虽然得到了瞬时的动量分布, 但却与 $t = +\infty$ 的不同. 这时因为动量算符与哈密顿算符不对易, 动量不守恒. 但若在电场消失的时候, 投影到 $\psi_{\vec k}^{(-)}$ 上, 由于它是哈密顿的本征函数, 投影的模方(概率)不会随时间改变.

最后的问题就是, 当 $t = +\infty$ 时, 投影到平面波和库仑波是否相同呢? 要验证这一点, 只需验证
\begin{equation}
\braket*{\psi_{\vec k}^{(-)}}{\vec k'} = \delta(\vec k - \vec k')
\end{equation}
我们还是用球谐展开来验证
\begin{equation}\ali{
\braket*{\psi_{\vec k}^{(\pm)}}{\vec k'} &= \sum_{l,m} \braket*{\psi_{\vec k}^{(\pm)}}{C_{l,m}(k)}\ket{C_{l,m}(k)} \sum_{l',m'} \bra{s_{l',m'}(k')}\braket{s_{l',m'}(k')}{\vec k'} \\
&= \sum_{l,m} \braket*{\psi_{\vec k}^{(\pm)}}{C_{l,m}(k)}\braket{C_{l,m}(k)}{s_{l,m}(k')}\braket{s_{l,m}(k')}{\vec k'}
}\end{equation}
但是径向积分貌似不会做…… 就当是对的吧.
