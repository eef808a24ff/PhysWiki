% 哈密顿正则方程

\pentry{拉格朗日方程\upref{Lagrng}}

哈密顿正则方程是一组以 $N$ 个广义坐标 $q_i$ 和 $N$ 个广义动量(\autoref{Lagrng_eq13}\upref{Lagrng})作为因变量的方程组, 共有 $2N$ 条方程. 与拉格朗日方程相比, 虽然方程的个数增多了, 但是方程却由二阶变为了一阶.

我们定义一个系统的哈密顿量为
\begin{equation}\label{HamCan_eq1}
H = \sum_i \dot q_i p_i - L
\end{equation}
其中 $L$ 为拉格朗日量, $p_i$ 为广义坐标 $q_i$ 的共轭动量(见).

我们先来证明哈密顿量等于系统能量. 将系统看做质点系, 由于 $L = T - V$ 且 $V$ 与 $\dot q$ 无关, 有
\begin{equation}
p_i = \pdv{L}{\dot q_i} = \pdv{\dot q_i} \sum_j \frac12 m_j \dot{\vec r}_j^2
= \sum_j m_j \dot{\vec r}_j \pdv{\dot{\vec r}_j}{\dot q_i}
\end{equation}
利用\autoref{Lagrng_eq27}\upref{Lagrng}, 有
\begin{equation}
p_i = \sum_j m_j \dot{\vec r}_j  \pdv{\vec r_j}{q_i}
\end{equation}
所以\autoref{HamCan_eq1} 中的求和项为
\begin{equation}
 \sum_i \dot q_i p_i = \sum_j m_j \dot{\vec r}_j \sum_i \pdv{\vec r_j}{q_i}\dv{q_i}{t}
= \sum_j m_j \dot{\vec r}_j^2 = 2T
\end{equation}
其中第二步用到了 $r(q_1(t), q_2(t) \dots)$ 的全微分\upref{TDiff}. 上式代回\autoref{HamCan_eq1}, 可证明 $H = T + V$ 等于系统总能量. 证毕.