% 气体分子的速度分布
% 分布函数|能均分|速率分布|麦克斯韦|平均动能

\pentry{多变量分布函数\upref{MulPdf}}
当我们描述大量气体分子的运动时, 在某时刻可以用一个速度分布函数 $f(v_x, v_y, v_z)$ 描述气体分子的速度分布, 或用矢量符号记为 $f(\bvec v)$, 也可以用球坐标表示为 $f(v, v_\theta, v_\phi)$.

我们以下考虑\textbf{各向同性}的速度分部, 即 $f(v, v_\theta, v_\phi)$ 与方向 $v_\theta, v_\phi$ 无关, 可以简写为 $f(v)$. 另外以下将分子视为质点, 不考虑其转动和振动等内部运动.

\subsection{速度的能均分定理}
我们来考虑分子的平均动能以及各个方向的平均动能之间的关系. 由于经典力学中动能为 $E_k = mv^2/2$, 我们只需要计算分子速度(以及各个分量速度)平方的平均值.

令 $v_i$ 为速度分量 $v_x, v_y, v_z$ 中的一个, 有
\begin{equation}
\ev{v_i^2} = \int v_i^2 f(v_x, v_y, v_z) \dd{v_x}\dd{v_y}\dd{v_z}
\end{equation}
由于速度分布各向同性, 有
\begin{equation}
\ev{v_x^2} = \ev{v_y^2} = \ev{v_z^2}
\end{equation}
而速度模长平方的分布为
\begin{equation}
\begin{aligned}
\ev{v^2} &= \int (v_x^2 + v_y^2 + v_z^2) f(v_x, v_y, v_z) \dd{v_x}\dd{v_y}\dd{v_z}\\
&= \sum_{i = x,y,z} \int v_i^2 f(v_x, v_y, v_z) \dd{v_x}\dd{v_y}\dd{v_z}\\
&= \sum_{i = x,y,z} \ev{v_i^2} = \frac{1}{3} \ev{v_i^2}
\end{aligned}
\end{equation}
这就说明, 分子的平均动能和各个分量之间的平均动能满足
\begin{equation}
\bar E_{kx} = \bar E_{ky} = \bar E_{kz} = \frac{1}{3} \bar E_k
\end{equation}
这相当于把总平均动能\textbf{均分}到了三个方向上, 所以称为\textbf{能均分定理}.

\subsection{速率分布}
若把速度(矢量)的模长 $v = \abs{\bvec v}$ 叫做速率, 则类比\autoref{MulPdf_eq5}~\upref{MulPdf}, 速率的分布函数为
\begin{equation}
F(v) = 4\pi v^2 f(v)
\end{equation}

令气体分子的数量为 $N$, 则随机一个分子速度绝对值落在 $[v_a, v_b]$ 范围的概率为(类比\autoref{MulPdf_eq4}~\upref{MulPdf})
\begin{equation}
P_{ab} = \int_a^b F(v) \dd{v} = \int_a^b 4\pi v^2 f(v) \dd{v}
\end{equation}
所以近似地, 该区间中的分子数量为
\begin{equation}
N_{ab} = P_{ab} N
\end{equation}

(未完成)