% 宇宙中的距离

\pentry{FRW 度规\upref{FRW}, 宇宙学红移}

\subsection{非可观测距离}
为了有利于定义可观测距离,我们首先定义非观测距离.也称为度规距离.FRW 度规可以写成以下形式
\begin{equation}
ds^2=-dt^2+a(t)^2(d\chi^2+S^2_k (\chi) d\Omega^2),
\end{equation}
其中
[xxx]. % 未完成

\textbf{度规距离}可以被定义为空间度规中立体角的因子
\begin{equation}
d_m=S_k(\chi).
\end{equation}

当取平直宇宙的情况,即$k=0$时,度规距离可退化为\textbf{共动距离(comoving distance)} $\chi$. 共动距离可以用宇宙学红移因子 $z$ 和哈勃常数 $H(z)$ 来表达, 具体可写成
\begin{equation}
\chi(z)=\int^{t_0}_{t_1} \frac{dt}{a(t)}=\int^z_0 \frac{dz}{H(z)}.
\end{equation}
这里需要强调,度规距离和共动距离都是非可观测距离.

\subsection{光度距离}
由于Type IA 超新星被认为是一个拥有绝对光度 $L$ 的星体(每秒发出恒定的能量), 所以此类星体被称作\textbf{标准烛光(standard candle)}.Type IA 超新星爆发的光通过宇宙传到地球后,地球上观测到的能流密度$F$(单位面积每秒所观测到的能量)可以用于测量距离.我们先假设某一静止Type IA 超新星, 能流 $F$ 与此超新性的共动距离 $\chi$ 的关系为
\begin{equation}
F=\frac{L}{4\pi \chi^2}.
\end{equation}
然而,由于宇宙的膨胀,在 FRW 度规中,上述关系可以被以下三个原因所修正:

1、在时间$t_0$时刻超新星的光到达地球,超新星的光穿过地球的固有面积为 $4\pi d_m^2$;

2、在膨胀的宇宙中,以地球作为参照系,超新星在往后退,所以要除以一个宇宙学红移因子 $1+z$;

3、相对于超新星,地球也在往后退,所以地球接收到的光子也会产生效应,因而需要再次除以一个宇宙学红移因子 $1+z$

所以,能流 $F$ 与距离的正确关系应为
\begin{equation}
F=\frac{L}{4\pi d_m (1+z)^2}.
\end{equation}

\subsection{角距离}