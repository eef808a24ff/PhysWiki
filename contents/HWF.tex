% 类氢原子的波函数

\pentry{波函数简介} %未完成链接
类氢原子(原子核有 $Z$ 个质子, 只有一个核外电子, 的波函数在球坐标中表示为
\begin{equation}
\psi_{nlm} (r,\theta ,\phi) = R_{nl}(r) Y_{l,m}(\theta, \phi)
\end{equation}
其中 $n$ 是\bb{主量子数}($n = 1, 2, \dots$), $l$ 是\bb{角量子数}($l = 0, 1, \dots, n - 1$), $m$ 是\bb{磁量子数}($m = -l, -l+1, \dots, l$). $R_{nl}(r)$ 是归一化的\bb{径向波函数}, $Y_{l,m}(\theta, \phi)$ 是归一化的\bb{球谐函数}(见“球谐函数列表\upref{YlmTab}”).

如果忽略原子核的运动, 以下的 $a$ 是玻尔半径, 如果不忽略, 将 $a$ 就是约化玻尔半径.% 两个链接未完成  

\subsubsection{径向波函数 $R_{nl}(r)$}

注意 $Z$ 和 $a$ 的作用是把径向波函数关于原点收缩 $Z/a$ 倍(并保持波函数归一化).
\begin{equation}\upref{HWF_eq2}
R_{nl}(r) = \sqrt{\qty(\frac{2 Z}{na})^3 \frac{(n - l - 1)!}{2n (n + l)!}} \E^{-Zr/(na)} \qty(\frac{2Zr}{na})^l  L_{n-l-1}^{2l+1}\qty(\frac{2Zr}{na})
\end{equation}
其中 $L_n^l(x)$ 是\bb{连带拉盖尔多项式(associated Laguerre polynomial)}. 以下给出前几个径向波函数

\begin{equation}
n = 1 \qquad
R_{10}(r) = 2\qty(\frac{Z}{a})^{3/2}\exp(-Zr/a)
\end{equation}
\begin{equation}
n = 2 \qquad
\leftgroup{
R_{20}(r) &= \frac{1}{\sqrt 2} \qty(\frac{Z}{a})^{3/2} \qty(1 - \frac12 \frac{Zr}{a}) \exp(-\frac{Zr}{2a})\\
R_{21}(r) &= \frac{1}{\sqrt{24}} \qty(\frac{Z}{a})^{3/2} \frac{Zr}{a} \exp(-\frac{Zr}{2a})
}\end{equation}
\begin{equation}
n = 3 \qquad
\leftgroup{
R_{30}(r) &= \frac{2}{\sqrt {27}} \qty(\frac{Z}{a})^{3/2} \qty(1 - \frac23 \frac{Zr}{a} + \frac{2}{27} \frac{Z^2r^2}{a^2}) \exp(-\frac{Zr}{3a})\\
R_{31}(r) &= \frac{8}{27\sqrt 6} \qty(\frac{Z}{a})^{3/2} \qty(1 - \frac16 \frac {Zr}{a}) \frac {Zr}{a} \exp(-\frac{Zr}{3a})\\
R_{32}(r) &= \frac{4}{81\sqrt {30}} \qty(\frac{Z}{a})^{3/2} \frac{Z^2r^2}{a^2} \exp(-\frac{Zr}{3a})}
\end{equation}
