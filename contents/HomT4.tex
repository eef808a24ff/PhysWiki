% 高阶同伦群
\pentry{基本群\upref{HomT3}}

%未完成
%应该是需要大量的图示来直观展示

基本群的符号被定义为$\pi_1$.你可能好奇为什么要有一个下标$1$——这是因为基本群只是同伦群中的一种,而仅仅用$\pi_1$群来描述一个拓扑空间往往是不够的.

考虑三个拓扑空间:二维球面(如地球表面)$S^2$,挖去中心的三维球体(如挖去地心的地球)$D^3-x_0$以及$n$维几何空间$\mathbb{R}^n$.容易看出,这三个空间中的任意道路都彼此同伦,因此对任何基点来说都只存在一个回路类,意味着它们的基本群都是平凡群(只有一个元素).这样一来,对于空间中有几个洞、空间的维度是什么等信息就丢失了,我们就需要推广基本群的概念来描述这些性质.这就是高阶同伦群.

\subsection{球和球面}
在$n$维欧几里得空间中,集合$\{(x_1, x_2,\cdots,x_n)|\sum^n_{i=1}x_i^2\leq 1\}$也被称为半径为$1$的$n$维球体,记为$B^n$;集合$\{x_1, x_2, \cdots, x_n|\sum^n_{i=1}x_i^2=1\}$是$B^n$的表面,记为$\partial B^n$,或$S^{n-1}$.记号$\partial B^n$表达的是“$B^n$的边界\upref{Topo0}”,而$S^{n-1}$表达的是“$n$维球面”.

更一般地,所有和球同胚的拓扑空间都被看成球,所有和球面同胚的都被看成球面.比如说,$n$维空间里的“立方体”,$\{(x_1, \cdots, x_n|x_i\in[0, 1],\forall i=1, \cdots, n)\}$,都可以被看成是球.

\begin{theorem}{球面粘接}
对于任意的正整数$n$,把$B^n$表示为一个立方体$\{(x_1, \cdots, x_n|x_i\in[0, 1],\forall i=1, \cdots, n)\}$.$B^n$显然是$\mathbb{R}^n$的子空间.在$B^n$上取一个等价关系$\sim$,使得$B^n$边界上的点都等价,其它点则只和自己等价.则商空间$B^n/\sim$同胚于$S^n$.
\end{theorem}

举个例子,一张平面桌布就是一个$B^2$空间,桌布的边缘是$S^1$;如果把桌布边缘粘合成一个点,那么所得的商空间就是$S^2$.

\subsection{同伦群}

基本群是用保基点回路类定义的.由于道路都是区间$I$到拓扑空间的嵌入映射,而回路是首尾相连的,因此一条回路可以看成是$S^1$到拓扑空间的嵌入映射.这时,我们也可以把基本群称作$1$\textbf{阶同伦群}.

类似地,如果把$S^n$嵌入到拓扑空间中,并将道路的首尾相连运算推广到$n$维情况,那么我们还可以得到$n$\textbf{阶同伦群}的定义.

\subsubsection{1阶道路的首尾相连运算}



