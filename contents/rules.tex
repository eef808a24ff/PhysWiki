% 编写规范

\subsection{编辑器}

\subsubsection{TeXworks}

\textbf{设置}: 菜单中的 Edit > Preference 设置默认字体为 Microsoft YaHei UI(11pt), 默认编译器为 XeLaTeX,编码选择 UTF-8.

\textbf{快捷键}:
编辑器中 Ctrl+T 编译, 
Ctrl+单击跳转到对应的 pdf 或代码, 
在 pdf 中 Alt+左箭头返回上一个位置, 
代码中 \lstinline|\beq|+Tab 生成公式环境,\lstinline|\sub|+Tab 生成 subsection, Ctrl+G 查找下一个.

\subsubsection{TeXstudio}
\textbf{设置}: 安装 TeXlive 再安装 texstudio 即可使用, 无需任何配置.
general 中设置界面字体为 9.
build 中设置默认编译器为 XeLaTeX.
editor 中设置字体为 Microsoft YaHei, 大小 11.
Inline Checking 关掉.

\textbf{快捷键}:
F5 编译并预览 pdf, 若没有 pdf 预览, 或者右键 go to pdf 即可显示 pdf, Ctrl + 单击可以跳转到 pdf 或者代码.

\subsection{其他软件使用规范}
本书使用 TeXLive2017 软件中的 XeLaTeX 进行编译(暂不兼容更高版本).如果 Windows 中编译卡在 eu1lmr.fd 上的时间较长,说明 font config 有问题,在 Windows 的控制行运行“fc-cache -fv”,重启 TeXLive,多试几次即可.

搜索文件夹内所有文档的内容用 FileSeek 软件,搜索空格用“\lstinline|\ 空格|”,搜索“\$”用“\lstinline|\$|”,以此类推.

画图用 Adobe Illustrator, 用知乎的公式编辑器在图中添加公式(把编辑器中的公式另存为 svg 然后在 Illustrator 中打开)\footnote{不推荐的老方法是使用 MathType 添加公式,希腊字母粗体正体矢量用从 Symbol 字体中插入(勾选 bold),更简单的方法是,先输入希腊字母,选中,然后在 Style 里面选 Vector-Matrix}. 图片中的文字必须是 12 号, 如果字太小, 就把图片缩小而不是字放大. 图片中的线条尽量用 1pt 粗细. 图片插入书中后, 图中的字体应比书中的略小. 要画箭头, 先画一条直线, 然后选画笔图标, 在左下角的菜单中选 Arrows > Arrow\_Standard, 选 1.23 号箭头, 再把粗细改成 0.3155pt (相当于 1pt 粗的直线).

\subsection{文件版本管理}
使用 GitHub Desktop, 用 MacroUniverse/PhysWiki 项目管理所有文件, 每次 commit 必须完成以下步骤.
\begin{itemize}
\item 与 GitHub 同步(fetch/pull)
\item 检查变化的内容
\item 用 FileSeek 查找所有文档中的空心句号并替换
\item 确保所有文档可以顺利编译
\item 用 PhysWikiScan 更新 littleshi.cn
\item commit 以后检查 history 无误后 push 到 GitHub
\end{itemize}
每次 commit 的标题尽量使用下列之一
\begin{itemize}
\item 常规更新:包括完善词条,新词条等.
\item 模板更新:模板有更新.
\item 批量修改:在多个文件中修改某一格式规范, 这种修改比较危险, 需要谨慎.
\item 词条统计:统计文件夹,对照表,和书中的词条,查看不一致或缺失.
\end{itemize}
定期检查的内容
\begin{itemize}
\item 解决编译产生的 warning
\item 把 ManicTime 记录的写作时间记录到“timer.xls”
\end{itemize}

GitHub 会忽略 “.gitignore” 文件指定的文件类型: *.toc, *.aux, *.log, *.out, *synctex.gz.

词条统计的方法: 首先把 contents 文件夹中的所有文件名按顺序排列, 复制到表格中, 然后把词条对照表中的所有标签在表格中找到对应项, 做标记, 并把对照表中的词条名粘贴到表格中. 最后到 PhysWiki.tex 中逐个把标签在表格中找到对应项, 做标记, 对照词条名, 并对照词条文件中第一行的词条名.

\subsection{词条编写规范}

每个词条文件必须有一个独一无二的标签(即使在不同文件夹中), 词条标签必须限制在 6 个字符内,必须在 PhysWiki.tex, 词条标签对照表和词条文件名中一致. 词条的中文名必须在主文件, 词条标签对照表和词条文件的第一行注释中一致. 中文名中空格用 “\textbackslash 空格” 实现, 不能出现公式环境(尽量用英文单词代替比如 gamma 代替 $\Gamma$). 词条文件一般放在 contents 目录下, 并在主文件中用 \lstinline|\entry{}{}| 命令输入中文名和标签. 主文件有 Debug.tex, PhysWiki.tex 和 PhysWikiNote.tex 三个, 它们共用一个模板(others 目录). 新词条必须现在 Debug.tex 中编辑, 完成后再从中删除并将 entry 插入到 PhysWiki.tex (小时物理百科)或 PhysWikiNote.tex (小时物理笔记) 中. Debug.tex 中的 entry/Entry 命令的后面可以用 \lstinline|\newpage| 命令强制换页, 但不允许在其他文件中这么做.

PhysWiki.tex 中与 PhysWiki1.tex 重复的部分不能修改, 只能从 PhysWiki1.tex 中复制. PhysWiki.tex 中与 PhysWikiNotes.tex 中重复的部分也不能修改, 只能从 PhysWikiNotes.tex 中复制.

PhysWiki.tex 中已有的词条只允许放在 Debug.tex 的“修改审阅中” 部分, 否则必须放在“创作中” 部分.

引用词条用 \lstinline|\upref| 命令,“预备知识”用 \lstinline|\pentry| 命令,“应用实例”用 \lstinline|\eentry| 命令.

\subsection{错别字替换}
可以时常搜索替换: “一下”(以下), “这是” (这时), “收到”(受到), “符号”(负号), “带入”(代入).

