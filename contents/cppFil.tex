% C/C++ 多文件编译笔记

\begin{itemize}
\item include 头文件相当于直接把头文件的内容插入源文件(.c 或 .cpp)中, 头文件可以没有或者有任意后缀名(包括 .c 和 .cpp)。 所以下文我们说 “源文件中” 也包括源文件 include 的头文件中。
\item 一个源文件编译成一个 object 文件(.o), 最后将所有 object 文件 link 起来就成了可执行程序。
\item 一个源文件中定义的函数要在另一个源文件调用前要先声明(declare)。
\item 每个源文件可以单独编译成 object 文件, 即使调用了其他源文件中定义的函数, 编译的时候也不需要有其他源文件存在。
\item 只有在 link 阶段编译器才会检查一个 o 文件中调用的函数是否能在另一个 o 文件中找到。
\item 某个源文件中定义的宏只在这一个源文件中有效, 即从定义开始到文件底部(或者到 undef)。
\item ODR (one definition rule) 要求一个函数不能在同一个或不同源文件中有重复的定义(definition) 不是声明(declaration)。
\item inline 的函数/变量需要在使用到它的每个源文件中都有一样的定义(所以一般放到头文件中)。
\item class 的声明也需要在使用到它的每个源文件中都有一样的定义(所以一般放到头文件中)。
\item inline 的含义已经与是否真的 inline 几乎无关了, 只是为了避免 ODR 错误的一种手段。
\end{itemize}
