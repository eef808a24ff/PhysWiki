% 刚体的运动方程

\pentry{刚体的平面运动方程\upref{RBEM}}

一般情况下下刚体的运动方程要比平面运动复杂得多, 但我们仍然可以将运动分解为质心的运动以及刚体绕质心的旋转, 前者由合力决定, 所以仍然有(\autoref{RBEM_eq1}\upref{RBEM})
\begin{equation}
M\vec a_c = \sum_i \vec F_i
\end{equation}
所以相对于平面运动, 该问题的困难在于绕质心转动的计算. 虽然角动量定理仍然满足, 但转动惯量将有可能随时间变化. 这是因为刚体有可能沿不同的转轴转动, 而不同的转轴对应的角动量一般不同.

% 未完成: 这个话题我真的不熟啊. 瞬时转轴, 
% 能否先用的思想来说明? 就是把时间分成许多 \Delta t, 每个 dt 假设某些量为常数, 等等
% 陀螺问题: 之前的假设是转动惯量很大, 但事实上陀螺是除了进动还会张动的.

\subsection{平动动能与转动动能}
\pentry{柯尼西定理\upref{Konig}}
由柯尼西定理, 刚体的动能可以分为\bb{平动动能}和\bb{转动动能}两部分, 其中
