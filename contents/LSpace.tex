% 矢量空间
% 线性代数|矢量|几何矢量|矢量空间|线性空间|集合|交换律|结合律|分配率|多项式|线性相关|线性无关|基底|n维空间|行矢量|列矢量|子空间|内积

\pentry{几何矢量\upref{GVec}}

\textbf{矢量空间}(也叫\textbf{向量空间},\textbf{线性空间})是一类满足特定条件的\textbf{集合}, 集合内的元素都称为\textbf{矢量}.这里的“矢量” 是一个广义的概念, 不一定具有长度和方向. 高中所学的几何矢量是这些广义矢量的一个特例.

\subsection{定义}
\textbf{矢量空间}是满足特定条件的集合, 集合中有无穷多个元素, 每个元素叫做一个\textbf{矢量}. 任意矢量空间内必须定义两个矢量的\textbf{加法}(用 “+” 表示)和常数与矢量之间的\textbf{数乘} 两种运算, 得到的结果也必须在同一空间中. 我们把这样的运算叫做\textbf{闭合}的. 两种运算的必须满足如下性质, 其中 $\bvec u,\bvec v,\bvec w$ 为空间中任意三个矢量, $a,b$ 为任意两个数. 若限制 $a, b$ 只能取实数, 那么我们把这个矢量空间称为\textbf{实数矢量空间}, 若 $a, b$ 可以取任意复数, 则称为\textbf{复数矢量空间}. 为了与矢量区分, 我们称实数或复数为\textbf{标量}.

\subsubsection{加法运算}
\begin{enumerate}
\item 满足加法交换律 $\bvec u + \bvec v = \bvec v + \bvec u$.
\item 满足加法结合律 $(\bvec u + \bvec v) + \bvec w = \bvec u + (\bvec v + \bvec w)$.
\item 存在零矢量,使得 $\bvec v + \bvec 0 = \bvec v$.
\item 空间中任何矢量 $\bvec v$ 存在逆矢量 $-\bvec v$,使得 $\bvec v + (-\bvec v) = \bvec 0$.
\end{enumerate}

\subsubsection{数乘运算}
\begin{enumerate}
\item 乘法分配律 $a(\bvec u + \bvec v) = a\bvec u + a\bvec v$ 
\item 乘法分配律 $(a + b)\bvec v = a\bvec v + b\bvec v$
\item 乘法结合律 $a (b \bvec v) = (ab) \bvec v$
\end{enumerate}

注意这里所说的矢量是一种抽象的矢量, 而不必是高中所学的有方向和长度的箭头. 例如多项式就可以构成矢量空间.

\begin{example}{多项式}\label{LSpace_ex1}
所有不大于 $n$ 阶的多项式 $c_n x^n + c_{n-1} x^{n-1} + \dots + c_1 x + c_0$ 可以构成一个实数矢量空间或复数矢量空间.定义矢量加法为两多项式相加, 满足
\begin{itemize}
\item 闭合性:两个不大于 $n$ 阶的多项式相加仍然为不大于 $n$ 阶的多项式.
\item 交换律:多项式相加显然满足交换律.
\item 零矢量:常数 0 可以看做 0 阶多项式, 任何多项式与之相加都不改变.
\item 逆矢量:把任意多项式乘以 $-1$ 就得到它的逆矢量, 任意多项式与其逆矢量相加等于 0.
\end{itemize}
定义矢量数乘为多项式乘以常数, 显然也满足数乘的各项要求, 不再赘述.
\end{example}

\begin{exercise}{几何矢量}
证明 1,2,3 维空间中的所有几何矢量各自构成一个实数矢量空间.
\end{exercise}

注意矢量空间的定义并不需要包含内积(点乘) 的概念, 但我们可以在其基础上额外定义内积, 这样的空间叫做\textbf{内积空间}, 留到以后介绍.

除了内积, 我们可以把 “几何矢量\upref{GVec}” 中介绍的概念都拓展到一般的矢量空间中.

\subsection{基底和维数}

如果一个矢量空间中存在 $N$ 个线性无关的非零矢量 $\bvec \beta_1, \bvec \beta_2 \dots \bvec \beta_N$(记为 $\{\bvec \beta_i\}$),使得空间中的任意矢量都可以用它们的线性组合表示(系数可以为零),那么 $\{\bvec \beta_i\}$ 就是这个空间的一组\textbf{基底}, 且这个空间是一个 $\boldsymbol{N}$ \textbf{维空间}.

可以证明, $N$ 维矢量空间中任意 $N$ 个线性无关的矢量都可以构成一组基底, 且空间中的任意矢量都能唯一地在基底上展开(见\autoref{GVec_eq5}\upref{GVec}), 而展开系数就是\textbf{坐标}(再次注意, 复数矢量空间中, 矢量的坐标可以是复数).
\begin{equation}
\bvec v = \sum_{i=1}^N c_i \bvec \beta_i
\end{equation}

\begin{exercise}{}
三维几何矢量空间中, 建立直角坐标系, 基底为 $\uvec x, \uvec y, \uvec z$. 请证明直角坐标(即关于基底 $\uvec x, \uvec y, \uvec z$ 的坐标)为 $(2, 1, 1)$, $(1, 3, 1)$, $(1, 1, 4)$ 的三个矢量线性无关, 并用这三个矢量作为基底, 求直角坐标为 $(1, 1, 1)$ 的矢量关于这组基底的坐标.
\end{exercise}

有了坐标的概念, 矢量的加法和数乘运算就都可以对应到坐标的运算. 对于 $N$ 维矢量空间(类比\autoref{GVec_eq8}\upref{GVec} 和 \autoref{GVec_eq9}\upref{GVec}).
\begin{equation}
\bvec u + \bvec v = \pmat{u_1\\u_2\\ \vdots\\u_N}_{\{\bvec\beta_i\}} + \quad \pmat{v_1\\v_2\\ \vdots \\v_N}_{\{\bvec\beta_i\}} = \quad \pmat{u_1 + v_1\\u_2 + v_2\\ \vdots \\ u_N + v_N}_{\{\bvec\beta_i\}}
\end{equation}
\begin{equation}
\lambda \bvec v = \lambda\pmat{v_1\\v_2\\ \vdots \\v_N}_{\{\bvec\beta_i\}} = \pmat{\lambda v_1 \\ \lambda v_2\\ \vdots \\ \lambda v_N}_{\{\bvec\beta_i\}}
\end{equation}

\begin{exercise}{复数行矢量}
我们把 $N$ 个复数 $c_1, \dots, c_N$ 按顺序排成一行(或一列), 叫做\textbf{行矢量}(或\textbf{列矢量}). 给它们定义恰当的加法和数乘运算, 使所有行矢量(或列矢量)可以构成一个 $N$ 维矢量空间, 并给出该空间的一组基底.

注意由于我们使用了复数, 即使 $N \leqslant 3$ 时我们也无法将这些代数矢量与几何矢量对应起来.
\end{exercise}
