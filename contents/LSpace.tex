% 矢量空间

\pentry{几何矢量\upref{GVec}}

\bb{矢量空间}(也叫\bb{向量空间},\bb{线性空间})是一类满足特定条件的\bb{集合}, 集合内的元素都称为\bb{矢量}.这里的“矢量” 是一个广义的概念, 不一定具有长度和方向. 高中所学的几何矢量是这些广义矢量的一个特例.

\subsection{定义}
任意矢量空间内必须定义矢量的\bb{加法}(用“+”表示)和\bb{数乘} 两种运算,可作用于空间中(集合内)的任意矢量, 得到的结果也必须在同一空间中. 我们把这样的运算叫做\bb{闭合}运算. 两种运算的必须满足如下性质( $a,b$ 为任意两个实数,$\vec u,\vec v,\vec w$ 为空间中任意三个矢量)

\subsubsection{加法运算}
\begin{enumerate}
\item 满足加法交换律 $\vec u + \vec v = \vec v + \vec u$.
\item 满足加法结合律 $(\vec u + \vec v) + \vec w = \vec u + (\vec v + \vec w)$.
\item 存在零矢量,使得 $\vec v + \vec 0 = \vec v$.
\item 空间中任何矢量 $\vec v$ 存在逆矢量 $-\vec v$,使得 $\vec v + (-\vec v) = \vec 0$.
\end{enumerate}

\subsubsection{数乘运算}
\begin{enumerate}
\item 乘法分配律 $a(\vec u + \vec v) = a\vec u + a\vec v$ 
\item 乘法分配律 $(a + b)\vec v = a\vec v + b\vec v$
\item 乘法结合律 $a (b \vec v) = (ab) \vec v$
\end{enumerate}

\begin{exam}{多项式}\label{LSpace_ex1}
所有不大于 $n$ 阶的多项式 $c_n x^n + c_{n-1} x^{n-1} + \dots + c_1 x + c_0$ 可以构成一个矢量空间.定义矢量加法为两多项式相加, 满足
\begin{itemize}
\item 闭合性:两个不大于 $n$ 阶的多项式相加仍然为不大于 $n$ 阶的多项式.
\item 交换律:多项式相加显然满足交换律.
\item 零矢量:常数 0 可以看做 0 阶多项式, 任何多项式与之相加都不改变.
\item 逆矢量:把任意多项式乘以 $-1$ 就得到它的逆矢量, 任意多项式与其逆矢量相加等于 0.
\end{itemize}
定义矢量数乘为多项式乘以常数, 显然也满足数乘的各项要求, 不再赘述.
\end{exam}

\begin{exer}{几何矢量}
证明 1,2,3 维空间中的所有几何矢量各自构成一个矢量空间.
\end{exer}

我们可以把“几何矢量\upref{GVec}” 中的所有不涉及\bb{内积}(点乘)的概念都拓展到矢量空间中.

\subsection{线性相关和线性无关}
如果存在至少一组不全为零系数 $c_i$ 使几个矢量的线性组合等于零, 这些矢量就被称为\bb{线性相关}的
\begin{equation}
\sum_i^N c_i \vec v_i = \vec 0
\end{equation}
对于任何一个不为零的项 $j$, 矢量 $\vec v_j$ 都可以表示为其他矢量的线性组合. 只需把上式除以 $c_j$ 即可
\begin{equation}
\vec v_j = \sum_{i \ne j}\frac{c_i}{c_j} \vec v_i
\end{equation}
如果不存在这样的系数, 这些矢量就是\bb{线性无关}的. 

\subsection{基底和维数}
如果一个矢量空间中存在 $N$ 个线性无关的非零矢量 $\vec \beta_1, \vec \beta_2 \dots \vec \beta_N$(记为 $\{\vec \beta_i\}$),使得空间中的任意矢量都可以用它们的线性组合表示(系数可以为零),那么 $\{\vec \beta_i\}$ 就是这个空间的一组\bb{基底}, 且这个空间是一个 $\boldsymbol{N}$ \bb{维空间}.

可以证明, $N$ 维矢量空间中任意 $N$ 个线性无关的矢量都可以构成一组基底, 且空间中的任意矢量都能唯一地在基底上展开(见\autoref{GVec_eq5}\upref{GVec})
\begin{equation}
\vec v = \sum_{i=1}^N c_i \vec \beta_i
\end{equation}

有了坐标的概念, 矢量的加法和数乘运算就都可以对应到坐标的运算. 对于 $N$ 维矢量空间(类比\autoref{GVec_eq8}\upref{GVec} 和 \autoref{GVec_eq9}\upref{GVec}).
\begin{equation}
\vec u + \vec v = \pmat{u_1\\u_2\\ \vdots\\u_N}_{\{\vec\beta_i\}} + \quad \pmat{v_1\\v_2\\ \vdots \\v_N}_{\{\vec\beta_i\}} = \quad \pmat{u_1 + v_1\\u_2 + v_2\\ \vdots \\ u_N + v_N}_{\{\vec\beta_i\}}
\end{equation}
\begin{equation}
\lambda \vec v = \lambda\pmat{v_1\\v_2\\ \vdots \\v_N}_{\{\vec\beta_i\}} = \pmat{\lambda v_1 \\ \lambda v_2\\ \vdots \\ \lambda v_N}_{\{\vec\beta_i\}}
\end{equation}

\begin{exer}{复数行矢量}
我们把 $N$ 个复数 $c_1, \dots, c_N$ 按顺序排成一行(或一列), 叫做\bb{行矢量}(或\bb{列矢量}). 给它们定义恰当的加法和数乘运算, 使所有行矢量(或列矢量)可以构成一个 $N$ 维矢量空间, 并给出该空间的一组基底.

注意由于我们使用了复数, 即使 $N \les 3$ 时我们也无法将这些代数矢量与几何矢量对应起来.
\end{exer}

\subsection{子空间}
如果一个矢量空间中的所有矢量都在另一个矢量空间内, 且量空间中加法和数乘运算的定义相同. 那么前者就是后者的子空间. 注意所有矢量空间都是它本身的子空间.

% 几个矢量\bb{张成}的空间是子空间, 其维度等于这些矢量中线性无关的个数.

% 未完成

\subsection{点乘}
原则上点乘运算不是矢量空间所必须的,但物理中的矢量空间几乎都定义了点乘运算.不同的空间有不同的定义两个矢量点乘
% 未完成

