% 二体系统

\pentry{质心\ 质心系\upref{CM}, 自由度\upref{DoF}}

我们现在考虑两个仅受相互作用的质点 $A$ 和 $B$, 它们的质量分别为 $m_A$ 和 $m_B$. 由于不受系统外力, 在任何惯性系中它们的质心都会做匀速直线运动.%(未完成,引用源)

现在定义它们的\bb{相对位矢}(也叫\bb{相对坐标})为点 $A$ 指向点 $B$ 的矢量
\begin{equation}\label{TwoBD_eq1}
\vec R = \vec r_B - \vec r_A
\end{equation}
且定义\bb{相对速度}和\bb{相对加速度}分别为 $\vec R$ 的导数 $\dot{\vec R}$ 和二阶导数 $\ddot{\vec R}$.
在质心系中观察, 由于质心始终处于原点, 两质点的位矢 $\vec r_A$ 和 $\vec r_B$ 满足
\begin{equation}\label{TwoBD_eq2}
m_A \vec r_A + m_B \vec r_B = \vec 0
\end{equation}
联立\autoref{TwoBD_eq1} 和\autoref{TwoBD_eq2} 可以发现在质心系中 $\vec R, \vec r_A, \vec r_B$ 间始终存在一一对应的关系, 所以质心系中不受外力的二体系统只有三个自由度\upref{DoF}
\begin{equation}\label{TwoBD_eq3}
\vec r_A = \frac{-m_B}{m_A + m_B} \vec R \qquad \vec r_B = \frac{m_A}{m_A + m_B} \vec R
\end{equation}

\subsection{运动方程}

现在令质点 $A$ 对 $B$ 的作用力为 $\vec F$ (与 $\vec R$ 同向), 则由牛顿第三定律, $B$ 对 $A$ 有反作用力 $- \vec F$. 两质点加速度分别为(牛顿第二定律) $\vec a_A =  -\vec F/m_A$, $\vec a_B =  \vec F/m_B$. 所以相对加速度为
\begin{equation}
\ddot{\vec R} = \ddot{\vec r}_B - \ddot{\vec r}_A = \frac{m_A+m_B}{m_Am_B} \vec F
\end{equation}
若定义两质点的\bb{约化质量}为
\begin{equation}
\mu = \frac{m_A m_B}{m_A + m_B}
\end{equation}
且将上式两边同乘约化质量, 我们得到相对位矢的牛顿第二定律
\begin{equation}\label{TwoBD_eq6}
\vec F = \mu\ddot{\vec R}
\end{equation}
也就是说, 在质心系中使用相对位矢, 二体系统的运动规律就相当于单个质量为 $\mu$, 位矢为 $\vec R$ 的质点的运动规律, 我们姑且将其称为\bb{等效质点}. 而 $A$ 对 $B$ 的作用力可以看成等效质点的受力.

\subsection{机械能守恒}

再来看系统的动能.使用\autoref{TwoBD_eq3} 把系统在质心系中的总动能用相对位矢表示得
\begin{equation}
E_k = \frac12 (m_A \dot{\vec r}_A^2 + m_B \dot{\vec r}_B^2) = \frac12 \frac{m_A m_B}{m_A + m_B} \dot{\vec  R}^2 = \frac12 \mu \dot{\vec  R}^2
\end{equation}
这恰好是等效质点动能.

若两质点间的相互作用力的大小只是二者距离 $R = \abs{\vec R}$ 的函数, 我们可以用一个标量函数 $F(R)$ 来表示力与距离的关系, 即
\begin{equation}
\vec F(\vec R) = F(R) \uvec R
\end{equation}
注意 $F(R)>0$ 时两质点存在斥力, $F(R)<0$ 时存在引力.

根据“势能\upref{V}” 中的\autoref{V_eq20}, 我们可以定义势能函数 $V(R)$ 为 $F(R)$ 的一个负原函数. 现在写出二体系统在质心系中的机械能为
\begin{equation}
E = \frac12 \mu \dot{\vec  R}^2 + V(R)
\end{equation}
由于系统不受外力, 机械能守恒.