% 氦原子波函数数值分析

\pentry{氦原子数值解 TDSE\upref{HeTDSE}, 库仑波函数\upref{CulmWf}}

\subsection{平均能量}
用于基态能量估计

\subsection{动量谱(6 维)}
氢原子的散射态是库仑波函数\upref{CulmWf}, 而氦原子的(双电子)散射态没有解析解, 于是就假设时间足够大时, 两电子间没有相互作用, 就可以直接把两个库仑波函数做张量积再对称/反对称化即可. 而这样一来, 散射态和束缚态之间显然就不正交了, 所以分析时需要将所有束缚态去掉. 爱华的方法是直接减去基态的投影, 然而这样一来激发态的分量就仍然存在了. 我认为如果电离波包和未电离的波包已经分离的话, 可以像做氢原子那样直接把一定半径内的波函数直接挖掉.

库仑波函数(我们只需要平面波出射)为($\eta = Z/k$, $Z = -2$)
\begin{equation}
\psi_{\vec k}(\vec r) = \sqrt{\frac{2}{\pi}}\sum_{l, m} \frac{\I^l}{r k} \E^{-\I \sigma_l(\eta)} F_l (\eta, kr) Y_{l, m}^* (\uvec k) Y_{l, m}(\uvec r)
\end{equation}
对称化后的散射态为\footnote{注意对称化的时候选择交换 $\vec r$ 还是交换 $\vec k$ 会导致结果看起来不同(实际上是等效的), 为了稍后积分方便, 我们选择交换 $\vec k$}
\begin{equation}\ali{
& \qquad \psi_{\vec k_1, \vec k_2}(\vec r_1, \vec r_2) = \frac{1}{\sqrt 2} [\psi_{\vec k_1} (\vec r_1) \psi_{\vec k_2} (\vec r_2) + \psi_{\vec k_2} (\vec r_1) \psi_{\vec k_1} (\vec r_2)]\\
&= \frac{\sqrt 2}{\pi} \sum_{l_1, m_1} \sum_{l_2, m_2} \frac{\I^{l_1 + l_2}}{r_1 r_2 k_1 k_2} \times\\
&\qquad [(\E^{-\I [\sigma_{l_1}(\eta_1) + \sigma_{l_2} (\eta_2)]} F_{l_1} (\eta_1, k_1 r_1) F_{l_2}(\eta_2, k_2 r_2) Y_{l_1, m_1}^* (\uvec k_1) Y_{l_2, m_2}^* (\uvec k_2)) +\\
& \qquad (\E^{-\I [\sigma_{l_1}(\eta_2) + \sigma_{l_2} (\eta_1)]} F_{l_1} (\eta_2, k_2 r_1) F_{l_2}(\eta_1, k_1 r_2) Y_{l_1, m_1}^* (\uvec k_2) Y_{l_2, m_2}^* (\uvec k_1))]\times\\
& \qquad Y_{l_1, m_1} (\uvec r_1) Y_{l_2, m_2} (\uvec r_2)
}\end{equation}

TDSE 的波函数(假设已经满足交换对称)为
\begin{equation}\ali{
&\Psi(\vec r_1, \vec r_2) = \sum_{L, M, l_1, l_2}  \frac{1}{r_1 r_2} \psi_{l_1, l_2}^{L, M}(r_1, r_2)\mathcal{Y}_{l_1, l_2}^{L, M}(\uvec r_1, \uvec r_2)\\
& \quad = \sum_{L, M, l_1, l_2} \frac{1}{r_1 r_2} \psi_{l_1, l_2}^{L, M}(r_1, r_2) \sum_{m_1, m_2} \bmat{l_1 & l_2 & L\\ m_1 & m_2 & M} Y_{l_1, m_1} (\uvec r_1) Y_{l_2, m_2} (\uvec r_2)
}\end{equation}
投影得(过程略, 注意两个 $Y_{l_1, m_1} (\uvec r_1) Y_{l_2, m_2} (\uvec r_2)$ 项积分会产生两个 $\delta$ 函数)
% 未完成: 未验证, 且与爱华的代码不同
\begin{equation}\ali{
\braket{\psi_{\vec k_1, \vec k_2}}{\Psi} &= \frac{\sqrt 2}{\pi k_1 k_2} \sum_{L, M, l_1, l_2} \I^{l_1 + l_2} \times \\
&[\E^{\I [\sigma_{l_1}(\eta_1) + \sigma_{l_2} (\eta_2)]} I_{l_1, l_2}^{L, M}(k_1, k_2) \mathcal{Y}_{l_1, l_2}^{L, M}(\uvec k_1, \uvec k_2) + \\
&\E^{\I [\sigma_{l_1}(\eta_2) + \sigma_{l_2} (\eta_1)]} I_{l_1, l_2}^{L, M}(k_2, k_1) \mathcal{Y}_{l_1, l_2}^{L, M}(\uvec k_2, \uvec k_1)]
}\end{equation}
其中
\begin{equation}
I_{l_1, l_2}^{L, M}(k_1, k_2) = \int_0^\infty \int_0^\infty F_{l_1} (\eta_1, k_1 r_1) F_{l_2}(\eta_2, k_2 r_2)  \psi_{l_1, l_2}^{L, M}(r_1, r_2) \dd{r_1} \dd{r_2}
\end{equation}

投影的平方就是概率分布
\begin{equation}
P(\vec k_1, \vec k_2) = \abs{\braket{\psi_{\vec k_1, \vec k_2}}{\Psi}}^2
\end{equation}

\subsection{JAD(joint angular distribution)}
\begin{equation}
\int_0^\infty \int_0^\infty P(\vec k_1, \vec k_2) k_1^2 k_2^2 \delta(k_1 - k_2) \dd{k_1} \dd{k_2}
\end{equation}
