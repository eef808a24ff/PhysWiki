% 厄米矩阵的本征问题
% 线性代数|厄米矩阵|本征值|正交归一|本征矢|本征方程
% 未完成: 先把这个词条完成, 然后复制粘贴到对称矩阵词条里面, 符号替换一下就行

\pentry{厄米矩阵\upref{HerMat}, 矩阵的本征方程\upref{MatEig}, 正交子空间\upref{OrthSp}}

本词条将 “对称矩阵的本征值问题\upref{SymEig}” 拓展到厄米矩阵, 结论和过程都相似. 我们来证明 $N$ 维厄米矩阵 $\mat H$ 存在 $N$ 个两两正交归一的本征矢 $\bvec v_1, \dots, \bvec v_N$.

% 未完成: 举例最重要
% 未完成: 参考 docx 版本

\subsection{本征值为实数}
本征方程为
\begin{equation}
\mat H \bvec v_i = \bvec v_i
\end{equation}
将本征方程左边乘以 $\bvec v_i$ 得
\begin{equation}
\bvec v_i\Her \mat H \bvec v_i = \lambda_i \bvec v_i\Her \bvec v_i
\end{equation}
将等式两边取厄米共轭(注意矢量也可以看成矩阵), 由\autoref{HerMat_eq2}\upref{HerMat} 和\autoref{HerMat_eq1}\upref{HerMat} 可得
\begin{equation}
\bvec v_i\Her \mat H\Her \bvec v_i = \bvec v_i\Her \mat H \bvec v_i = \lambda_i^* \bvec v_i\Her \bvec v_i
\end{equation}
对比两式, 得 $\lambda_i = \lambda_i^*$, 所以 $\lambda_i$ 必为实数.

\subsection{本征矢的正交性}
下面来证明不同本征值对应的本征矢正交, 即
\begin{equation}\label{HerEig_eq1}
\bvec v_i\Her \bvec v_j = 0 \qquad (a_i \ne a_j)
\end{equation}
首先令
\begin{equation}
s = \bvec v_i\Her (\mat H \bvec v_j) = \bvec v_i\Her (\lambda_j \bvec v_j) = \lambda_j \bvec v_i\Her \bvec v_j
\end{equation}
使用矩阵乘法结合律\autoref{Mat_eq1}\upref{Mat} 以及\autoref{HerMat_eq2}\upref{HerMat} 得
\begin{equation}
s = (\mat H \bvec v_i)\Her \bvec v_j = \lambda_i^* \bvec v_i\Her \bvec v_j = \lambda_i \bvec v_i\Her \bvec v_j
\end{equation}
以上两式相减得 % 简并空间内, 我们可以认为地指定正交归一基底, 所以只需要
\begin{equation}
(\lambda_j - \lambda_i)\bvec v_i\Her \bvec v_j = 0
\end{equation}
因为 $\lambda_i \ne \lambda_j$, 所以 $\bvec v_i\Her \bvec v_j = 0$.

\subsection{简并}
在 “矩阵的本征方程\upref{MatEig}” 中, 我们定义若令 $\lambda_i$ 的本征矢空间的维度是 $n_i$, 当 $n_i = 1$, 我们说 $\lambda_i$ 是\textbf{非简并}的, 当 $n_i > 1$ 就说 $\lambda_i$ 是 $n_i$ 重\textbf{简并}的, 把 $n_i$ 叫做\textbf{简并数}.

根据\autoref{HerEig_eq1}, 对于厄米矩阵, 所有不同的 $\lambda_i$ 对应的本征矢子空间是两两正交的, 且 $\sum_i n_i = N$, 所以所有这些子空间的直和就是 $\mat A$ 的定义域空间.

\subsection{基底}
对任意的 $N$ 维厄米矩阵 $\mat H$ 都能得到由本征矢构成的 $N$ 个正交归一基底.
\begin{itemize}
\item 如果不存在简并, 这组基底是唯一的, 且它们的本征值各不相同.
\item 如果存在简并, 每个子空间中可以选出 $n_i$ 个正交归一基底, 将他们放在一起就得到总空间中的 $N$ 个正交归一基底. 注意每个简并子空间中的正交归一基底的选取都是任意的.
\end{itemize}
