% 线性方程组高斯消元法

% 未完成:如果有无穷多种解, 该如何表示它们呢? 什么是特解,什么是通解。

\pentry{矩阵\upref{Mat}}

线性方程组
\begin{equation}\label{GAUSS_eq1}
\leftgroup{
&a_{1,1}x_1 + a_{1,2}x_2 + \dots + a_{1,n}x_n&=\quad &y_1\\
&a_{2,1}x_1 + a_{2,2}x_2 + \dots + a_{2,n}x_n&=\quad &y_2\\
&\qquad \qquad \dots  \qquad \dots \qquad  \dots\\
&a_{m,1}x_1 + a_{m,2}x_2 + \dots + a_{m,n}x_n&=\quad &y_m}\\
\end{equation}
可以写成矩阵和向量的形式
\begin{equation}\label{GAUSS_eq2}
\mat A \vec x = \vec y
\end{equation}
其中 $\mat A$ 是维度 $m \times n$ 的矩阵,称为\bb{系数矩阵}.
\begin{equation}
\mat A={
	\left[ \begin{matrix}
	a_{1,1} &a_{1,2} &\cdots &a_{1,n} \\
	a_{2,1} &a_{2,2} &\cdots &a_{2,n} \\
	\vdots  &\vdots  &\ddots &\vdots  \\
	a_{m,1} &a_{m,2} &\cdots &a_{m,n} \\
	\end{matrix} 
	\right ]}
\end{equation}
$\vec x$ 是 $n$ 维列矢量,$\vec x=[x_1,x_2,\dots,x_m,\dots,x_n]\Tr$.
$\vec y$ 是 $m$ 维列矢量,称为“常数列”.$\vec y=[y_1,y_2,\dots,y_m]\Tr$.
$\mat A$ 和 $\vec y$ 是已知的,而 $\vec x$ 是未知的, 即方程组待求的解\footnote{以下仅讨论 $m \les n$ 的情况}.

\subsection{最简形式}
若系数矩阵 $\mat A$ 是一个阶梯形的矩阵,即线性方程组有如下形式:
\begin{equation}\label{GAUSS_eq4}
{
	\left[ \begin{matrix}
	a_{1,1} &a_{1,2} &a_{1,3} &\cdots &a_{1,n} \\
	        &a_{2,2} &a_{2,3} &\cdots &a_{2,n} \\
	        &        &a_{3,3} &\cdots &a_{3,n} \\
             &        &        &\ddots &\vdots  \\
	        &        &        &       &a_{n,n} \\
	\end{matrix} 
	\right ]}
{
	\left[ \begin{matrix}
	x_1 \\
	x_2 \\
     x_3 \\
     \vdots \\
	x_n \\
	\end{matrix} 
	\right ]}=
{
	\left[ \begin{matrix}
	y_1 \\
	y_2 \\
     y_3 \\
     \vdots \\
	y_n \\
	\end{matrix} 
	\right ]}
\end{equation}
当 $a_{k,k}\neq 0 \quad (k=1,2,\dots,n)$ 时,显然方程组具有唯一的一组解
\begin{equation}\label{GAUSS_eq5}
\leftgroup{
x_n &=\frac{y_n}{a_{n,n}} \\
x_k &=\frac{1}{a_{k,k}} \qty(y_k - \sum_{i=k+1}^n a_{k,i} x_i ) \quad (k=1,2,\dots,n-1) }\\
\end{equation}
特别的,若 $a_{k,k} = 1 \quad (k=1,2,\dots,n)$,则称之为线性方程组的\bb{最简形式}.

那么,对于\autoref{GAUSS_eq1} 这样一般形式的线性方程组,能否转化为\autoref{GAUSS_eq4} 这样的形式,进而求解呢?回答是肯定的. % 未完成:GAUSS_eq4 是三角形矩阵,未必所有方程组都能化为三角形,而是梯形。

\subsection{Gauss消元法}

\subsubsection{增广矩阵}
线性方程组的系数矩阵 $\mat A$ 和常数列 $\vec y$ 可以包含在一个矩阵 $\mat B$ 中,称这个矩阵为“增广矩阵”.
\begin{equation}
\mat B=[\mat A ,\vec y]={
	\left[ \begin{array}{cccc|c}
	a_{1,1} &a_{1,2} &\cdots &a_{1,n} &y_1 \\
	a_{2,1} &a_{2,2} &\cdots &a_{2,n} &y_2 \\
	\vdots  &\vdots  &\ddots &\vdots  &\vdots \\
	a_{m,1} &a_{m,2} &\cdots &a_{m,n} &y_m \\
	\end{array} 
	\right ]}
\end{equation}

\subsubsection{矩阵初等行变换}
定义以下三种矩阵变换为“初等行变换”:

(a)对调矩阵中的任意两行(记作 $\vec r_i \leftrightarrow \vec r_j$,$i$,$j$ 表示行号);

(b)将矩阵中任意一行的所有元素乘以一个非零数 $k$(记作 $\vec r_i \times k$);

(c)把矩阵中某一行所有元素乘以数 $k$ 后加到另一行的对应元素上(记作 $\vec r_i + \vec r_j \times k$).

对增广矩阵进行有限次初等行变换,不改变方程组的解,这在直观上是容易理解的.

\subsubsection{消元法的一般步骤}
任何非零矩阵,经过有限次初等行变换,总可以转化为阶梯形矩阵.

增广矩阵 $\mat B$化为阶梯形矩阵的一般步骤是:

第1步,将第1行所有元素乘以 $p_{1,i}$,加到第二至最后一行,使矩阵 $\mat B$ 的第一列元素除第一行外以全为零;
\begin{equation}\begin{aligned}
\vec r_i^{(1)} &= \vec r_i + \vec r_1 \times p_{1,i} \quad (i=2,3,\dots,m) \\
p_{1,i} &= - \frac{a_{i,1}}{a_{1,1}} \\
\end{aligned}\end{equation}
得到新的增广矩阵 $\mat B^{(1)}$
\begin{equation}
\mat B^{(1)}={
	\left[ \begin{array}{cccc|c}
	a_{1,1} &a_{1,2}       &\cdots &a_{1,n}       &y_1 \\
	0       &a_{2,2}^{(1)} &\cdots &a_{2,n}^{(1)} &y_2^{(1)} \\
	\vdots  &\vdots        &\ddots &\vdots        &\vdots \\
	0       &a_{m,2}^{(1)} &\cdots &a_{m,n}^{(1)} &y_m^{(1)} \\
	\end{array} 
	\right ]}
\end{equation}

第 $k$ 步,将第 $k$ 行所有元素乘以 $p_{k,i}$,加到第 $k+1$ 至最后一行,使矩阵 $\mat B$ 的第 $k$ 列元素除第1行至第 $k$ 行以外全为零;
\begin{equation}\begin{aligned}
\vec r_i^{(k)} &= \vec r_i^{(k-1)} + \vec r_k^{(k-1)} \times p_{k,i} \quad (i=k+1,k+2,\dots,m) \\
p_{k,i} &= - \frac{a_{i,k}^{(k-1)}}{a_{k,k}^{(k-1)}} \\
\end{aligned}\end{equation}
得到新的增广矩阵 $\mat B^{(k)}$
\begin{equation}
\mat B^{(k)}={
	\left[ \begin{array}{ccccccc|c}
	a_{1,1} &a_{1,2}       &\cdots &a_{1,k}         &a_{1,k+1}         &\cdots &a_{1,n}         &y_1 \\
	0       &a_{2,2}^{(1)} &\cdots &a_{2,k}^{(1)}   &a_{2,k+1}^{(1)}   &\cdots &a_{2,n}^{(1)}   &y_2^{(1)} \\
	\vdots  &\vdots        &\ddots &\vdots          &\vdots            &\ddots &\vdots          &\vdots \\
	0       &0             &\cdots &a_{k,k}^{(k-1)} &a_{k,k+1}^{(k-1)} &\cdots &a_{k,n}^{(k-1)} &y_k^{(k-1)} \\
     0       &0             &\cdots &0               &a_{k+1,k+1}^{(k)} &\cdots &a_{k+1,n}^{(k)} &y_{k+1}^{(k)} \\
     \vdots  &\vdots        &\ddots &\vdots          &\vdots            &\ddots &\vdots          &\vdots \\
     0       &0             &\cdots &0               &a_{m,k+1}^{(k)}   &\cdots &a_{m,n}^{(k)}   &y_m^{(k)} \\
	\end{array} 
	\right ]}
\end{equation}

至多进行 $m-1$ 步初等行变换的迭代,最终将增广矩阵化为阶梯形.
\begin{equation}
\mat B^{(m-1)}={
	\left[ \begin{array}{cccc|c}
	a_{1,1} &a_{1,2}       &\cdots &a_{1,n}         &y_1 \\
	0       &a_{2,2}^{(1)} &\cdots &a_{2,n}^{(1)}   &y_2^{(1)} \\
	        &              &\ddots &\vdots          &\vdots \\
	        &              &       &a_{m,n}^{(m-1)} &y_m^{(m-1)} \\
	\end{array} 
	\right ]}
\end{equation}

观察最终简化的每个具体问题的增广矩阵,很容易判断线性方程组解的情况(无解、有唯一解或有无穷多组解).若矩阵 $\mat B^{(m-1)}$ 满足 $m=n$ 且 $a_{k,k}\neq 0 \quad (k=1,2,\dots,n)$,则线性方程组 $\mat A \vec x = \vec y$ 有唯一解.利用\autoref{GAUSS_eq5} 就可以回代出这个解.

\subsubsection{特殊情况}
在进行第 $k$ 步消元的时候,可能会遇到 $a_{k,k}^{(k-1)}=0$ 的情况,顺序迭代就无法继续下去,但是这个时候并不能直接判定方程组无解或有无穷多组解.可以从第 $k$ 行往下寻找 $a_{i,k}^{(k-1)} \neq 0$ 的一行,将其与第 $k$ 行调换,使迭代继续进行.

如果从第 $k$ 行往下所有的 $a_{i,k}^{(k-1)}$ 都等于零,则在矩阵待简化的这部分中,依次向下一列检索不为零的元素(假设这个元素为 $a_{s,t}^{(k-1)}$),将它所在的行与当前的第 $k$ 行调换.此时再令
\begin{equation}
p_{k,i} = - \frac{a_{i,t}^{(k-1)}}{a_{k,t}^{(k-1)}}
\end{equation}
其余表达式不变,继续迭代.如果检索不到符合条件的 $a_{s,t}^{(k-1)}$,即 $a_{s,t}^{(k-1)} \equiv 0 \quad(s=k,k+1,\dots,m;\;t=k,k+1,n+1)$,那么矩阵已经最简化,消元过程终止.

\begin{exam}{高斯消元法解线性方程组}\label{GAUSS_ex1}
\begin{equation}
\leftgroup{
&2x_1 + 3x_2 + 4x_3 + 5x_4&=\quad &6\\
&4x_1 + 6x_2 + 5x_3 + 6x_4&=\quad &8\\
&6x_1 + 9x_2 + 2x_3 + 3x_4&=\quad &5\\
&2x_1 + 3x_2 + 5x_3 + 4x_4&=\quad &9}\\
\end{equation}
解:

将该方程组写成增广矩阵形式
\begin{equation}
\mat B={
	\left[ \begin{array}{cccc|c}
	2 &3 &4 &5 &6 \\
	4 &6 &8 &6 &8 \\
	6 &9 &9 &8 &9 \\
	2 &3 &5 &4 &9 \\
	\end{array} 
	\right ]}
\end{equation}
开始消元
\begin{equation}
\begin{aligned}
\vec r_2 - 2 &\vec r_1 \\
\vec r_3 - 3 &\vec r_1 \\
\vec r_4 -   &\vec r_1 \\
\end{aligned}
\quad \Longrightarrow \quad
\mat B={
	\left[ \begin{array}{cccc|c}
	2 &3 &4  &5  &6  \\
	0 &0 &0  &-4 &-4 \\
	0 &0 &-3 &-7 &9  \\
	0 &0 &1  &-1 &3  \\
	\end{array} 
	\right ]}
\end{equation}

发现此时,矩阵第二列自第二行以下全为零,所以需要依次向下一列寻找不为零的元素.找到第四行第三列元素 $1$ 是符合条件.继续消元
\begin{equation}
\begin{aligned}
\vec r_2 \leftrightarrow &\vec r_4 \\
\vec r_3 + 3 &\vec r_2 \\
\vec r_4 - 0 &\vec r_2 \\
\end{aligned}
\quad \Longrightarrow \quad
\mat B={
	\left[ \begin{array}{cccc|c}
	2 &3 &4  &5   &6  \\
	0 &0 &1  &-1  &3  \\
	0 &0 &0  &-10 &18 \\
	0 &0 &0  &-4 &-4  \\
	\end{array} 
	\right ]}
\end{equation}
\begin{equation}
\vec r_4 - 0.4 \vec r_3
\quad \Longrightarrow \quad
\mat B={
	\left[ \begin{array}{cccc|c}
	2 &3 &4  &5   &6  \\
	0 &0 &1  &-1  &3  \\
	0 &0 &0  &-10 &18 \\
	0 &0 &0  &0   &-11.2  \\
	\end{array} 
	\right ]}
\end{equation}

显然,本例中方程组无解.
\end{exam}

