% 惯性张量
% 刚体|惯性张量|坐标系基底变换
% 所有连接未完成

\pentry{转动惯量\upref{RigRot}}

\subsection{瞬时转轴}
当刚体绕某点做任意转动时, 我们在每个时刻仍然能找到一个经过该点的\bb{瞬时转轴} 以及延转轴的瞬时角速度(矢量) $\bvec \omega$.

某时刻瞬时转轴的定义是: 该时刻刚体在内任意落在转轴上的点速度为零. 事实上, 我们只需要在某时刻找到刚体上瞬时速度为 0 的任意两个不同点, 就可以过这两点作出瞬时转轴, 并确保该直线上的所有点瞬时速度都为 0.

\begin{example}{进动陀螺的瞬时转轴}
在陀螺的例子(\autoref{AMLaw_ex2}\upref{AMLaw}) 中, 我们可能会认为陀螺的瞬时转轴就是陀螺的轴. 但陀螺的轴时时刻刻都在运动, 除了与地面接触的点外, 任意一点的瞬时速度都不为 0.

如果陀螺的自转和进动的方向相同, 我们可以知道真正的瞬时转轴同样经经过与地面的接触点, 但会在陀螺轴的上方. 我们只需要找到陀螺圆盘表面上瞬时速度为 0 的一点即可求出瞬时转轴的倾角.
\end{example}

\subsection{惯性张量}
刚体的角动量等于\bb{惯性张量} $\mat I$ 乘以瞬时角速度矢量 $\bvec \omega$
\begin{equation}
\bvec L = \mat I \bvec \omega
\end{equation}
惯性张量是一个 3 维方阵, 其阵元一般记为
\begin{equation}
\ten I = \begin{pmatrix}
I_{xx}& I_{xy}& I_{xz} \\
I_{yx}& I_{yy}& I_{yz} \\
I_{zx}& I_{zy}& I_{zz}
\end{pmatrix}
\end{equation}
如果将 $x, y, z$ 分别记为 $x_1, x_2, x_3$, 则 $\mat I$ 的矩阵元为
\begin{equation}
I_{ij} = \delta_{i, j} \int r^2 \rho(\bvec r)\dd{V} - \int x_i x_j \rho(\bvec r)\dd{V} \qquad (i, j = 1, 2, 3)
\end{equation}

\subsection{推导}
\begin{equation}
\bvec L = \sum_i m_i \bvec r_i \bvec v_i = \sum_i m_i \bvec r_i \cross (\bvec \omega \cross \bvec r_i) = \sum_i m_i r_i^2 \bvec \omega - \sum_i m_i (\bvec \omega \vdot \bvec r_i) \bvec r_i
\end{equation}
写成分量的形式, 并将求和表示为密度 $\rho$ 的积分得
\begin{equation}
\pmat{L_x\\ L_y\\ L_z} = \int \rho r^2 \pmat{\omega_x\\ \omega_y\\ \omega_z} \dd{V} - \int \rho
\begin{pmatrix}
xx & xy & xz\\
yx & yy & yz\\
zx & zy & zz
\end{pmatrix}
\pmat{\omega_x\\ \omega_y\\ \omega_z} \dd{V}
\end{equation}

\subsection{坐标系变换}
注意惯性张量矩阵与坐标系的选取有关, 我们先建立一个与刚体相对静止的参考系叫做\bb{体坐标系(body frame)}, 一般我们选择体坐标系的原点在刚体质心处, 但刚体做任意转动时, 体坐标系并不是一个惯性系, 所以我们还要选一个\bb{实验室坐标系(lab frame)}. 令体坐标系的矢量到实验室系的基底变换矩阵\upref{Rot3D}为 $R$, 记体坐标系和实验室系中的惯性张量分别为 $\mat I_0$ 和 $\mat I$, 则实验室系中的惯性张量等于
\begin{equation}
\mat I = \mat R \mat I_0 \mat R\Tr
\end{equation}

% 例子未完成, 一根细杆与转轴成一定倾角转动(计算杆需要提供的力矩)
% 例子未完成, 球体的惯性张量是对角的
% 例子未完成, 一个长方体的惯性张量
