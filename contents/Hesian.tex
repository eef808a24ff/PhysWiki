% 海森矩阵
% 线性代数|多元微积分|矩阵|海森矩阵(Hessian)|泰勒展开|矢量|梯度矢量|雅可比矩阵

一个二阶可导的多元函数 $f(\vec x)$ 的\bb{海森矩阵(Hessian)} $\mat H$ 定义为
\begin{equation}
H_{ij} = \pdv{f}{x_i}{x_j}
\end{equation}
$f(\vec x)$ 的泰勒展开\upref{NDtalr}的前两项可以用梯度矢量和海森矩阵表示为
\begin{equation}
f(\vec x) = f(\vec x_0) + (\vec x - \vec x_0) \grad f(\vec x_0) + \frac12(\vec x - \vec x_0) \mat H (\vec x - \vec x_0) + \order{(x-x_0)^3}
\end{equation}
海森矩阵可以看做梯度矢量 $\grad f$ 的雅可比矩阵, 即
\begin{equation}
H_{ij} = \pdv{x_j} \qty(\pdv{f}{x_i})
\end{equation}
所以有
\begin{equation}
\dd{(\grad f)} = \mat H \dd{\vec x}
\end{equation}

如果 $f(\vec x)$ 是一个二阶函数, 海森矩将不随 $\vec x$ 变化. 所以有
\begin{equation}
\grad f(\vec x) - \grad f(\vec x_0) = \mat H (\vec x - \vec x_0)
\end{equation}
则函数的极值点为(令 $\grad f = 0$)
\begin{equation}
\vec x = \vec x_0 - \mat H^{-1} \grad f(\vec x_0)
\end{equation}
这就是牛顿法寻找函数极小值的主要思路.
