% 刚体

\pentry{质心\upref{CM}, 球坐标系\upref{Sph}}

当我们要考虑一个物体的质量分布带来的力学效应时, 就不能再将其简化为一个质点. 许多情况下我们考虑的物体在某过程中形变较小可忽略不计, 这时我们就可以忽略它运动过程中的的任何形变, 从而大大简化问题. 我们把这种模型叫做\bb{刚体}. 在分析刚体时, 我们通常把刚体看做是质点系. 要这么做, 我们可以把刚体划分为无限多个体积无限小的微元, 再把每个微元近似为一个质量相同的质点即可. 

\subsection{刚体的自由度}

在没有任何约束的情况下, 三维空间中每个质点有 3 个\bb{自由度}, 即用三个完全独立的变量才能完全确定位置, 所以 $N$ 个质点组成的质点系共有 $3N$ 个自由度. 然而完全确定一个刚体的位置只需要 6 个变量, 这是因为刚体模型通过假设“任意两个质点之间距离不变”, 给质点系的位置施加了 $3N - 6$ 个条件. 如何得出 6 个自由度呢? 我们可以假设第一个质点有 3 个自由度, 第二个质点由于要与第一个质点保持距离不变, 只有 2 个自由度, 而第三个质点要与前两个质点保持距离不变, 只有 1 个自由度. 有了前三个质点后(假设它们不共线), 剩下所有质点的位置都可以由与这三个质点的距离确定, 所以任何刚体都有 6 个自由度.

我们也可以这么划分 6 个自由度: 令其中 3 个决定刚体上某点的位置, 2 个决定该点到刚体上另一点的矢量的方向(球坐标\upref{Sph}中的两个角度), 最后一个决定刚体绕该矢量旋转的角度.

若刚体只能绕固定点旋转, 那么刚体就只有 3 个转动自由度. 若在此基础上, 刚体只能绕固定轴旋转, 那它就只剩一个自由度了(转过的角度).
