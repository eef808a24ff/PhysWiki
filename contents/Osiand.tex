% Ossiander Note
% ossiander|streaking|helium

\subsection{实验参数}
\begin{table}[ht]
\centering
\caption{实验 XUV 参数}\label{Osiand_tab1}
\begin{tabular}{|c|c|c|c|c|}
\hline
能量 ($\Si{eV}$) & 93.9 & 97.2 & 108.2 & 112.8 \\
\hline
能量 (a.u.) & 3.4508 & 3.5720 & 3.9763 & 4.1453 \\
\hline
FWHMI ($\Si{as}$) & 230 & 180 & 180 & 130 \\
\hline
总长度 (a.u.) & 26.1182 & 20.4404 & 20.4404 & 14.7625 \\
\hline
周期数 & 14.3443 & 11.6205 & 12.9356 & 9.7395 \\
\hline
\end{tabular}
\end{table}

\subsection{计算参数}

对于 $\sin^2$ 波包, 总长度的 $0.36405666$ 就是 \textbf{FWHMI (FWHM intensity)}. 即 $\cos(0.36405666 \pi/ 2)^4 = 1/2$.

IR 波包: 波长 $800\Si{nm}$, $\sin^2$ 波形, 峰值光强 $4\times 10^{11} \Si{W/cm^2}$, 电场 $3.3761$ a.u., FWHMI 为 $3 \Si{fs}$, 总长度 $340.6726$ a.u., 周期数 $3.0880$.

XUV 波包: $\sin^2$ 波形, 光强峰值 $10^{12} \Si{w/cm^2}$, 即电场强度峰值 $5.3380\times 10^{-3}$ a.u., FWHMI $200 \Si{as}$, 总长度 $22.7115$ a.u.

\begin{table}[ht]
\centering
\caption{计算 XUV 参数}\label{Osiand_tab2}
\begin{tabular}{|c|c|c|c|c|}
\hline
能量 ($\Si{eV}$) & 90 & 95 & 100 & 110 \\
\hline
能量 (a.u.) & 3.3074 & 3.4912 & 3.6749 & 4.0424\\
\hline
周期数 & 11.9552 & 12.6194 & 13.2836 & 14.6119\\
\hline
\end{tabular}
\end{table}

\subsection{代码}
以上数据的 Matlab 脚本
\begin{lstlisting}[language=matlab]
load constants.mat au_t eV2au_E w_cm22au_Ef nm2au_E
alpha = acos(0.5^0.25)*2/pi; % FWHMI / sin2 pulse length
% alpha = 0.36405666

%% experimental XUV
E1 = eV2au_E([93.9; 97.2; 108.2; 112.8]); % a.u.
disp('E1'); disp(round(E1,4));
FWHMI1 = [230; 180; 180; 130]*1e-18/au_t; % a.u.
Len1 = FWHMI1/alpha; % a.u.
disp('Len1'); disp(round(Len1,4));
T1 = 2*pi./E1;
NT1 = Len1./T1;
disp('NT1'); disp(round(NT1,4));

%% TDSE XUV
Ef2 = w_cm22au_Ef(1e12); % max E field
disp('Ef2'); disp(Ef2);
E2 = eV2au_E([90; 95; 100; 110]);
disp('E2'); disp(round(E2,4));
FWHMI2 = 200e-18/au_t;
Len2 = FWHMI2/alpha;
disp('Len2'); disp(round(Len2,4));
T2 = 2*pi./E2;
NT2 = Len2./T2;
disp('NT2'); disp(round(NT2,4));

%% TDSE IR
Ef3 = w_cm22au_Ef(4e11); % max E field
disp('Ef3'); disp(round(Ef3, 4));
FWHMI3 = 3e-15/au_t; % a.u.
Len3 = FWHMI3 / alpha;
disp('Len3'); disp(round(Len3, 4));
E3 = nm2au_E(800); % a.u.
T3 = 2*pi/E3; % a.u.
NT3 = Len3/T3;
disp('NT3 = '); disp(round(NT3, 4));
\end{lstlisting}
