% 正交函数系
% 微积分|定积分|正交函数系|归一|克罗内克 δ 函数|Kronecker

\pentry{定积分\upref{DefInt}}
因变量为实数的情况

定义
给出一组函数(有限或无限多个), $f_i(x)\; (i = 1,2\dots)$, 如果满足
\begin{equation}
\int_a^b f_i(x) f_i(x) \dd{x} \ne 0
\end{equation} 
当整数 $m \ne n$ 时
\begin{equation}
\int_a^b f_m(x) f_n(x) \dd{x} = 0
\end{equation} 
那么这一组函数就是区间 $[a,b]$ 内的一个正交函数系.

这一组函数的性质可以类比矢量的正交,“两个函数相乘再积分”这个步骤可以类比矢量的点乘.如果两个不同的矢量正交(垂直),则它们的点乘为零.如果它们的模长不为零,则一个矢量点乘自身不为零.

特殊地,若给正交函数系中的每个函数的平方进行归一化%未完成:引用
,使得
\begin{equation}
\int_a^b f_i(x) f_i(x) \dd{x} = 1
\end{equation} 
那么该正交函数系就是\bb{归一}的.其性质可以表示为
 \begin{equation}
\int_a^b f_m(x) f_n(x) \dd{x} = \delta_{mn}
\end{equation} 
其中 $\delta_{mn}$ 是克罗内克 $\delta$ 函数(Kronecker Delta Function).


\subsection{因变量为复数的情况}

若函数系中 $f_i(x)$ 的自变量为实数,因变量为复数,则正交的定义变为
 \begin{equation}
\int_a^b f_i^*(x) f_j(x) \dd{x} \ne 0 \qquad ( i \ne j )
\end{equation}    
归一化的定义变为
 \begin{equation}
\int_a^b f_i^*(x) f_i(x) \dd{x} = 1
\end{equation} 
正交归一条件可以统一写成
 \begin{equation}
\int_a^b f_i^*(x) f_j(x) \dd{x} = \delta_{ij}
\end{equation}