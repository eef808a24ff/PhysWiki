% 巴拿赫空间
% 巴拿赫空间|范数|收敛|柯西序列

\pentry{范数\upref{NormV}}
在赋范空间中, 极限
\begin{equation}
\lim_{n\to\infty} x_n = x
\end{equation}
的定义是
\begin{equation}
\lim_{n\to\infty} \|{x_n - x}\|= 0
\end{equation}
这时说序列$\{x_n\}$按范数收敛到$x$.

如果赋范空间中任意柯西序列都有极限, 那么该赋范空间就是\textbf{完备(complete)}的。 完备的赋范空间常称为\textbf{巴拿赫空间(Banach space)}。 其中, 完备的内积空间特别称作\textbf{希尔伯特空间 (Hilbert space)}.

“完备” 可以形象理解为空间中没有 “漏洞”。 有限维空间都是完备的。 可数维空间都是不完备的。 例如有限阶多项式组成的空间就是不完备的(例如柯西序列的极限可以是 $\E^x$, 但是 $\E^x$ 并不属于该空间)。

\subsection{巴纳赫空间的例子}
\begin{example}{$\mathbb C^N$ 空间}
$N$ 维实空间$\mathbb R^N$或者复空间$\mathbb C^N$在任何范数之下都是完备的, 因此在任何范数之下都是巴拿赫空间.
\end{example}

\begin{example}{连续函数空间}
有限闭区间上的所有连续函数构成的空间记为 $X := C[a, b]$, 这是一个不可数维的空间。 若令范数为
$$
\|{u}\|:= \max_{a \leqslant x \leqslant b} \abs{u(x)},
$$
则它成为一个巴拿赫空间. 这范数下收敛的连续函数序列恰为一致收敛的连续函数序列. 而若赋予$X$以$L^p$范数
$$
\|u\|_{L^p}:=\left(\int_{\mathbb{R^N}}|u(x)|^pdx\right)^{1/p},
$$
则在此范数下它不是完备的.

更一般地, 对于任何紧 Hausforff 空间$K$, 连续函数空间$C(K)$在范数
$$
\|f\|:=\sup_{x\in K}|f(x)|
$$
之下也是巴拿赫空间.
\end{example}

\begin{example}{$L^p$空间}
对于测度空间$(\Omega,\mathcal{A},\mu)$, 定义可测函数的$L^p$范数为
$$
\|f\|_{L^p(\mu)}=\left(\int_\Omega |f(x)|^pd\mu(x)\right)^{1/p},
$$
$$
\|f\|_{L^\infty(\mu)}=\text{ess sup}_{\Omega}|f|.
$$
如果将几乎处处相等的函数视为相同, 则当$1\leq p\leq\infty$时$\|\cdot\|_{L^p(\mu)}$便是一个范数, 而满足$\|f\|_{L^p(\mu)}$的可测函数的线性空间就是$L^p(\mu)$. 它是完备的. 当$\Omega=\mathbb{N}$, 而测度$\mu$为普通的计数测度时, 可测函数就是通常的序列, 此时将空间记为$l^p$, 而序列的范数是
$$
\|x\|_p=\left(\sum_{n=1}^\infty|x(n)|^p\right)^{1/p}.
$$
\end{example}

\subsection{巴拿赫空间的基本性质}
赋范线性空间在度量空间的意义下经过完备化之后成为巴拿赫空间. 完备的赋范空间在赋以等价范数后还是完备的. 可分的赋范线性空间的完备化空间还是可分的. 

一个赋范空间$(X,\|\cdot\|)$是完备的, 当且仅当由$\sum _{n=1}^{\infty }\|x_{n}\|<\infty $总可以推出$\sum _{n=1}^{\infty }x_{n}$按照范数收敛.

如果$(X,\|\cdot\|_X),(Y,\|\cdot\|_Y)$是巴拿赫空间, 那么其直积$X\times Y$在范数$\|(x,y)\|_{p}$之下也是巴拿赫空间. 有界线性算子空间$\mathfrak{B}(X,Y)$在算子范数下也是巴拿赫空间. 如果$X$巴拿赫空间, $M\subset X$是其闭子空间, 则商空间$X/M$在商范数之下也是巴拿赫空间. 这里用到的定义见\upref{NormV}.

巴拿赫空间的代数维数 (即其中极大线性无关向量组的势) 一定是不可数的.

无穷维巴拿赫空间中的闭单位球$\{\|x\|\leq1\}$在其范数拓扑下一定不是紧的.

