% 常见物理量

\pentry{速度\ 加速度\upref{VnA}, 通量\upref{SurInt}}

我们已知速度和加速度都由极限定义, 事实上, 许多其他物理量同样通过极限来定义.

\subsection{密度}
我们常把\bb{质量密度} 简称为\bb{密度}, 记为 $\rho$, 定义为
\begin{equation}
\rho = \lim_{\Delta V \to 0} \frac{\Delta m}{\Delta V}
\end{equation}
其中 $\Delta m$ 是体积元 $\Delta V$ 内的质量. 当我们讨论的是宏观物体的密度时, 这个极限的理解是, $\Delta V$ 远小于宏观尺度, 却远大于微观尺度. 因为微观粒子(质子, 电子等) 的体积非常小, 它们之间有很大的空间, 如果按照数学定义取无限小, 很大概率会得到密度为零.

类似地我们可以定义其他物理量的密度, 如\bb{电荷密度}(将上式的质量 $m$ 换成净电荷 $q$), 粒子数密度(将 $m$ 换成粒子数 $N$)等.

\subsection{流密度}

在通量

该式中的 $\vec F$ 有时被称为\bb{流密度}. 以上面的水流场为例, 若把一定时间内流过曲面的水看做通量, 那么“单位体积中水的质量” 就是水的\bb{质量密度}, 而流密度就是密度乘以速度矢量.


质量流密度
电流密度
