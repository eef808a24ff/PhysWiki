%速度 加速度

\pentry{速度\ 加速度\ (一维)\upref{VnA1}, 矢量的导数\upref{DerV},矢量积分}% 未完成

在大学物理中,“位移”,“速度”和“加速度”都是矢量,既包括了大小,也包括方向.如果没有特殊说明,它们一般是指“瞬时速度”和“瞬时加速度”.

\subsection{速度的定义}
考察一个质点在运动过程中在某时刻经过某一点的速度,就取质点在这一点附近的一小段位移 $\Delta\vec r$,以及物体完成这段位移需要的时间 $\Delta t$. 那么当 $\Delta t$ 无穷小时,若 $\Delta \vec r/\Delta t$ 存在极限,则这个极限就是速度矢量 $\vec v$. 写成极限的形式,就是
\begin{equation}\label{VnA_eq1}
\vec v = \lim_{\Delta t \to 0} \frac{\Delta \vec r}{\Delta t}
\end{equation}

\subsection{速度与位矢的关系}

质点在运动时,其位矢 $\vec r$ 是时间 $t$ 的函数,质点在 $t_1$ 时刻的位矢为 $\vec r(t_1)$,经过时间 $\Delta t$, 位矢为 $\vec r(t_1 + \Delta t)$, 所以物体在 $\Delta t$ 时间内的位移为
\begin{equation}\label{VnA_eq2}
\Delta \vec r = \vec r (t_1 + \Delta t) - \vec r (t_1)
\end{equation}
\autoref{VnA_eq2} 代入\autoref{VnA_eq1},得
\begin{equation}\label{VnA_eq3}
\vec v(t_1) = \lim_{\Delta t \to 0} \frac{\vec r (t_1 + \Delta t) - \vec r(t_1)}{\Delta t}
\end{equation}
根据矢量求导\upref{DerV} 的定义,这就是位矢对时间的导数,即
\begin{equation}
\vec v = \dv{\vec r}{t}
\end{equation}

\eentry{匀速圆周运动的速度(求导法)\upref{CMVD}}

\subsection{加速度的定义}

通常情况下,质点运动轨迹上的每一点都会对应一个确定的速度矢量\footnote{注意上面的速度在定义时虽然取了两点,但是取极限以后,速度和位置是一一对应的,也就和时间一一对应,而不是两个位置和时间对应一个速度.}, 类比速度的定义, 加速的定义为
\begin{equation}
\vec a(t_1) = \lim_{\Delta t \to 0} \frac{\vec v(t_1 + \Delta t) - \vec v (t_1)}{\Delta t} = \dv{\vec v}{t}
\end{equation}
结合速度的定义,加速度为
\begin{equation}
\vec a = \dv{\vec v}{t} = \dv{t} \qty( \dv{\vec r}{t} ) = \dv[2]{\vec r}{t}
\end{equation}
所以,加速度是速度对时间的导数,或者位矢对时间的二阶导数.
\eentry{匀速圆周运动的速加速度(求导法\upref{Der})}

\subsection{由速度或加速度计算位矢}

如果已知速度关于时间的函数 $\vec v(t)$, 以及初始时间 $t_0$ 和位置 $\vec r_0$, 该如何得到位移—时间函数 $\vec r(t)$ 呢? 类比一维的情况\upref{VnA1}, 我们也可以通过矢量函数的定积分\upref{IntV}(见\autoref{IntV_ex1}) 来求出速度—时间函数进而求出位移—时间函数
\begin{equation}\label{VnA_eq7}
\vec v(t) = \vec v_0 + \int_{t_0}^{t} \vec a(t) \dd{t}
\end{equation}
\begin{equation}\label{VnA_eq8}\ali{
\vec r(t) &= \vec r_0 + \int_{t_0}^{t} \vec v(t) \dd{t}
}\end{equation}

\eentry{匀加速运动\upref{ConstA}}









