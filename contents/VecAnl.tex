% 矢量分析总结

\subsection{标量场和矢量场}
\subsubsection{标量场}
\begin{equation}\label{VecAnl_eq1}
\Phi=\Phi(x,y,z)
\end{equation}
 $\Phi$的数值是空间位置的函数
 等值面
\begin{equation}\label{VecAnl_eq2}
\Phi=C
\end{equation}
 例如气压场、温度场.
\subsubsection{矢量场(详见??}%\upref{Vfiel})
\begin{equation}\label{VecAnl_eq3}
\bvec {A}= \bvec{A}(x,y,z)
\end{equation}
$\bvec{A}$的大小、方向是空间位置的函数.
例如速度场$\bvec{v}$、电场$\bvec{E}$.\\
\qquad 场线,
 有方向的曲线,其上每一点切线方向都与$\bvec{A}$的方向一致.\\
\qquad 场管 ,
由一束场线围城的管状区域.

\subsection{标量场的梯度}%(详见\upref{Grad}
\subsubsection{方向微商}
\begin{equation}\label{VecAnl_eq4}
\frac{\partial \Phi}{\partial l}=\lim_{\Delta l \to 0}\frac{\Delta \Phi}{\Delta l}
\end{equation}
标量场$\Phi$在P点沿$\Delta \bvec{l} $ 方向的方向微商.

\subsubsection{标量场$\Phi$的梯度}
沿方向微商最大的方向(即$\Delta \bvec{n}$方向).
\begin{equation}\label{VecAnl_eq5}
grad \Phi = \Delta \Phi =\frac{\partial \Phi}{\partial n}
\end{equation}
$\Delta \Phi$方向总于$\Phi$的等值面垂直.\\
标量场的梯度是矢量场\\
电势U是标量场,其负梯度$\bvec{E}$是矢量场.

\subsection{矢量场的通量和散度\ 高斯定理}%(\upref{Divgnc})
\subsubsection{定义}
通量
\begin{equation}\label{VecAnl_eq6}
\Phi_A=\iint\limits_{(S)} \bvec{A} \dd{\bvec S} = \iint\limits_{(S)} A\cos\theta dS
\end{equation}
流速场、流量、电通量、磁通量.

\subsubsection{散度}
\begin{equation}\label{VecAnl_eq7}
div \bvec{A}=\nabla \cdot \bvec{A}=\lim_{\Delta V \to 0}\frac{\Phi}{\Delta V}
\end{equation}
矢量场的散度是标量场.

\subsubsection{坐标表示}
\begin{equation}\label{VecAnl_eq8}
\nabla \cdot \bvec{A}= \frac{\partial A_x}{\partial x}+\frac{\partial A_y}{\partial y}+\frac{\partial A_z}{\partial z}
\end{equation}
\begin{equation}\label{VecAnl_eq9}
\oint \bvec{A}=\iiint_V \nabla \cdot \bvec{A} dV
\end{equation}
矢量场通过任意闭合曲面S的通量等于它向包围体积V内的散度积分.
\subsection{矢量场的环量和旋度\ Stokes定理}%(\upref{Curl})

\subsubsection{定义}
环量$\Gamma$
\begin{equation}\label{VecAnl_eq10}
\Gamma=\oint_L \bvec{A} \cdot d \bvec{l}
\end{equation}
矢量场$\bvec{A}$沿闭合回路线积分.\\
$\delta S$为闭线L包围面积,$\bvec{n}$为右旋单位法向量.\\
旋度 \qquad rot $\bvec{A}$\\
\begin{equation}\label{VecAnl_eq11}
rot \bvec{A}=\nabla \times \bvec{A}=\lim_{\Delta S \to 0} \frac{\oint_L \bvec{A} d\bvec{l}}{\Delta S}
\end{equation}
矢量场的旋度仍是矢量场.

\subsubsection{坐标表示}
\begin{equation}\label{VecAnl_eq12}
\nabla \times \bvec{A}=
\begin{vmatrix}
\bvec{i} & \bvec{j} & \bvec{k} \\
\frac{\partial}{\partial x} & \frac{\partial}{\partial y} & \frac{\partial}{\partial z}\\
A_x & A_y & A_z
\end{vmatrix}
\end{equation}
\begin{equation}\label{VecAnl_eq13}
\nabla = \bvec{i}\frac{\partial}{\partial x}+ \bvec{j} \frac{\partial}{\partial}y + \bvec{k} \frac{\partial}{\partial z}
\end{equation}

\subsubsection{stokes 定理}
\begin{equation}\label{VecAnl_eq14}
\oint_L \bvec{A} d\bvec{l}=\iint_S (\nabla \times \bvec{A})\cdot d\bvec{S}
\end{equation}
矢量场在任意闭合回路L上的环量等于它为边界的曲面S上旋度的积分.

\subsection{一些重要公式}
\subsubsection{场量乘积的微分公式}
梯度
\begin{equation}\label{VecAnl_eq15}
\nabla(\Phi \Psi)=(\nabla \Phi)\Psi+\Phi(\nabla \Psi)
\end{equation}
\begin{equation}\label{VecAnl_eq16}
\nabla(\bvec{A} \cdot \bvec{B})=(\bvec{A} \cdot \nabla)\bvec{B}+(\bvec{B}\cdot\nabla)+\bvec{A}\times(\nabla \times \bvec{B})+\bvec{B}\times(\nabla \times \bvec{A})
\end{equation}
其中$(\bvec{B}\cdot\nabla)\bvec{A}$即为张量.\\
散度\\
\begin{equation}\label{VecAnl_eq17}
\nabla\cdot(\Phi\bvec{A})=\nabla\Phi\cdot\bvec{A}+\Phi\nabla\cdot\bvec{A}
\end{equation}
\begin{equation}\label{VecAnl_eq18}
\nabla\cdot(\bvec{A}\times\bvec{B})=\bvec{B}\cdot\nabla\times\bvec{A}-\bvec{A}\cdot\bvec{B}
\end{equation}
旋度\\
\begin{equation}\label{VecAnl_eq19}
\nabla\times(\Phi\bvec{A})=\Phi\nabla\times\bvec{A}+\nabla\Phi\times\bvec{A}
\end{equation}

\subsubsection{二阶微商公式}
\begin{equation}\label{VecAnl_eq20}
\nabla\times\nabla\Phi=0
\end{equation}
\begin{equation}\label{VecAnl_eq21}
\nabla\cdot\nabla\times\bvec{A}=0
\end{equation}
\begin{equation}\label{VecAnl_eq22}
\nabla\times(\nabla\times\bvec{A})=\nabla(\nabla\cdot\bvec{A})-\nabla\cdot\nabla\bvec{A}
\end{equation}
\begin{equation}\label{VecAnl_eq23}
\nabla\cdot\nabla=\nabla^2
\end{equation}
此为Laplace算符.

\subsection{矢量场的分类}
\subsubsection{有散场和无散场}
散度为0,即无源,为无散场;散度不为0,即有源,有散场.\\
有公式:
\begin{equation}\label{VecAnl_eq24}
\nabla\cdot\nabla\times\bvec{A}=0
\end{equation}
知,任何矢量场的旋度永远是无散场.\\
任何无散场$\bvec{B}$可表达成某矢量场的旋度.
\begin{equation}\label{VecAnl_eq25}
\bvec{B}=\nabla\times\bvec{A} ,\nabla\cdot\bvec{B}=0
\end{equation}
\subsubsection{有旋场和无旋场}
旋度为0,为无旋场;反之为有旋场.
\begin{equation}\label{VecAnl_eq26}
\nabla\times\nabla\Phi=0
\end{equation}
任何标量场的梯度永远是无旋场.\\
任何无旋场$\bvec{A}$可表示为某个标量场$\Phi$的梯度.
\begin{equation}\label{VecAnl_eq27}
\bvec{A}=\nabla\Phi,\nabla\times\bvec{A}=0
\end{equation}

\subsubsection{谐和场}
谐和场为某一矢量$\bvec{A}$在某空间内既无散又无旋,由于其无旋,所以可以由势场表示:
\begin{equation}\label{VecAnl_eq28}
\bvec{A}=\nabla\Phi,\nabla\times\bvec{A}=0
\end{equation}
同样由于其为无散场,所以有:
\begin{equation}\label{VecAnl_eq29}
\bvec{B}=\nabla\times\bvec{A} ,\nabla\cdot\bvec{B}=0
\end{equation}
故可以导出Laplace方程:
\begin{equation}\label{VecAnl_eq30}
\nabla\cdot\nabla=\nabla^2
\end{equation}
谐和场的势函数满足Laplace方程.
