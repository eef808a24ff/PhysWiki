% 理想气体状态方程

理想气体状态方程

\begin{equation}\label{PVnRT_eq1}
PV = nRT
\end{equation}
其中 $P$ 是压强, $V$ 是体积, $n$ 是摩尔数, $R$ 是理想气体常量, $T$ 是绝对温度.

理想气体常数定义为
\begin{equation}\label{PVnRT_eq2}
R = k_B N_A
\end{equation}
其中 $k_B$ 为\textbf{玻尔兹曼常数}, $N_A$ 为\textbf{阿伏伽德罗常数}(一摩尔粒子中的粒子数).

从微观角度, 绝对温度可以由分子平均动能定义
\begin{equation}\label{PVnRT_eq3}
\bar E_k = \frac32 k_B T
\end{equation}

\subsection{由经典力学推导}
分子密度趋近于 0, 导致分子之间没有任何作用. 假设气体分子的运动各向同性, 速度大小分部函数为任意函数.

假设长方体容器的 $x, y, z$ 三个方向的边长分别为 $a, b, c$, 则体积为  $V = abc$. 考虑一个初始延任意方向运动的分子, 与容器壁发生完全弹性碰撞, 它在 $x$ 方向的周期为 $2a/v_x$, 每个周期带给 $x = a$ 容器壁的冲量为 $2m v_x$, 该容器壁面积为 $bc$, 所以受到的平均压强为冲量除以周期除以面积 $mv_x^2/V$, 即
\begin{equation}
P_i V = mv x_i^2
\end{equation}
如果有 $N$ 个分子, 质量都为 $m$, 那么
\begin{equation}
P V = 2 N \qty(\frac12 m \bar v_x^2) = 2 N \bar E_{kx}
\end{equation}
等式右边等于 $2N$ 乘以所有分子在 $x$ 方向的平均动能. 由于我们假设分子运动各向同性, 所以总动能等于单方向动能的 3 倍(注意这里我们假设是 3 维空间, 如果是 $N_d$ 维空间, 就是 $N_d$ 倍). 所以有
\begin{equation}
P V = \frac23 N \bar E_k
\end{equation}
使用\autoref{PVnRT_eq2} 和\autoref{PVnRT_eq3} 消去 $E_k$ 和 $N$, 就可得理想气体状态方程\autoref{PVnRT_eq1}.
