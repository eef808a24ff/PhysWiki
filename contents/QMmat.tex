% 量子力学与矩阵

\pentry{量子力学简介\upref{QM0}, 平面旋转矩阵\upref{Rot2D}, 重积分\upref{IntN}, 厄米矩阵本征值问题} % 链接未完成

在量子力学简介\upref{QM0} 中, 我们知道粒子的状态由波函数描述. 许多量子力学教材会从波函数和薛定谔方程将其, 而另一些教材会从矩阵的形式讲起. 前者由于涉及到求解微分方程, 数学要求会相对较高, 而后者只需涉及一些有限维的线性代数. 二者乍看起来不同, 但实际上几乎是等效的\footnote{最大的区别和难点是前者是无穷维而后者是有穷维}. 为了让读者先对量子力学的理论结构由一个总体的了解, 避免迷失在数学计算中, 我们先介绍矩阵的形式.

以下将\autoref{QM0_ex2}\upref{QM0} 展开讨论, 使用更多线性代数的语言, 将量子力学的基本假设应用起来. 该例中的角动量本征态 $\ket{x+}$, $\ket{x-}$ 和 $\ket{z+}$, $\ket{z-}$ 之间的关系是线性的. 这很容易让我们联想到矢量空间的基底(如果不考虑 $y$ 方向的自旋, 这个矢量空间是二维的). 根据矢量空间中的定义, 这四个本征态符合作为矢量的要求: 我们可以给它们乘以常数, 可以把它们相加(见\autoref{QM0_eq2}\upref{QM}), 等等(参考\autoref{LSpace_ex1}\upref{LSpace}). 正是由于这个原因, 我们把量子力学中的态(即波函数)称为\bb{状态矢量(state vector)}, 简称\bb{态矢}.

当明确了它们是矢量后, 我们就可以对这些态(波函数)定义点乘(点乘). 以(某个时刻的)一维(波)函数 $f(x)$, $g(x)$ 为例(可以用狄拉克符号记为 $\ket{f}$, $\ket{g}$), 我们将其内积定义为(注意这里的内积是有顺序的)% 未完成: 这个概念是否应该已经在傅里叶级数中讲过?
\begin{equation}
\braket{f}{g} = \int_{-\infty}^{+\infty} f(x)^* g(x) \dd{x}
\end{equation}
如果态矢 $\ket{f}$ 和 $\ket{g}$ 是二维或三维的波函数, 我们只需要用重积分即可(积分的范围是全空间, 即三个方向的上下限都为无穷大)\footnote{再次强调, \autoref{QM0_eq2}\upref{QM}讨论的态矢是抽象的, 并不能用波函数表示, 但为了方便理解, 我们姑且假设它们可以.}
\begin{equation}
\braket{f}{g} = \int f(\vec r)^* g(\vec r) \dd[3]{r} = \iiint f(x, y, z)^* g(x, y, z) \dd{x}\dd{y}\dd{z}
\end{equation}

\subsection{角动量算符}
在 “量子力学简介\upref{QM0}” 中, 由于数学工具上的不足, 我们并没有提及某个物理量的本征态是怎么得到的. 

在量子力学中, 每个物理量都可以对应一个\bb{算符}, 算符可以想象为对(波)函数的一种操作, 算符作用在(波)函数上可以得到一个新的函数. 例如某时刻函数为 $\sin x$, 求导算符 $\dv*{x}$ 作用在 $\sin x$ 上就得到一个新的函数 $\cos x$. 又例如坐标 $x$ 也可以作为一个算符, 我们定义将其作用在任意函数 $\Psi(x, t)$ 上, 就是将其相乘, 即 $x\Psi(x, t)$. 又例如任意函数 $f(x)$ 也可以是一个算符, 我们定义将其作用在 $\Psi(x, t)$ 上得 $f(x)\Psi(x, t)$.

不难发现这些算符都是线性的, 即我们可以用矩阵来表示它们. 注意能这么做的前提是我们考察的矢量空间是有限维的\footnote{这里要提醒一下态矢所在的矢量空间的维度和物理空间的维度不是一回事, 例如将一维(物理上的空间维度, 即直线运动)波函数 $\sin(x)$ 和 $\cos(x)$ 作为基底, 可以组成一个二维态矢空间.}, 例如现在我们只讨论以 $\ket{z+}$ 和 $\ket{z-}$ 为基底的二维态矢空间, 无限维空间在数学上会产生许多麻烦.

(未完成)

\subsection{回收的内容}
如果我们对处于 $\ket{z+}$ 或者 $\ket{z-}$ 态的粒子测量 $z$ 方向角动量, 显然也会分别得到确定的值 $\hbar/2$ 和 $-\hbar/2$. 如果对 $\ket{z+}$ 或者 $\ket{z-}$ 测量 $x$ 方向的角动量, 就需要先将 $\ket{z+}$ 和 $\ket{z-}$ 用 $\ket{x+}$ 和 $\ket{x-}$ 表示(将\autoref{QM0_eq2}\upref{QM0} 中的两式相加或相减即可).

% 回收: 如果我们把它看作 这事实上跟平面旋转变换相同(未完成)% 此处引用一个几何矢量的例子, 未完成
%相同, 我们只需要把

