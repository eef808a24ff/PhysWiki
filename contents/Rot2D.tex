% 平面旋转矩阵
% 线性代数|矩阵|平面旋转变换|线性变换|平面旋转矩阵|基底变换|逆变换

\pentry{矩阵\upref{Mat}}

平面旋转变换\upref{Rot2DT}属于线性变换,可以用矩阵 $\mat R_2$ 表示.虽然我们可以直接把变换写成矩阵乘以列矢量的形式,但这里我们用另一种方法推导一次, 能更好地帮助理解和记忆.

已知单位矢量 $\uvec x = (1, 0)\Tr, \uvec y = (0, 1)\Tr$ 逆时针旋转 $\theta$ 为
\begin{equation}
\mat R_2\pmat{1\\0} = \pmat{\cos \theta \\ \sin \theta}
\qquad
\mat R_2\pmat{0\\1} = \pmat{-\sin\theta \\ \cos\theta}
\end{equation}
要求任意矢量 $\bvec v = (v_1, v_2)\Tr$ 的旋转矩阵, 可以将 $\bvec v$ 表示成 $\uvec x$ 和 $\uvec y$ 的线性组合(\autoref{GVec_eq1}\upref{GVec}) $\bvec v = v_1\uvec x + v_2\uvec y$. 由\autoref{Mat_eq17}\upref{Mat}, 该线性组合的旋转变换等于 $\uvec x, \uvec y$ 分别做旋转变换再做同样的线性组合, 即
\begin{equation}
\ali{
\mat R_2 \qty(v_1 \uvec x + v_2 \uvec y)
&= v_1 \mat R_2\uvec x + v_2 \mat R_2\uvec y\\
&= v_1\pmat{\cos \theta \\ \sin \theta} 
  + v_2 \pmat{-\sin\theta \\ \cos\theta} \\
&= \pmat{\cos \theta & - \sin \theta \\ \sin \theta & \cos \theta }
\pmat{v_1 \\v_2}
}\end{equation}
所以旋转矩阵为
\begin{equation}\label{Rot2D_eq3}
\mat R_2 = \begin{pmatrix}
\cos\theta & - \sin\theta\\
\sin\theta &\cos\theta
\end{pmatrix}
\end{equation}
这与平面旋转变换\upref{Rot2DT}得出的结果一致.

把这个推导推广到一般情况, 就是如果已知每个基底 $\bvec \beta_i$ 的线性变换(记变换矩阵为 $\mat A$)结果为 $\bvec \alpha_i = \mat A \bvec \beta_i$, 那么变换矩阵的第 $i$ 列就是第 $i$ 个列矢量 $\bvec \alpha_i$.

\subsection{主动和被动理解}
我们将任意矢量 $\bvec v$ 的旋转变换记为 $\bvec u = \mat R_2 \bvec v$, 我们把 $\bvec v$ 看做是二维空间中某矢量关于基底 $\uvec x, \uvec y$ 的坐标. 若我们把 $\bvec u$ 看做是另一矢量关于 $\uvec x, \uvec y$ 的坐标, 那么 $\uvec u$ 就等于 $\uvec v$ 逆时针旋转 $\theta$ 角. 旋转矩阵的这种理解被称为\textbf{主动}的.

另一种可能的理解是, $\bvec u$ 和 $\bvec v$ 代表二维空间中的同一矢量关于不同基底的展开. 我们把 $\bvec u$ 使用的基底记为 $\uvec u_1, \uvec u_2$, $\bvec v$ 使用的基底记为 $\uvec v_1, \uvec v_2$. 我们有
\begin{equation}
u_1 \uvec u_1 + u_2 \uvec u_2 = v_1 \uvec v_1 + v_2 \uvec v_2
\end{equation}
将 $\mat R_2$ 的矩阵元记为 $R_{ij}$, 不难证明两组基底之间的关系为
\begin{equation}
\leftgroup{
\uvec v_1 = R_{11} \uvec u_1 + R_{21} \uvec u_2 \\
\uvec v_2 = R_{12} \uvec u_1 + R_{22} \uvec u_2
}\end{equation}
将矩阵元代入可知, 基底 $\uvec u_1, \uvec u_2$ 分别是基底 $\uvec v_1, \uvec v_2$ 顺时针旋转 $\theta$ 角所得. 我们把这种理解叫做\textbf{被动}的, 即旋转矩阵表示同一个矢量的\textbf{基底变换}.

\subsection{逆矩阵}
我们既可以使用平面旋转变换\upref{Rot2DT}中求逆变换的方法把 $\theta$ 变为 $-\theta$ 再化简求出 $\mat R_2$ 的逆矩阵,也可以通过解方程求逆矩阵(\autoref{Mat_eq25}). 但最方便的是,由于 $\mat R_2$ 是一个单位正交阵% 链接未完成: \upref{UniMat}?
, 我们只需要把矩阵转置即可得到逆矩阵.
\begin{equation}
\mat R_2^{-1} = \mat R_2\Tr = \begin{pmatrix}
\cos\theta &\sin\theta \\
-\sin\theta &\cos\theta
\end{pmatrix}
\end{equation}

