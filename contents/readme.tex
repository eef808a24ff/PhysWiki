% 编辑器使用说明

\subsection{综述}
\begin{itemize}
\item 本编辑器为本站自主开发, 可直接将 $\LaTeX$ 代码转换为网页而不是 pdf, 从而实现了实时编译(快捷键 Ctrl + S)
\item 我们近期也会开发在线编译 pdf 并下载的功能
\item 该编辑器除公式外只支持有限的 $\LaTeX$ 的命令(建议使用工具栏)
\item 编辑器处于测试阶段, 请自行使用下载按钮备份
\item 我们在模板中加入了许多自定义命令(见编写规范\upref{Sample}), 但不会影响 $\LaTeX$ 原有的命令(除了 \textbackslash vec)。
\end{itemize}

\subsection{百科的结构}

百科\link{网页版}{http://littleshi.cn/online/}中的所有词条是一个 $\LaTeX$ 的 document 环境, 每个部分是一个 \textbackslash part, 每章是一个 \textbackslash chapter, 词条是 \textbackslash section, 蓝色的小标题是 \textbackslash subsubsection, 黑色的小标题是 \textbackslash subsubsection。 编辑器打开的一个词条文件就是一个 subsection。

\subsection{公式}
\begin{itemize}
\item 公式环境支持大部分 $\LaTeX$ 命令, 严格来说是所有 \link{MathJax}{https://www.mathjax.org/} 支持的命令\footnote{MathJax 项目用于在网页上显示 $\LaTeX$ 公式}。
\item 支持部分 \link{Physics 宏包}{http://mirrors.ibiblio.org/CTAN/macros/latex/contrib/physics/physics.pdf}中的命令。
\item 我们把 \textbackslash vec 命令重新定义为正体和粗体, 如 \textbackslash vec a 被显示为 $\vec a$。
\end{itemize}

\subsection{格式要求}

\begin{itemize}
\item 行内公式用美元符号
\item 独立公式用 equation 环境, align 环境或者 gather 环境
\end{itemize}
