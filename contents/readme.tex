% 编辑器使用说明

\subsection{综述}
\begin{itemize}
\item 本编辑器为本站自主开发, 可直接将 LaTeX 代码转换为网页而不是 pdf, 从而实现了实时编译(快捷键 Ctrl + S)
\item 我们近期也会开发在线编译 pdf 并下载的功能
\item 该编辑器除公式外只支持有限的 LaTeX 的命令(建议使用工具栏)
\item 编辑器处于测试阶段, 请自行使用下载按钮备份
\item 我们在模板中加入了许多自定义命令(见编写规范\upref{Sample}), 但不会影响 LaTeX 原有的命令(除了 \lstinline|\vec|)。
\end{itemize}

\subsection{百科的结构}

百科\link{网页版}{http://littleshi.cn/online/}中的所有词条是一个 LaTeX 的 document 环境, 每个部分是一个 \lstinline|\part|, 每章是一个 \lstinline|\chapter|, 词条是 \lstinline|\section|, 蓝色的小标题是 \lstinline|\subsubsection|, 黑色的小标题是 \lstinline|\subsubsection|。 编辑器打开的一个词条文件就是一个 subsection。 用 texlive 编译 pdf 的时候所有词条文件都会通过 \lstinline|\input| 插入到主文件 PhysWiki.tex 中。 PhysWiki.tex 打开并编辑。

littleshi.cn/online/ 中的目录就是根据 PhysWiki.tex 中的内容生成的, 所以发布词条后必须修改该文件才能在目录中显示词条。

\section{编辑器说明}
如果要在预览和编辑器之间跳转, 可以通过搜索关键词实现。 例如在预览窗口复制一段文字, 在编辑窗口搜索并黏贴就可以跳转到对应内容。

\subsection{公式}
\begin{itemize}
\item 公式环境支持大部分 LaTeX 命令, 严格来说是所有 \link{MathJax}{https://www.mathjax.org/} 支持的命令\footnote{MathJax 项目用于在网页上显示 LaTeX 公式}。
\item 支持部分 \link{Physics 宏包}{http://mirrors.ibiblio.org/CTAN/macros/latex/contrib/physics/physics.pdf}中的命令。
\item 我们把 \lstinline|\vec| 命令重新定义为正体和粗体, 如 \lstinline|\vec a| 被显示为 $\vec a$。
\end{itemize}

\subsection{格式要求}

\begin{itemize}
\item 行内公式用美元符号
\item 独立公式用 equation 环境, align 环境或者 gather 环境
\end{itemize}
