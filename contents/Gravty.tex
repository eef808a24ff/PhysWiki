% 万有引力 引力势能
% 万有引力定律|引力场|保守场|引力势能

\pentry{牛顿运动定律\upref{New3}}

\subsection{万有引力和引力场}

若两个质点质量分别为 $m_1$ 和 $m_2$, 位置矢量分别为 $\vec r_1$ 和 $\vec r_2$, 则质点 1 对质点 2 的\bb{万有引力(gravitational force)} 为
\begin{equation}\label{Gravty_eq1}
\vec F_{12} =  - \frac{G m_1 m_2}{r_{12}^2} \uvec r_{12} = - \frac{G m_1 m_2}{\abs{\vec r_2 - \vec r_1}^3}(\vec r_2 - \vec r_1)
\end{equation}
其中 $r_{12} = \abs{\vec r_2 - \vec r_1}$ 是两点间的距离,$\uvec r_{12} = (\vec r_2 - \vec r_1)/\abs{\vec r_2 - \vec r_1}$ 是从 1 指向 2 的单位矢量.由该式, 质点 2 对质点 1 的万有引力为 $\vec F_{21} = -\vec F_{12}$, 符合牛顿第三定律.

我们类比高中所学电场的概念, 把以上 $m_1$ 对 $m_2$ 的作用力看做是 $m_1$ 在空间中产生的引力场对 $m_2$ 的作用力. 定义 $m_1$ 产生的\bb{引力场}为
\begin{equation}\label{Gravty_eq2}
\vec g(\vec r) = -\frac{Gm_1}{\abs{\vec r - \vec r_1} ^3}(\vec r - \vec r_1)
\end{equation}
若 $m_1$ 是某天体的质量, 这里的 $\vec g$ 就是它的重力加速度. 可见重力加速度会随位置的不同而变化.

\subsection{万有引力势能}
\pentry{力场\ 势能\upref{V}, 球坐标系中的梯度算符\upref{SphNab}}

在寻找万有引力的势能以前,我们先来证明所有具有 $\vec F(\vec r) = F(r) \uvec r$ 形式的力场都是保守场\upref{V}. 质点延一段轨迹 $\mathcal{L}$ 从 $\vec r_1$ 移动到 $\vec r_2$ 时,力场对质点做功\upref{Fwork} 可以用线积分\upref{IntL} 表示(以 $r$ 作为参数)
\begin{equation}\label{Gravty_eq3}
W = \int_\mathcal{L} \vec F(\vec r) \vdot \dd{\vec r} = \int_{r_1}^{r_2} F(r) \dd{r}
\end{equation}
其中第二步是因为
\begin{equation}
\vec F(\vec r) \vdot \dd{\vec r} = F(r)\uvec  r \vdot (\dd{r} \uvec  r + r\dd{\theta} \uvec \theta ) = F(r) \dd{r}
\end{equation}
显然线积分的结果只与初末位置(与原点的距离)有关,而与路径 $\mathcal{L}$ 的选择无关,$\vec F(\vec r)$ 是保守力场.

现在我们来寻找引力对应的势能.假设质量为 $M$ 的质点固定在坐标原点,考察质量为 $m$ 的质点位置矢量为 $\vec r$. 由于场对物体做功等于初势能减末势能\upref{V}
令质点沿着引力场从 ${\vec r_1}$ 延任意曲线移动到 $\vec r_2$, 我们有
\begin{equation}
V(\vec r_1) - V(\vec r_2) = \int_{r_1}^{r_2} F(r) \dd{r} =  - GMm\int_{r_1}^{r_2} \frac{1}{r^2} \dd{r}  =  - GMm \qty( \frac{1}{r_1} - \frac{1}{r_2} )
\end{equation}
可见任意位置的势能函数可以取
\begin{equation}
V(\vec r) = V(r) = - \frac{GMm}{r}
\end{equation}
根据势能的定义也可以给 $V(\vec r)$ 加上任意常数,但习惯上我们令无穷远处势能为 0,而上式恰好满足这点.拓展到任意具有 $\vec F(\vec r) = F(r)\uvec r$ 形式的力场,其势能可以用不定积分得到
\begin{equation}\label{Gravty_eq7}
V = -\int F(r) \dd{r}
\end{equation}
这与一维的情况相同.%链接未完成

我们也可以反过来通过引力势能求出引力场.使用球坐标的梯度算符\autoref{SphNab_eq1}\upref{SphNab} 得
\begin{equation}
\vec F = -\grad V = GMm \qty( \uvec r\pdv{r} + \uvec \theta \frac{1}{r}\pdv{\theta} + \uvec \phi \frac{1}{r\sin \theta}\pdv{\phi} )\frac{1}{r} =  - \frac{GMm}{r^2}\uvec r
\end{equation}
也可以使用直角坐标的梯度
\begin{equation}\label{Gravty_eq8}
\vec F = -\grad V = GMm \qty(\uvec x\pdv{x} + \uvec y\pdv{y} + \uvec z\pdv{z} )\frac{1}{\sqrt{x^2 + y^2 + z^2}}
\end{equation}
以 $x$ 分量为例
\begin{equation}
\pdv{x}(x^2 + y^2 + z^2)^{-1/2} =  - x(x^2 + y^2 + z^2)^{-3/2} =  - \frac{x}{r^3}
\end{equation}
另外两个分量类似可得 $- y/r^3$ 和 $- z/r^3$,代入\autoref{Gravty_eq8} 得
\begin{equation}
\vec F = -\grad V =  - \frac{GMm}{r^3} (x\uvec x + y\uvec y + z\uvec z) =  - \frac{GMm}{r^2}\frac{\vec r}{r} =  - \frac{GMm}{r^2}\uvec r
\end{equation}
