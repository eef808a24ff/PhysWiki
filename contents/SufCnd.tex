% 充分必要条件
% 充分条件|必要条件|命题

% 未完成: 画韦恩图, 引用韦恩图词条, 说明条件越多, 在韦恩图上范围就越小

若由命题 $A$ 能推导出命题 $B$, 则 $A$ 是 $B$ 的充分条件, $B$ 是 $A$ 的必要条件.如何理解这个定义呢?下面举两个例子.

\begin{example}{}
命题 $A$:四边形 $ABCD$ 是一个正方形.

命题 $B$:四边形 $ABCD$ 的四条边相等.

首先我们考虑 $A$ 对 $B$ 的关系.显然,由 $A$ 可以推出 $B$, 说明 $A$ 中有充分的信息能得到 $B$, 所以叫做 $B$ 的\textbf{充分条件}. $A$ 中包括得到 $B$ 所必要的信息,还\textbf{可能}包括一些其他信息,例如由命题 $A$ 可以得出四边形任意两条临边垂直. 这些多出来的信息并不一定是得到 $B$ 所必须的,因为还有许多其他的四边形四条边相等但并不是正方形.

那如何判断 $A$ 中有没有多余的信息呢?我们可以反过来试图用 $B$ 推导命题 $A$, 若原则上得不出 $A$ (而不是因为我们逻辑水平不够),则证明 $A$ 中有多余的条件.这时我们说 $A$ 不是 $B$ 的\textbf{必要条件},因为 $A$ 中的一些信息是多余的,也就是没有必要的.综上, $A$ 是 $B$ 的\textbf{充分非必要条件}.

现在我们从 $B$ 的角度考虑.虽然由条件 $B$ 不能推导出条件 $A$, 但是 $B$ 是 $A$ 中信息的一部分, $B$ 必须要成立才有可能使 $A$ 成立,也就是说如果 $B$ 不成立 $A$ 就不可能成立(四条边不全相等的四边形一定不是正方形).所以说 $B$ 是 $A$ 的必要条件.另外,由 $B$ 中的少量信息不能得到 $A$, 所以 $B$ 不是 $A$ 的充分条件. 综上, $B$ 是 $A$ 的\textbf{必要非充分条件}.
\end{example}


\begin{example}{}
命题 $A$ :三角形 $X$ 的其中两内个角分别为 90° 和 45°.

命题 $B$ :三角形 $X$ 有两个 45° 的内角.

利用三角形三个内角和为 180° 的事实,可以从 $A$ 推出 $B$, 说明 $A$ 是 $B$ 的充分条件, $B$ 是 $A$ 的必要条件.但也可以从 $B$ 推出 $A$, 说明 $B$ 是 $A$ 的充分条件, $A$ 是 $B$ 的必要条件.所以 $A$ 和 $B$ 既是彼此的充分条件也是彼此的必要条件.所以我们说 $A$ 和 $B$ \textbf{互为充分必要条件}.若 $A$ 是 $B$ 的充分必要条件, $B$ 一定也是 $A$ 的充分必要条件.因为两种表述都意味着 $A$,  $B$ 命题\textbf{等效},所提供的信息都是一样的,两者都没有任何多余的或者缺失的信息.
\end{example}

需要注意的是 
\begin{enumerate}
\item 充分/必要条件是两个命题之间的关系,若直说一个命题是充分/必要条件没有意义.
\item 讨论充分/必要条件需要在一定的前提下进行.以上两个例子中的前提如: 我们讨论的是欧几里得几何中的平面四边形和三角形. 当然,我们也可以把这个前提直接写在命题中.
\item 在证明 $A$ 是 $B$ 的充分必要条件时,需要分别证明 $A$ (相对于 $B$)的充分性和必要性.充分性需要由 $A$ 证明 $B$, 必要性需要由 $B$ 证明 $A$. 
\item 在证明 $A$ 是 $B$ 的充分非必要条件时,除了需要证明 $A$ 的充分性,还需非必要性,即 $B$ 不能推出 $A$. 只要我们可以举出一个 $B$ 成立 $A$ 不成立的反例,就立刻证明了不可能由 $B$ 推出 $A$. 
\end{enumerate}
