%法拉第电磁感应定律

\pentry{磁通量\upref{BFlux}}

\subsection{电磁感应定律的积分形式}

闭合线圈产生的感生电动势%未完成:链接
等于线圈内磁通量随时间的变化率.方向由楞次定律%未完成:链接
决定.

即
\begin{equation}
\varepsilon  =  -\dv{\Phi}{t} =  - \dv{t} \int \vec B \vdot \dd{\vec a} =  - \int \dv{\vec B}{t} \vdot \dd{\vec a}
\end{equation} 
另一方面,感生电动势是由感生电场产生的. 
\begin{equation}
\varepsilon  = \oint \vec E \vdot \dd{\vec r}
\end{equation} 
合并上面两式,得
\begin{equation}
\oint \vec E \vdot \dd{\vec r}  =  - \int \pdv{\vec B}{t} \vdot \dd{\vec a} 
\end{equation} 
如果我们假设感生电场只与电场的分布和变化率有关,则这个公式对空间中任何假想中的回路都成立,而不需要有真正的线圈存在.注意上式中的磁场是空间中的所有磁场. 
\subsection{电磁感应定律的微分形式}

旋度定理告诉我们,若对任意闭合回路,一个矢量场对曲面正方向的面积分等于另一个场在曲面边界线正方向的线积分,那么前者是后者的旋度.应用到上式, 可得电场的旋度为
\begin{equation}
\curl \vec E =  - \pdv{\vec B}{t}
\end{equation} 

