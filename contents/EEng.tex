% 电场的能量

首先我们来回顾一下平行板电容器(\autoref{Cpctor_exe2})的能量
\begin{equation}
W = \frac12 CV^2 = \frac12 \epsilon \frac Sd (Ed)^2 = \frac 12 \epsilon \tau E^2
\end{equation}
其中 $\tau = Sd$ 为平行板间长方体的体积. 这条公式容易让我们联想到电势能储存在电场之中. 在电动力学中, 这种理解是正确的, 我们不妨把电场含有的能量称为\bb{电场能}. 通过上式, 我们假设空间中任意一点的电场能密度为 $\epsilon E^2/2$, 则总电场能等于
\begin{equation}
W = \frac12 \epsilon \int \vec E(\vec r)^2 \dd[3]{r}
\end{equation}
可以证明该式定义的电场能与\autoref{QEng_eq6}\upref{QEng} 定义的电势能是完全等效的.

\subsection{证明}