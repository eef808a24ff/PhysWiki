% 超导唯象解释——伦敦方程
% 麦克斯韦|库仑规范|伦敦方程|伦敦规范

\pentry{麦克斯韦方程\upref{MWEq}, 库伦规范\upref{Cgauge}}

本章为唯象论,不涉及超导本质.考虑超导本质需用量子的角度去解释即BCS理论.
\subsection{伦敦第一方程}
假定超导体中有两种载流电子——正常传到电子与超导电子,对于超导电子有
\begin{equation}
\pdv{t}\bvec J_s=\alpha\bvec E \label{edy34_eq1}
\end{equation}
其中$\bvec J_s$代表超导体中的超导电流密度,$\alpha=n\dfrac {n_se^2}m$,该方程理解,在超导电子运动速度远小于光速$c$的情况下,磁力对超导电子的影响忽略,此时由$\bvec F=m \bvec a$写在电场中的形式$m\pdv{v}{t}=-e\bvec E$推出.
实验证明与超导体的相关性质吻合.为第二方程引出给出先决条件$E=0$,否则电子速度会不断上升.
\subsection{伦敦第二方程}
\begin{equation}
\curl\bvec J_s=-\alpha\bvec B \label{edy34_eq2}
\end{equation}
由\autoref{edy34_eq1}取旋度加之$\curl E=\pdv{B}{t}$推得.
\subsection{伦敦规范}
在库伦规范的前提下,矢势$\bvec A$并不唯一,为了使矢势$\bvec A$唯一确定,而在超导体表面$S$上引入限定$\bvec A$的法向分量为$0$,即
\begin{equation}
\div \bvec A=0
\end{equation}
\begin{equation}
\bvec e_n \cdot \bvec A|_s=0
\end{equation}
\subsection{伦敦方程解释超导现象}
\begin{theorem}{迈斯纳效应}
当材料处于超导态时,随着进入导体内部深度的增加,磁场迅速衰减,磁场主要存在于导体表面一定厚度的薄层内.
\end{theorem}
恒定情形时(正常电子所导致的电流为零,超导电流与深度有关.)对于超导体内的磁场和电流满足的麦克斯韦—伦敦方程为
\begin{equation}
\div \bvec B=0\label{edy34_eq3}
\end{equation}
\begin{equation}
\curl \bvec B=\mu_0 \bvec J_s\label{edy34_eq4}
\end{equation}
\begin{equation}
\div \bvec J_s=0\label{edy34_eq5}
\end{equation}
\begin{equation}
\curl \bvec J_s=-\alpha\bvec B
\end{equation}
将\autoref{edy34_eq4}取旋度,按照矢量乘积展开,同时代入\autoref{edy34_eq3}与\autoref{edy34_eq2}得
\begin{equation}
\laplacian \bvec B=\dfrac{1}{\lambda_L^2}\bvec B \label{edy34_eq7}
\end{equation}
其中 $\lambda_L= 1/\sqrt{\mu_0\alpha}=\sqrt{\dfrac m {\mu_0n_se^2}}$,由此可以看出磁场$\bvec B$随着位置变化迅速变化.方程解亦可能形式为磁场随着位置变化迅速上升或下降,但当考虑现实情况迈斯纳效应时,可以确定真解.

求解超导电流,通过对\autoref{edy34_eq2}取旋度,及\autoref{edy34_eq4}\autoref{edy34_eq3}的带入得
\begin{equation}
\laplacian\bvec J_s=\dfrac 1 {\lambda_L^2}\bvec J_s
\end{equation}
同\autoref{edy34_eq7}相同形式,故而可以得到相同的结果,即超导电流分布在超导体表面.
