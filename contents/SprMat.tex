% 稀疏矩阵

\bb{稀疏矩阵(Sparse Matrix)}有不同的储存方式(数据结构), 这里介绍几种常见的\footnote{参考 Wikipedia 的 Sparse Matrix 词条}.

\subsection{Banded}
Banded 矩阵只储存矩阵主对角线上下的若干条对角线, 上带宽和下带宽分别指定主对角线上面和下面有几条对角线, 例如三对角矩阵的上带宽和下带宽都是 1. 带内即使有矩阵元为零也必须储存. 这样就可以按照 row major 或者 column major 来储存.

\subsection{Coordinate List (COO)}
COO 格式列出非零矩阵元和对应的行标列标. 通常将它们储存为三个数组 \x{a}, \x{ia}, \x{ja}, 顺序任意. 除此之外, 有时还需要储存三个数组的长度 \x{nnz} (none zero) 以及矩阵的尺寸.

\subsection{Compressed Row Storage (CRS)}
也叫 Compressed Sparse Row (CSR), 储存为三个数组 \x{a}, \x{ia}, \x{ja}, 非零矩阵元按照 row major 的顺序储存, \x{ja} 是对应矩阵元的列标, 第 \x{n} 行矩阵元是 \x{a(ia(n) : ia(n+1)-1)}.

\subsection{Compressed Cow Storage (CCS)}
也叫 Compressed Sparse Cow (CSC), 与 CRS 一样, 只是改为 column major.
