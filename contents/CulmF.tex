% 库仑函数

\pentry{常微分方程\upref{ODE}}

\footnote{参考 \link{NIST 相关页面}{https://dlmf.nist.gov/33.2}}在球坐标系中解库仑势场中的薛定谔方程, %链接未完成
会得到径向方程
\begin{equation}
\dv[2]{u}{\rho} + \qty[1 - \frac{2\eta}{\rho} - \frac{l(l+1)}{\rho^2}]u = 0
\end{equation}
其解为\bb{第一类库仑函数} $F_l(\eta, \rho)$ 和\bb{第二类库仑函数} $G_l(\eta, \rho)$. 第一类库仑函数的解析式为\footnote{这里的两套正负号是等效的.}
\begin{equation}
F_l(\eta, \rho) = A_l(\eta) \rho^{l+1} \E^{\mp\I\rho} M(l+1\mp\I\eta, 2l+2, \pm 2\I\rho)
\end{equation}
其中
\begin{equation}
A_l(\eta) = \frac{2^l \E^{-\pi\eta/2} \abs{\Gamma(l+1+\I\eta)}}{(2l+1)!}
\end{equation}
$M(a, b, z)$ 是\bb{库默尔(Kummer) M 函数}, 也叫第一类合流超几何函数\upref{HypGeo}, 记为 $_1 F_1(a;b;z)$. 库仑函数也可以用\bb{惠特克(Whittaker) M 函数}来表示得更紧凑
\begin{equation}
F_l(\eta, \rho) = \qty(\frac{\mp\I}{2})^{l+1} A_l(\eta) M_{\pm\I\eta, l+\frac12}\qty(\pm 2\I\rho)
\end{equation}
其中惠特克 M 函数与库默尔 M 函数的关系为
\begin{equation}
M_{\kappa, \mu}(z) = \E^{-z/2} z^{\mu + 1/2} M(\mu - \kappa + 1/2, 1 + 2\mu, z)
\end{equation}
$F_l(\eta, \rho)$ 是一个实函数, 类似 $r$ 乘以第一类球贝赛尔函数\footnote{当 $\eta = 0$ 时二者相等.}, 有
\begin{equation}
F_l(\eta, 0) = 0 \qquad \eval{\dv{F_l(\eta, \rho)}{\rho}}_0 = 
\leftgroup{&A_0(\eta) &\qquad &(l = 0)\\ & 0 &\qquad &(l > 0)}
\end{equation}
且渐进形式为
\begin{equation}\label{CulmF_eq7}
\lim_{\rho\to \infty} F_l(\eta, \rho) = \sin[\rho - \pi l/2 - \eta\ln(2\rho) + \sigma_l(\eta)]
\end{equation}
其中 $\sigma_l(\eta)$ 是\bb{库仑相移(Coulomb phase shift)}
\begin{equation}\label{CulmF_eq8}
\sigma_l(\eta) = \arg[\Gamma(l+1+\I\eta)]
\end{equation}

第二类库仑函数 $G_l(\eta, \rho)$ 可以由 $H_l^\pm(\eta, \rho)$ 得到, 后者是两类库仑函数的线性组合\footnote{类比 $\exp(\I x) = \cos x + \I\sin x$}
\begin{equation}
H_l^\pm(\eta, \rho) = G_l(\eta, \rho) \pm \I F_l(\eta, \rho)
\end{equation}	
定义为
\begin{equation}\label{CulmF_eq10}
H_l^\pm(\eta, \rho) = (\mp\I)^l \E^{\pi\eta/2 \pm \I \sigma_l(\eta)} W_{\mp\I\eta, l + 1/2}(\mp 2\I\rho)
\end{equation}
其中 $W_{\kappa, \mu}(z)$ 是\bb{惠特克(Whittaker) W 函数}
\begin{equation}\label{CulmF_eq11}
W_{\kappa, \mu}(z) = \E^{-z/2} z^{\mu + 1/2} U(\mu - \kappa + 1/2, 1 + 2\mu, z)
\end{equation}
其中 $U(a, b, z)$ 是\bb{库默尔(Kummer) U 函数}, 定义为
\begin{equation}
U(a, b, z) = \frac{\Gamma(1 - b)}{\Gamma(a + 1 - b)} M(a, b, z) + \frac{\Gamma(b - 1)}{\Gamma(a)} z^{1 - b} M(a + 1 - b, 2 - b, z)
\end{equation}
\autoref{CulmF_eq11} 代入\autoref{CulmF_eq10} 得
\begin{equation}
H_l^\pm(\eta, \rho) = \E^{\pm\I\theta_l(\eta, \rho)}(\mp 2\I\rho)^{l + 1 \pm\I\eta} U(l + 1 \pm \I\eta, 2l + 2, \mp2\I\rho)
\end{equation}
其中
\begin{equation}
\theta_l(\eta, \rho) = \rho - \eta\ln(2\rho) - \frac12 l\pi + \sigma_l(\eta)
\end{equation}
是\autoref{CulmF_eq7} 中的渐进相位.

$G_l(\eta,\rho)$ 的渐进形式为
\begin{equation}
\lim_{\rho\to \infty} G_l(\eta, \rho) = \cos\theta_l(\eta, \rho)
\end{equation}

库仑函数的导数可以由惠特克函数的导数\footnote{可以使用 Wolfram Alpha 或 Mathematica}得到.
\begin{equation}
\pdv{z} [M_{a, b}(z)] = \qty(\frac12 - \frac{a}{z}) M_{a, b}(z) + \frac{1}{x} \qty(a + b + \frac12) M_{a+1, b}(z)
\end{equation}
\begin{equation}
\pdv{z} [W_{a, b}(z)] = \frac{1}{2z} \qty[(z - 2a) W_{a, b}(z) - 2W_{a+1, b}(z)]
\end{equation}
可见程序中求导数大概要比求函数值多用一倍时间(因为要求两次惠特克函数), 但同时也顺便求出了函数值. 如果同时需要两类库仑函数及它们的导数, 只需求 $H^+$ 函数的导数, 这就顺便求出了 $H^+$, 再分别取实部虚部即可. 另外只需要把 $H^+$ 做复共轭即可得到 $H^-$.

一种验证函数值是否正确的方法是使用以下性质
\begin{equation}
\mathcal{W}\{G, F\} = \mathcal{W}\{H^\pm, F\} = 1
\end{equation}
其中 $\mathcal{W}\{f_1(x), f_2(x)\}$ 是二阶\bb{朗斯基行列式(Wronskian determinant)}
\begin{equation}
\mathcal{W}\{f_1(x), f_2(x)\} = \begin{vmatrix}
f_1(x)  & f_2(x) \\
f_1'(x) & f_2'(x)
\end{vmatrix} = f_1(x) f_2'(x) - f_2(x) f_1'(x)
\end{equation}


由渐进形式可得径向归一化积分\footnote{积分时可忽略 $\sin$ 中的额外相位, 但我不会证.}与球贝赛尔函数的相同
\begin{equation}
\int_0^\infty F_l(Z/k', k' r)F_l(Z/k, kr) \dd{r} = \int_0^\infty \sin(k'r)\sin(kr) \dd{r} = \frac{\pi}{2}\delta(k - k')
\end{equation}