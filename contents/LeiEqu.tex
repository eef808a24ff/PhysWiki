% 莱布尼兹公式

\pentry{导数\upref{Der}, 高阶导数\upref{HigDer}}

设函数 $u=u(x), v=v(x)$ 均有 $n$ 阶导数, 形如 $y=uv$ 的函数的 $n$ 阶导数, 可由\bb{莱布尼兹(Leibniz)公式}求出.

不加推导的给出莱布尼兹公式,可用数学归纳法证明.
\begin{equation}
\begin{aligned}
(uv)^{(n)}=&u^{(n)}v+nu^{(n-1)}v'+\frac{n(n-1)}{2!}u^{(n-2)}v'' \\
&+ \cdots +\frac{n(n-1) \cdots (n-k+1)}{k!}u^{(n-k)}v^{(k)}+\cdots+uv^{(n)}
\end{aligned}
\end{equation}

规定一个函数的零阶导数等于函数本身,即$u^{(0)}=u$, 于是莱布尼兹公式可以写成如下的形式
\begin{equation}
(uv)^{(n)}=\sum_{k=0}^{n} C_{n}^{k}u^{(u-k)}v^{(k)}
\end{equation}

\begin{exam}{求函数 $y=xe^{-x}$ 的 $n$ 阶导数}
可以通过莱布尼兹方程对该函数直接求 $n$ 阶导数.
\begin{equation}
\begin{aligned}
(x\E^{-x})^{(n)}&=\sum_{k=0}^{n} C_{n}^{k}(\E^{-x})^{(n-k)}\\
&=x(\E^{-x})^{(n)}+n(\E^{(-x)})^{(n-1)}\\
&=x(-1)^n\E^{-x}+n(-1)^{n-1}\E^{-x}\\
&=(-1)^{n-1}(n-x)\E^{-x}
\end{aligned}
\end{equation}
\end{exam}
