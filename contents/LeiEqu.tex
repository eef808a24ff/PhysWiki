%Leibniz公式
\pentry{导数\upref{Der},高阶导数\upref{HigDer}}

设函数$u=u(x),v=v(x)$均有n阶导数,形如$y=uv$的函数的n阶导数,可由Leibniz公式求出.

不加推导的给出Leibniz公式,可用数学归纳法证明之.
\begin{equation}
\begin{aligned}
(uv)^{(n)}=&u^{(n)}v+nu^{(n-1)}v'+\frac{n(n-1)}{2!}u^{(n-2)}v'' \\
&+ \cdots +\frac{n(n-1) \cdots (n-k+1)}{k!}u^{(n-k)}v^{(k)}+\cdots+uv^{(n)}
\end{aligned}
\end{equation}

规定一个函数的零阶导数等于函数本身,即$u^{(0)}=u$,于是Leibniz公式可以写成如下的形式
\begin{equation}
(uv)^{(n)}=\sum_{k=0}^{n} C_{n}^{k}u^{(u-k)}v^{(k)}
\end{equation}

\begin{exam}{求函数$y=xe^{-x}$的n阶导数}
	可以通过Leibniz方程对该函数直接求n阶导数.
	\begin{equation}
	\begin{aligned}
	(xe^{-x})^{(n)}&=\sum_{k=0}^{n} C_{n}^{k}(e^{-x})^{(n-k)}\\
	&=x(e^{-x})^{(n)}+n(e^{(-x)})^{(n-1)}\\
	&=x(-1)^ne^{-x}+n(-1)^{n-1}e^{-x}\\
	&=(-1)^{n-1}(n-x)e^{-x}
	\end{aligned}
	\end{equation}
\end{exam}
