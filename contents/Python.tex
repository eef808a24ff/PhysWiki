% Python 入门

到 2019 年, Python 已经成为最流行的语言, 无论是在科学计算还是在计算机领域都有大量的使用者. 本书中我们主要介绍 python 在科学计算中的应用.

如果不想安装软件可以直接使用用浏览器访问 \href{https://jupyter.org/}{Jupyter Notebook} 运行 Python 程序, 要在本地使用 Python 推荐安装 Anoconda. 以下我们用前者进行讲解. Jupyter Notebook 的优点是交互式编程, 即每输入一个命令都可以立即执行(快捷键 Shift + Enter), 利于学习和实验.

\subsection{作为计算器}
请在 Jupyter Notebook 中尝试输入以下命令并执行(运行结果略). Python 程序使用 \lstinline|#| 注释

四则运算
\begin{lstlisting}[language=python]
2 + 2
\end{lstlisting}
\begin{lstlisting}[language=python]
123 / 456
\end{lstlisting}
幂运算
\begin{lstlisting}[language=python]
3 ** 2
\end{lstlisting}
整数除法, 即相除再向下取整
\begin{lstlisting}[language=python]
4 // 3
\end{lstlisting}
求余
\begin{lstlisting}[language=python]
4 % 3 # 使得 a = a // b + a % b 恒成立
\end{lstlisting}
使用括号
\begin{lstlisting}[language=python]
(123 - 234*2)**2 / (34 + 54**4)
\end{lstlisting}
各种数学函数都在 math 模块中, 需要加载.
\begin{lstlisting}[language=python]
import math
\end{lstlisting}
使用模块中的函数, 要在前面加上模块名和一点. 例如开方 (square root)
\begin{lstlisting}[language=python]
math.sqrt(284)
\end{lstlisting}
自然指数函数
\begin{lstlisting}[language=python]
math.exp(5.1)
\end{lstlisting}
这样做虽然略显麻烦, 但可以区分不同模块中同名函数. 在确保没有冲突的情况下我们也可以用以下方法加载模块中的指定函数, 如
\begin{lstlisting}[language=python]
from math import sqrt, exp, sin, cos
\end{lstlisting}
现在使用这些函数就不需要 \lstinline|math.| 的前缀了
\begin{lstlisting}[language=python]
sin(1)
\end{lstlisting}
我们甚至可以用这种方式引入所有函数而无需前缀(但名称冲突的可能性更大, 不建议使用)
\begin{lstlisting}[language=python]
from math import *
\end{lstlisting}
从模块中不仅可以引入函数, 还有常数, 例如圆周率和自然对数底(注意 e 这种单字母名称很可能会产生冲突, 所以不建议取消 \lstinline|math.| 前缀)
\begin{lstlisting}[language=python]
sin(pi/2)
\end{lstlisting}
\begin{lstlisting}[language=python]
log(e)
\end{lstlisting}

math 模块中的其他常用函数如: 绝对值 (absolute value)
\begin{lstlisting}[language=python]
fabs(-32)
\end{lstlisting}
自然对数
\begin{lstlisting}[language=python]
log(0.5)
\end{lstlisting}
以 10 为底的对数
\begin{lstlisting}[language=python]
log10(1000)
\end{lstlisting}
弧度转为角度
\begin{lstlisting}[language=python]
degrees(pi/2)
\end{lstlisting}

\subsection{变量}
要计算一个长方体的面积, 我们直接把三个数字相乘, 也可以先把这三个数字赋值给三个\textbf{变量}然后相乘
\begin{lstlisting}[language=python]
a = 1
b = 2
c = 3
area = a*b*c
\end{lstlisting}
在 Jupyter Notebook 执行这四行发现并没有输出, 无论是一次性执行还是分开执行. 这是因为赋值命令默认不输出结果. 要强制输出结果可以用
\begin{lstlisting}[language=python]
print(area)
\end{lstlisting}
