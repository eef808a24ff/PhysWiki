% 3j 符号

\pentry{CG 系数\upref{SphCup}}

每个 3j 符号和 CG 系数都一一对应且相差一个常数, 用圆括号表示 3j 符号, 方括号表示 CG 系数, 有
\begin{equation}
\pmat{j_1 &j_2 &j_3\\ m_1 &m_2 &m_3}
= \frac{(-1)^{j_1 - j_2 - m_3}}{\sqrt{2j_3 + 1}} \bmat{j_1 &j_2 &j_3\\ m_1 &m_2 &-m_3}
\end{equation}

\subsection{对称性}
3j 符号具有很好的对称性. 首先,任意交换两列等于在前面加 $(-1)^{j_1+j_2+j_3}$
\begin{equation}
\ali{
&\pmat{j_3 &j_2 &j_1\\ m_3 &m_2 &m_1}
= \pmat{j_1 &j_3 &j_2\\ m_1 &m_3 &m_2}
= \pmat{j_2 &j_1 &j_3\\ m_2 &m_1 &m_3}\\
&= (-1)^{j_1+j_2+j_3}\pmat{j_1 &j_2 &j_3\\ m_1 &m_2 &m_3}
}\end{equation}
如果交换两次, 3j 符号不变
\begin{equation}
\pmat{j_1 &j_2 &j_3\\ m_1 &m_2 &m_3}
= \pmat{j_2 &j_3 &j_1\\ m_2 &m_3 &m_1}
= \pmat{j_3 &j_1 &j_2\\ m_3 &m_1 &m_2}
\end{equation}
将第二行取相反数也等于在前面加 $(-1)^{j_1+j_2+j_3}$
\begin{equation}
\pmat{j_1 &j_2 &j_3\\ -m_1 &-m_2 &-m_3}
= (-1)^{j_1+j_2+j_3} \pmat{j_1 &j_2 &j_3\\ m_1 &m_2 &m_3}
\end{equation}

\subsection{选择定则}
3j 符号的选择定则直接告诉我们哪些 3j 符号等于 0. 有了选择定则, 我们就无需计算不符合定则的 3j 符号.

从 CG 系数的选择定则可得\bb{三角不等式}(三个不等式等效)
\begin{equation}
\abs{j_1 - j_3} \les j_2 \les j_1 + j_3 \qquad
\abs{j_2 - j_3} \les j_1 \les j_2 + j_3 \qquad
\abs{j_3 - j_1} \les j_2 \les j_3 + j_1
\end{equation}
以及
\begin{equation}
m_1 + m_2 + m_3 = 0
\end{equation}

除此之外, 以上每个对称性也可以得到一个选择定则: 当 $j_1 + j_2 + j_3$ 为奇数时, 如果任意两列相同, 结果为 0
\begin{equation}\label{ThreeJ_eq7}
\pmat{j &j &j_3\\ m &m &m_3} = \pmat{j_1 &j &j \\ m_1 & m & m} =  \pmat{j &j_2 &j \\ m & m_2 & m} = 0
\end{equation}
当 $j_1 + j_2 + j_3$ 为奇数时, 若 $m_1 = m_2 = m_3 = 0$, 结果也为 0
\begin{equation}\label{ThreeJ_eq8}
\pmat{j_1 &j_2 &j_3\\ 0 & 0 & 0} = 0
\end{equation}
