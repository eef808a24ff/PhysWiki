% 公理系统
% 公理|定义|定理|证伪|真命题

\subsection{公理,定义和定理}
一个\textbf{命题(proposition)}就是一个陈述句.我们讨论问题时会大量使用命题来作描述,这些命题有的是成立的,称为\textbf{真命题(true proposition)};反之,有些命题不成立,称为\textbf{假命题(false proposition)}.比如说,我们通常认为命题“水是剧毒的”是一个假命题.但是这个命题是对于我们人类的生理学而言才假的,对于某些生物,水确实是致命毒药.也就是说,一个命题是真是假,要看该命题被放在什么样的环境下讨论的;讨论问题的环境,被称为\textbf{理论(theory)}.在理论限定的框架下讨论问题,本质上是在判断一个个命题在给定框架下的真假性,判断这一动作被称为\textbf{证明(prove)}.特别地,在汉语中,有时也会使用“证伪”一词,来简略表达“证明为假”的意思.

我们是不可能证明所有的命题的,所以任何理论必须有一个出发点,也就是一些基础命题.这些基础命题本身不可证明,而是被默认成立;它们决定了理论的样貌,理论中一切其它命题都是由这些命题根据逻辑推演得到的.这样的基础命题被称为一个理论的\textbf{公理(axiom)}.合格的公理系统中,公理之间彼此不能矛盾,比如命题“1是整数”和“1不是整数”不能作为同一个公理系统的两条公理.

有了公理系统以后,我们还需要明确所讨论的对象是什么.比如我用了皮亚诺公理来定义小学四则运算,那么为了讨论“$1+1$ 等于几”这样的问题,我首先需要明确“1”和“$+$”具体指什么.用来明确概念的命题,被称为\textbf{定义(definition)}.如果说公理系统是创建了一个宇宙的基础参数的话,那么定义就是在给这个宇宙里已经自然存在的事物进行命名,这样才能讨论这些事物.当我们说“把……称为”时,也是一种下定义的方式.

最后,任何一个理论的绝大部分内容都是在使用基础命题来进行推演,看哪些命题为\textbf{真}.\textbf{真命题}被称为\textbf{定理(theorem)}. 有时候,根据定理作用的不同,我们也可能称其中一些为\textbf{引理(lemma)}、\textbf{推论(corollary)}等.所有定理加在一起就构成了整个理论.

不同的公理系统可能推演出相同的命题,也可能推演出彼此矛盾的命题,更可能存在一些无法判断是否成立的命题.一个公理系统中所无法判断是否成立的命题,就叫做独立于这个公理系统的命题.这就好比我在描述一个多面体,使用公理“这个多面体有六个相同的面,且六个面彼此要么垂直要么平行”,那么就可以推演出定理“六个面都是正方形”,但是命题“这个多面体是蓝色的”就是独立于所给公理的命题,无法被证明或证伪.

如果两个公理系统能够推演出完全一样的命题(定理),那么这两个公理系统就是等价的.如果公理系统$A$能够推演出公理系统$B$的一切定理,但是$B$不能推出$A$中的一切定理,即$A$能推演出的一些定理实际上是独立于$B$的命题,那么可以认为是公理系统$A$包含了公理系统$B$.\textbf{在阅读本段话时,请注意命题和定理的区别:定理是在给定公理体系下的真命题.}

特别地,\textbf{哥德尔第一不完备性定理(Gödel's first incompleteness theorem)}说明,一切含有小学算术的公理系统中,总存在一些定理,如果它们为真,那么永远无法证明它们为真.这意味着系统里有一些真命题,但是我们没有任何能力意识到它们是真命题.由于还存在很多很难证明的命题,在一个命题被证明前,我们很难知道它究竟是假命题、可以证明的真命题,还是永远证不出来的真命题.前两类假以时日都可以通过证明得知命题的真假,但最后一类永远都会是吊人胃口的难题,因为你永远都没法知道它到底能不能被证出来.
