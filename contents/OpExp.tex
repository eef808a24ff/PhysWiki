% 算符的指数函数 波函数传播子
% 算符|量子力学|指数函数|波函数|传播子

\pentry{一阶线性常微分方程组\upref{ODEsys}}

\subsection{含时薛定谔方程的解}

量子力学中的算符和有限维矩阵的性质往往有相同之处, 然而当拓展到无穷维的情况时往往就需要高级得多的数学(泛函分析), 我们暂不详细介绍这些数学, 而是直接通过类比给出结论.

例如, 我们把哈密顿算符 $H$ 看作是无穷维矩阵, 薛定谔方程可记为与一阶线性常微分方程组(\autoref{ODEsys_eq1}\upref{ODEsys})相同的形式
\begin{equation}
\dv{t} \ket{\psi(t)} = -\I H \ket{\psi(t)}
\end{equation}
当哈密顿算符 $H$ 不含时, 解为(根据\autoref{ODEsys_eq2}\upref{ODEsys})
\begin{equation}
\ket{\psi(t)} = \exp(-\I H t) \ket{\psi(0)}
\end{equation}
当哈密顿算符含时, 形式上可以把解记为
\begin{equation}
\ket{\psi(t)} = \Q {\mathcal T} \exp(\int_0^ t H(t')\dd t') \ket{\psi(0)}
\end{equation}
我们把以上的 exp 项称为\textbf{传播子(propagator)}, 其定义依然是使用指数函数的级数展开.
