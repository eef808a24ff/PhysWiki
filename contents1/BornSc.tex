% 波恩近似(散射)

我们还是要解出连续态的不含时波函数,且无穷远处的动量为 $\vec k_i$ (入射平面波的动量).
\begin{gather}
-\frac{\hbar^2}{2m} \laplacian\psi  + V\psi  = E\psi \\
(\laplacian + k^2)\psi  = \frac{2m}{\hbar^2} V\psi  \equiv U(\vec r)\psi
\end{gather}
这是非齐次亥姆霍兹方程,其格林函数% 未完成: 格林函数 亥姆霍兹方程的格林函数
为(球面波)
\begin{equation}
G(R) =  -\frac{\E^{\I kR}}{4\pi R}
\end{equation}
满足
\begin{equation}
(\laplacian + k^2)G(\vec r) = \delta^3(\vec r)
\end{equation}
\bb{薛定谔方程的积分形式}为
\begin{equation}\label{BornSc_eq5}
\psi(\vec r) = \psi_0(\vec r) + \int G(\abs{\vec r - \vec r'})U(\vec r')\psi (\vec r') \dd[3]{r'}
\end{equation}
$\psi_0$ 是自由粒子波函数,由于无穷远处积分项消失($1/r$), $\psi_0(\vec r\to\infty)$ 要求具有动量 $\vec k_i$,唯一的选择是平面波
\begin{equation}
\psi_0(\vec r) = A\E^{\I\vec k_i \vdot \vec r}
\end{equation}
由于微分截面定义在无穷远处,我们把格林函数取无穷远处的极限(远场),注意这个极限在定义中,所以并不算是一个近似.这是关于 $\vec r'$ 的平面波
\begin{gather}
\abs{\vec r - \vec r'} \approx r - \hat r \vdot \vec r' \approx r\\
G(\vec r, \vec r') =  - \frac{\E^{\I k \abs{\vec r - \vec r'}}}{4\pi\abs{\vec r - \vec r'}} \to  - \frac{\E^{\I kr}}{4\pi r} \E^{-\I \vec k_f \vdot \vec r'}
\end{gather}
其中 $\vec k_f = k\uvec r$ 是出射的方向,注意 $\abs{\vec k_i} = \abs{\vec k_f}$ 意味着弹性散射.

积分方程求近似解的一般方法是先把一个近似解代入积分内,积分得到一阶修正后的解,再次代入,得到二阶修正后的解,以此类推迭代.波恩近似中,假设势能相对于入射动能较弱,积分项相当于微扰,所以令初始(零阶)波函数为 $\psi_0(\vec r)$.代入\autoref{BornSc_eq5} 得一阶修正的波函数,叫做\bb{第一波恩近似}
\begin{equation}
\psi ^{(1)}(\vec r) = A\E^{\I\vec k_i \vdot \vec r} - A \frac{m}{2\pi\hbar^2} \frac{\E^{\I kr}}{r}\int \E^{\I (\vec k_i - \vec k_f) \vdot \vec r'} V(\vec r') \dd[3]{r'}
\end{equation}
根据定义,散射幅为
\begin{equation}
f(k, \hat r) =  - \frac{m}{2\pi\hbar^2} \int \E^{\I (\vec k_i - \vec k_f) \vdot \vec r'} V(\vec r') \dd[3]{r'}
\end{equation}
这相当于势能函数的空间傅里叶变换.

\subsection{高阶波恩近似}
把\autoref{BornSc_eq5} 多次代入\autoref{BornSc_eq5} 的积分中,得到精确解的“积分级数”形式
\begin{equation}\ali{
\psi(\vec r) &= \psi_0(\vec r) + \int \dd[3]{r'} G(k,\vec r,\vec r')U(\vec r')\psi_0(\vec r')  \\
&+ \int \dd[3]{r'} G(k,\vec r,\vec r')U(\vec r') \int \dd[3]{r''} G(k,\vec r',\vec r'')U(\vec r'')\psi_0(\vec r'') \\
&+ \int \dd[3]{r'} G(k,\vec r,\vec r')U(\vec r')\times\\
&\int \dd[3]{r''}G(k,\vec r',\vec r'')U(\vec r'')\int \dd[3]{r'''}G(k,\vec r'',\vec r''')U(\vec r''') \psi_0(\vec r''')
  ... 
}\end{equation}
若只计算是指包含前 $n$ 行,就叫第 $n$ 波恩近似.具体计算时,偶尔会用到二阶,基本不会用到三阶或以上.

非常有趣的是,即使我们不假设零阶波函数是平面波,波函数展开成上式时取前 $n$ 行的结果仍然是相同的.
