% 质点系

\pentry{牛顿第三定律\upref{New3}}

在考虑多个物体构成的系统时, 我们有时候可以把每个物体都近似为一个质点, 这样我们就得到了由有限个质点构成的系统, 称为\bb{质点系}.

令质点系中有 $N$ 个质点, 每个质点的受力都可以分为两类, 一是系统外界物体给该质点的力, 称为\bb{外力}, 二是来自系统内其他质点的力, 称为\bb{内力}. 对第 $i$ 个质点, 将外力和内力分别记为 $\vec F_i^{out}$ 和 $\vec F_i^{in}$. 

系统中所有质点所受的\bb{合力}等于\bb{合内力}加\bb{合外力}
\begin{equation}
\vec F_{tot} = \vec F_{tot}^{in} + \vec F_{tot}^{out} = \sum_i^N \vec F_i^{in} + \sum_i^N \vec F_i^{out}
\end{equation}
若将第 $j$ 个质点对第 $i$ 个质点的内力记为 $\vec F_{j\to i}$ 则上式中
\begin{equation}
\sum_i \vec F_i^{in} = \sum_{i,j}^{i \ne j} \vec F_{j \to i}
\end{equation}
任意两个质点 $k$ 和 $l$ 对该求和的贡献是一对相互作用力 $\vec F_{k \to l} + \vec F_{l \to k}$,而根据牛顿第三定律,相互作用力之和为零.所以上式求和为零. 所以, 质点系中合内力为零, 系统所受合力等于合外力
\begin{equation}
\vec F_{tot} = \vec F_{tot}^{out} = \sum_i^N \vec F_i^{out}
\end{equation}