% Crank-Nicolson TDSE 笔记

\subsection{氢原子的数值误差}
解三维氢原子的时候, 为了防止势能为无穷, 中心点要选在八个格点中间。

首先用的 $60\times 60\times 60$ a.u. 的范围, $401\times 401\times 401$ 的格点, $0.032$ a.u. 的时间步长。

TDSE 传播时, 如果把 log 概率灰度图的范围取得很大 ($10^{-25}$ 左右), 即使没有电场, 也会出现一个方形的花纹很快地扩散出来, 然后波函数还会向外温柔地扩散。

比较了用解析的基态波函数和 Imaginary Time 得到的基态波函数, 虽然两者都有类似情况, 但是后者的程度明显要轻。

爱华说 XUV 的强度并不太重要, 频率越大, 氢原子的截面就越小, 所以反而需要更大的强度才能电离出同样的概率。 但只要不超过 $1\times 10^{16} \Si{W/cm^2}$ (0.5338 a.u.)就好。

\subsection{Streaking}
这里的目标是用 TDSE 模拟出氢原子的 Streaking 谱, XUV 和 IR 参数就用 Ivanov 的论文 (Attosecond recorder of ...) 中的。

XUV 是 250 as, 41 eV (动能 1.007 a.u.), 还有 varying ellipticity, 最大强度约 $5\times 10^{-4}$ a.u. (不重要), IR 是 800 nm, 7.2 fs (297.5 a.u.), 强度 $3.5\times 10^{12} \Si{W/cm^2}$ (0.01 a.u.)。

用经典电子来估计 TDSE 所需 box 的大小, 速度为 $\sqrt{2}$ 左右, 若最大传播时间为 IR 长度的两倍, 可以走 841 a.u., 比我测试 XUV 用的 60 a.u. 盒子大了一个数量级。 再加上我的直角坐标网格不能改变长度, 结果就非常地惨烈了。

唯一有可能挽救我的程序的方法就是只模拟到 XUV 消失, 这时只要保证波函数还在盒内就可以做一个 FFT 然后用 Volkov 波函数来解析传播。 但是 XUV 消失的时候电离波包也就离原子核 40 a.u. 左右, 所以如果这时开始忽略库仑力, 那么动能谱将会有 2.5\% 的误差, 这应该是不能接受的。 即使波函数继续传播到 100 a.u. 开外, 动能普也会有 1\% 的误差。 所以师兄还是老的辣, 第一反应就是不行。

所以最后的解决方法只能是用参考文献 29 的超级复杂的球坐标 TDSE 代码了(Fortran)。
