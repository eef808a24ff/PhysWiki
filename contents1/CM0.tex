% 经典力学简介

\subsection{经典力学}
讲的是一堆(宏观低速)的粒子(质点)受一堆力以后的运动情况.牛顿三定律可以作为经典力学的公设,可以将其想象成一个懂经典力学的电脑,只需输入初始时刻所有粒子的状态(位置和速度/动量),以及每个粒子的受力/势能函数,就可以得到接下来每个粒子的运动方式(位置关于时间的函数).

\subsection{分析力学}
分析力学并没有给基于牛顿定律添加新的物理,只是把牛顿定律换成拉格朗日方程或者哈密顿方程(分析力学能算出来的东西用牛顿定律也能算出来).后者相当于一个更智能的电脑,对于一些粒子,无需告诉电脑它的受力只需告诉电脑它运动的约束即可.约束简单来说就是怎样的运动是不可能的:例如单摆中的质点就“不可能”沿着绳的方向运动,两个咬合的齿轮“不可能”一个转动一个不转.

\subsection{狭义相对论}
狭义相对论,讨论的问题相同.牛顿三定律只适用于宏观低速弱引力场条件,如果粒子的速度相对光速不可忽略,那就需要使用“更精确”的牛顿第二定律,且惯性参考系切换也变得更复杂(洛伦兹变换).当速度越低,狭义相对的计算结果越接近牛顿力学.相对论同样并不适用微观.

\subsection{广义相对论}
加上强引力场和非惯性系.
