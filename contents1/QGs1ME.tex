%量子气体(单能级巨正则系综法)

我们可以把一个包含许多粒子的系统看做热池, 把每个能级 ${\varepsilon_i}$ 看做一个系统. 为了便于理解, 可以把能级想象成一个盒子, 所有处于该能级的粒子都在盒内, 都具有能量 $\varepsilon_i$. 当 $\varepsilon_i$ 系统中粒子数为 $n_i$ 时, 系统的总能量为 $E = n_i \varepsilon_i$.  注意对于一个 $n_i$,  由于同种粒子不可区分, 系统只有一种状态, 所以在当前系统的巨配分函数中, 对能量的求和只有一项.
\begin{equation}\ali{
\Xi & = \sum_{n_i} \sum_{E_j}^{} \E^{(n_i\mu - E_j)\beta} = \sum_{n_i} \E^{(n_i\mu - E)\beta}  \\
& = \sum_{n_i} \E^{(n_i\mu - n_i \varepsilon_i)\beta} = \sum_{n_i} \qty[\E^{(\mu - \varepsilon_i)\beta}]^{n_i}
}\end{equation}
系统( $\varepsilon_i$ 能级)中的平均粒子数为
\begin{equation}\ali{
\expval{n_i} & = \frac{1}{\Xi }\sum_{n_i} \sum_{E_j} n_i \E^{(n_i\mu - E_j)\beta}\\
& = \frac{1}{\Xi } \sum_{n_i} n_i \qty[\E^{(\mu - \varepsilon_i)\beta }]^{n_i}
}\end{equation}
\subsection{费米子}
由于泡利不相容原理, 一个能级只能存在 $0$ 或 $1$ 个费米子(这里忽略自旋).
\begin{equation}
\Xi  = \sum_{n_i = 0}^1 {{\qty[\E^{(\mu - \varepsilon_i)\beta}]}^{n_i}}  = 1 + \E^{(\mu - \varepsilon_i)\beta}
\end{equation}
 ${\varepsilon_i}$ 能级的平均粒子数为
\begin{equation}
\begin{aligned}
\expval{n_i} & = \frac{1}{\Xi }\sum_{n_i=0}^1 {n_i}{{\left[ {\E^{(\mu - \varepsilon_i)\beta}} \right]}^{n_i}} = \frac{0 + \E^{(\mu - \varepsilon_i)\beta }}{1 + \E^{(\mu - \varepsilon_i)\beta}}  = \frac{1}{\E^{(\varepsilon_i - \mu)\beta} + 1}
\end{aligned}
\end{equation}
这就是著名的\bb{费米—狄拉克分布}.

\subsection{玻色子} 
任何能级都允许同时存在任意数量的玻色子, 所以上面两式中对 ${n_i}$ 的求和上限变为正无穷即可(见等比数列求和%链接未完成
以及类等比数列求和%连接未完成
). 但为了使求和收敛, 必须要求 $\E^{(\varepsilon_i - \mu)\beta} - 1 > 0$,  或者 $\mu  < \varepsilon_i$. 
\begin{equation}
\Xi  = \sum_{n_i = 0}^\infty \qty[\E^{(\mu - \varepsilon_i)\beta}]^{n_i}  = \frac{1}{1 - \E^{(\mu - \varepsilon_i)\beta}}
\end{equation}
\begin{equation}\ali{
\expval{n_i} & = \frac{1}{\Xi } \sum_{n_i = 0}^1 n_i \qty[\E^{(\mu - \varepsilon_i)\beta}]^{n_i} = \qty[1 - \E^{(\mu - \varepsilon_i)\beta}] \frac{\E^{(\mu - \varepsilon_i)\beta }}{\qty[1 - \E^{(\mu - \varepsilon_i)\beta}]^2}  = \frac{1}{\E^{(\varepsilon_i - \mu)\beta} - 1}
}\end{equation}
这就是著名的\bb{玻色—爱因斯坦分布}.

当每个能级的平均粒子数 $\expval{n_i}$ 都很小时, 即 $\expval{n_i} \ll 1$ 时, $\expval{n_i} = 1/(\E^{(\varepsilon_i - \mu)\beta} \pm 1)$ 的分母 $ \gg 1$,  分布可以近似为
\begin{equation}
\expval{n_i} = \frac{1}{\E^{(\varepsilon_i - \mu )\beta }} = \E^{(\m  - \varepsilon_i)\beta}
\end{equation}
这就是\bb{麦克斯韦—玻尔兹曼分布}, 对应理想气体. 由此可见, 当 % 未完成???

该分布的总粒子数为
\begin{equation}
N = \sum_i^\infty \expval{n_i} = \E^{\mu \beta}\sum_i^\infty  \E^{-\varepsilon_i\beta}  = zQ_1
\end{equation}
为了验证该式的正确性, 代入理想气体的化学势和单粒子配分函数, 上式成立.
\begin{equation}
\mu  = kT\ln \frac{N\lambda^3}{V}  \qquad
Q_1 = \frac{V}{\lambda ^3}
\end{equation}
这种方法虽然可以简单地求出分布函数, 但却不能求出其他物理量, 例如量子气体的压强, 熵, 等. 因为我们的系统只包含一个能级, 而不是大量粒子. 要使用标准的巨正则系综, 必须把包含大量粒子的量子气体作为系统, 并考虑每个粒子数对应的所有可能的能级分布.