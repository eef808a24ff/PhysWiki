% 代数矢量

\pentry{正交归一基底\upref{OrNrB}, 复数\upref{CplxNo}}

我们可以把 $N$ 个有序的数排成一行或一列, 分别称为\bb{代数行矢量}或\bb{代数列矢量}, 简称\bb{行矢量}或\bb{列矢量}, 本书中同样用加粗正体字母表示. 若一个代数矢量中所有的数都是实数, 且 $N \les 3$%未完成: 斜线的小于等于号是什么命令?
时, 我们可以把它看做是一个几何矢量在一组(往往是正交归一的)基底上的展开\autoref{GVec_eq5}. 这样就可以很自然地定义这种代数矢量的加法和数乘分别为\autoref{GVec_eq8} 和\autoref{GVec_eq9}.

当 $N > 3$ 或代数矢量中含有复数时, 不存在对应的几何矢量, 但我们仍然可以定义加法和数乘分别为
\begin{equation}
\vec u + \vec v = \sum_{i = 1}^N u_i + v_i
\end{equation}
\begin{equation}
\lambda \vec v = \sum_{i = 1}^N \lambda v_i
\end{equation}
注意 $N$ 维行矢量只能与 $N$ 维行矢量相加, 列矢量也一样. 有了这两个定义, 我们就可以类似几何矢量得到其他重要的概念, 如\bb{共线}, \bb{线性组合}, \bb{线性相关}, \bb{线性无关}, \bb{基底}, 和\bb{坐标}. 代数矢量也可以组成许多不同的矢量空间, 例如所有的 $N$ 维的实数行矢量组成一个 $N$ 维矢量空间, 所有的 $N$ 维复数列矢量组成一个 $N$ 维矢量空间, 等等.

为了便于理解, 我们仍然可以想象代数矢量与某种抽象的矢量一一对应, 这种抽象矢量不依赖于任何基底, 其对应的代数矢量是这些个体在某组基底上的展开系数(线性组合的系数).

\begin{exam}{用代数矢量表示多项式}
作为上述“抽象矢量” 的一个例子, 我们可以把以所有小于 $N$ 阶的多项式
\begin{equation}
P(x) = \sum_{n = 0}^{N-1} c_n x^n
\end{equation}
看做一个 $N$ 维矢量空间, 矢量的加法就是多项式相加, 数乘就是多项式与常数相乘. 一组简单的基底是
\begin{equation}
\vec \beta_n = x^n \qquad (n = 0, 1, \dots, N-1)
\end{equation}
显然这组基底是线性无关的, 且空间中所有矢量(多项式)都可以在该基底上展开. 任意 $N$ 维代数矢量 $(c_0, \dots, c_{N-1})\Tr$ 可以看做是一个多项式的坐标, 且有一一对应的关系. 代数矢量的加法和数乘对应多项式的加法和多项式和常数相乘.
\end{exam}

\subsection{点乘}
由于代数矢量没有一般的几何意义, 我们直接定义同一空间中两代数矢量的点乘为
\begin{equation}\label{NumVec_eq5}
\vec u \vdot \vec v = \sum_{i = 1}^N u_i^* v_i
\end{equation}
其中 $u_i^*$ 是 $u_i$ 的复共轭(\autoref{CplxNo}\upref{CplxNo_eq6}), 所以对于实数矢量, 点乘化简为
\begin{equation}
\vec u \vdot \vec v = \sum_{i = 1}^N u_i v_i
\end{equation}
现在我们可以定义任意代数矢量 $\vec v$ 的\bb{模长} 为(注意模长都是非负的实数)
\begin{equation}
\abs{\vec v} = \sqrt{\vec v \vdot \vec v} = \qty(\sum_i \abs{v_i}^2)^{1/2}
\end{equation}
且定义若两个代数矢量点乘为 0, 则它们互相\bb{正交}.

现在我们可以类比几何矢量定义代数矢量的\bb{正交归一基底}, 事实上在 “正交归一基底\upref{OrNrB}” 中, 我们并没有要求所有的矢量都是几何矢量, 而只要存在“ 点乘”, “ 模长” 和“ 正交” 的概念.

% 未完成, 在矩阵中提一下, \vec u \vdot \vec v = \vec u\Her \vec v, 即矩阵乘法.

%\begin{equation}
%\vec u \vdot \vec v = \qty(\sum_i u_i\vec x_i)\qty(\sum_j v_j\vec x_j) = \sum_{ij} u_i v_j \delta_{ij} = \sum_k u_k v_k
%\end{equation}

若我们把代数矢量看做另一种矢量的坐标, 那么只要这种矢量的点乘运算定义合理, 我们就可以用代数矢量(坐标)的点乘来计算.

\begin{exam}{正弦级数矢量空间}
区间 $[0, \pi]$ 内所有小于 $N$ 阶的正弦级数(这里规定系数可以是复数)
\begin{equation}
f(x) = \sum_{n=0}^{N-1} C_n\sin(nx)
\end{equation}
可以看做一个 $N$ 维矢量空间, 加法和数乘的定义就是函数相加和函数乘以常数. 除此之外, 我们还可以定义两个任意矢量 $f(x)$ 和 $g(x)$ 的点乘为
\begin{equation}
\int_0^\pi f^*(x)g(x) \dd{x}
\end{equation}
不难证明 $\sin(x), \sin(2x), \dots, \sin[(N-1)x]$ 就是一组正交归一基底(证明见“三角傅里叶级数\upref{FSTri}”).

令 $f(x)$ 的坐标为 $(a_1, \dots, a_{N-1})\Tr$, $g(x)$ 的坐标为 $(b_1, \dots, b_{N-1})\Tr$, 则空间中任意两个矢量的点乘可以表示为两个代数矢量的点乘
\begin{equation}\ali{
\int_0^\pi f^*(x)g(x) \dd{x}
&= \int_0^\pi \dd{x} \sum_{m=0}^{N-1} a_i^*\sin(mx) \sum_{n=0}^{N-1} b_i \sin(nx)\\
&= \sum_{m=0}^{N-1} \sum_{n=0}^{N-1} a_m^* b_n \delta_{mn} = \sum_k a_k^* b_k
}\end{equation}
\end{exam}

我们知道两个几何矢量的点乘不需要任何坐标系(基底)的概念, 所以我们可以使用它们在任意单位正交基底中的坐标计算点乘, 结果相同. 我们下面来证明代数矢量也满足类似的性质.

首先由\autoref{NumVec_eq5} 易证点乘的分配律, 即
\begin{equation}
(\vec u_1 + \ldots + \vec u_m)(\vec v_1 + \ldots + \vec v_n) = \sum_{i=1}^m \sum_{j=1}^n \vec u_i \vdot \vec v_j
\end{equation}

令两个代数矢量分别为 $\vec u = (u_1, u_2, \dots, u_N)\Tr$ 和 $\vec v = (v_1, v_2, \dots, v_N)\Tr$, 我们可以把它们看做是某矢量空间中两个抽象矢量在某组正交归一基底上的坐标. 这组基底对应的代数矢量分别为 $(1, 0, 0, \dots)\Tr$, $(0, 1, 0, \dots)\Tr$ 等. 若另有一组正交归一基底 $\uvec x_1, \uvec x_2, \dots, \uvec x_N$, 关于原基底的坐标分别为 $(x_{11}, x_{12}, \dots, x_{1N})\Tr$, $(x_{21}, x_{22}, \dots, x_{2N})\Tr$ 等, 则 $\vec u, \vec v$ 可以在新基底上展开为 $(u'_1, u'_2, \dots, u'_N)\Tr$ 和 $(v'_1, v'_2, \dots, v'_N)\Tr$. 所以有
\begin{equation}
\sum_k u_i^* v_i = \vec u \vdot \vec v = \qty(\sum_{i=1}^N u'_i \uvec x_i)\vdot\qty(\sum_{j=1}^N v'_j \uvec x_j) = \sum_{ij} {u'_i}^* v_j \delta_{ij} = \sum_{k=1}^N {u'_k}^* v_k
\end{equation}



% 未完成: 证明, 但要先证明酋矩阵的性质, 即如果所有的行正交归一, 那么所有的列也正交归一. 由 AA^T = 1, A^T = A^{-1}, 所以 A^T A = 1. 所以又涉及到逆矩阵的性质.
%所以, 代数矢量的点乘也可以看做是两个不取决于基底的抽象矢量见的一种性质.




