\subsection{电动力学研究什么}

高中所谓的“电磁学”就是简单的电动力学。

经典力学研究一堆粒子(质点)受一堆力后的运动情况,这里的粒子可以是任何有质量粒子(电子,质子,中子,甚至一块很小的石头),力也可以是任何力。电动力学研究经典力学中的一种特殊情况(先不讨论相对论!):即一堆带电荷的质点(既有质量又有电荷)受一堆电磁力后的变化。所以唯一需要做的是,弄清如何计算这些电磁力,弄清了如何计算这些力以后,再用经典力学就可以得出粒子的运动。

\subsection{电磁力和电磁场}
经典力学中万有引力是超距的,就是说只要两个质点存在,不管他们相隔多远都会立即有作用力(与质量和距离有关),如果它们之间的距离发生改变,那么这个作用力也会立即改变。这是不合理的(现在我们知道任何信息的传递都不可能超过光速,如果一个粒子动一下,另一个粒子马上就能感到力的变化,那就可以超过光速传递信息)。电动力学不同,我们需要先计算电场和磁场(场就像波一样,传播需要时间),再根据粒子所在位置的场计算粒子的受力(其他地方的场如何与粒子受力无关)。例如库仑定理的形式虽然和万有引力一样,但是在电动力学中我们先计算一个粒子在另一个粒子处产生的电场,再计算处于该电场中的粒子所受电场力(库仑力)。当两个粒子都静止时,这么做似乎和直接由距离计算力有没有区别,但如果某时刻其中一个粒子抖动了一下,电场(先不提磁场)就会像扰动的水波一样将这个扰动沿各个方向以一定的速度传播,直到传播到另一个粒子所在的地方,另一个粒子才能从电场中感觉到受力的变化。

磁场虽然产生的方式和对粒子的作用与电场不同,但也会像波一样传播。事实上,电场和磁场并不是独立传播的,上面说的电场扰动的传播时必须同时借助磁场。

\subsection{电荷}
电荷在电动力学中扮演了两个角色。一是电磁场是由电荷产生的。二是只有带电荷的粒子在电磁场中才会受力,其他粒子不会。

\subsection{麦克斯韦方程组}
麦克斯韦方程组是一组写描述场变化规律的数学公式。可以想象成一个懂电动力学的计算机,只要告诉它所有的带电粒子在哪里,以及如何运动(位置关于时间的函数),它就能计算出空间中任何一点的电磁场。和水波不同,电场和磁场在每个位置都既有方向也有大小(有方向和大小的量叫做矢量,这种场叫做矢量场)。

麦克斯韦方程组是电动力学的公设之一。

\subsection{电磁力}
任何一个带电粒子某时刻受到的电磁力(也叫广义洛伦兹力),都可以由它所在的位置处的电磁场(两个矢量)以及它当时运动速度(也是矢量)通过公式计算出来。这是电动力学的另一个公设。

可见,在经典力学的基础上加上麦克斯韦方程和广义洛伦兹力的公式后,如果我们知道一堆带电粒子在某个时刻的运动状态(位置和速度),我们就可以知道接下来每个粒子如何运动。
