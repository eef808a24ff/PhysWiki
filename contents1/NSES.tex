% 数值解一维薛定谔方程(试射法)

用试射法解以下薛定谔方程,其中 $V(x)$ 已知
\begin{equation}
-\frac{1}{2m}\psi(x) + V(x)\psi(x) = E \psi(x)
\end{equation}
先讨论偶函数势能的情况.当 $V(x)$ 为偶函数,定态波函数为奇函数或偶函数.前者可用初始条件 $\psi(0)=0$, $\psi'(0)=1$,后者可用初始条件 $\psi(0)=1$, $\psi'(0)=0$.这样解出来的波函数一般不满足归一化条件,若需要可另行归一化.

现在可用定步长的欧拉法或者龙格库塔四阶法来解 $x\in [0, x_{max}]$ 的薛定谔方程.显然程序中不能选取 $x\in [0,+\infty]$,但是要保证 $x_{max}$ 足够大,使解出薛定谔方程后有 $\psi(x_{max})\approx 0$, $\psi'(x_{max})\approx 0$.

但如何求得本征值 $E$ 呢?一种简单的方法是试射法. 顾名思义, 取不同离散的 $E$,用一定的条件判断对这些 $E$ 波函数在 $x_{max}$ 处是否满足边界条件 $\psi(x_{max}) \approx 0$\footnote{如果 $x_{max}$ 足够大,只需满足这一条件即可自动满足 $\psi'(x_{max})\approx 0$}.

一般来说,若 $E$ 略大于某本征值 $E_n$ 时,会有 $\psi(x_{max})>0$,略小时会有 $\psi(x_{max})>0$.所以画出 $\psi(x_{max})$ 关于 $E$ 的图像,用目测法找到零点即可.若需要更精确的本征值,可选取一个更小的区间,再次画图.

若觉得这样做比较麻烦,可用多区间的二分法自动寻找以上图像中的零点,方法如下.先确定本征值所在的范围, 把该范围先等分成 $N$ 等份,在每一份的两端判断 $\psi(x_{max})$ 的符号,如果有变号,则说明该份中至少有一个零点.继而可用二分法求解零点的精确值.在这种方法中,必须保证每个小份中至多有一个零点,否则有可能漏解.检查是否漏解可以用波函数节点的数量来判断.如果节点数 $N_{node}$ 是递增的,则没有漏解.