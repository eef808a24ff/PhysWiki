% 带电粒子的薛定谔方程
% 参考 Brandsen 的原子物理

\pentry{点电荷的拉格朗日和哈密顿量\upref{EMLagP}}

经典点电荷在电磁场中的哈密顿量为
\begin{equation}
H = \frac{1}{2m}(\vec p - q\vec A)^2 + q\phi
\end{equation}
其中 $\vec A$ 和 $\phi$ 都是位置和时间的函数. 注意这里的 $\vec p$ 是广义动量
\begin{equation}
\vec p = m\vec v + q\vec A
\end{equation}
算符仍然是 ${\vec p} = -\I\hbar\grad$. 只有 $\vec A = \vec 0$ 时才有 $\vec p = m\vec v$.

所以哈密顿算符是
\begin{equation}
H = \frac{\vec p^2}{2m} - \frac{q}{2m}(\vec A\vdot \vec p + \vec p \vdot \vec A) + \frac{q^2}{2m} \vec A^2 + q\phi
\end{equation}

这个方程在以下度规变换下形式不变
\begin{equation}
\vec A = \vec A' + \grad \chi
\end{equation}
\begin{equation}
\phi = \phi' - \pdv{\chi}{t}
\end{equation}
\begin{equation}
\Psi = \Psi' \exp(\I q\chi/\hbar)
\end{equation}
其中 $\chi(\vec r, t)$ 是一个任意可导函数. 将以上三式代入薛定谔方程, 只需要把不带撇的变量替换为带撇的变量.

\subsection{长度度规}
% 未完成

\subsection{速度度规}
% 未完成
