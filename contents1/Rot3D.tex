%空间旋转矩阵

% 先解释 平面旋转矩阵中的矩阵元是 x*x' 点乘得出的, 任何直角坐标的变换都可以用点乘完成.
% 顺便可以解释为什么如果所有列矢量正交归一, 所有行矢量也会正交归一.

\pentry{平面旋转矩阵\upref{Rot2D}}

类比平面旋转矩阵\upref{Rot2D},空间旋转矩阵是三维坐标的旋转变换,所以应该是 $3 \cross 3$ 的方阵.不同的是平面旋转变换只有一个自由度 $\theta $, 而空间旋转变换除了转过的角度还需要考虑转轴的方向. 如果直接从转轴和转动角度来定义该矩阵,矩阵比较复杂, 这里从略.

若已经知道空间直角坐标系中三个单位正交矢量
\begin{equation}
\uvec x=(1,0,0)\Tr \quad \uvec y=(0,1,0)\Tr \quad \uvec z=(0,0,1)\Tr
\end{equation}
 经过三维旋转矩阵变换以后变为 
\begin{equation}
({a_{11}},{a_{21}},{a_{31}})\Tr \quad ({a_{12}},{a_{22}},{a_{32}})\Tr \quad ({a_{13}},{a_{23}},{a_{33}})\Tr
\end{equation}
类比平面旋转矩阵\upref{Rot2D}

\begin{equation}
\mat R_3 = \begin{pmatrix}
{a_{11}}&{a_{12}}&{a_{13}}\\
{a_{21}}&{a_{22}}&{a_{23}}\\
{a_{31}}&{a_{32}}&{a_{33}}
\end{pmatrix}\end{equation}

