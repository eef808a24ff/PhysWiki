% Crank-Nicolson 算法(一维)

\subsection{薛定谔方程}
\begin{equation}
\ali{
\psi_i^{n+1} - \psi_i^n &= \frac{\I\Delta t}{4\Delta x^2} (\psi_{i-1}^n - 2\psi_i^n + \psi_{i+1}^n + \psi_{i-1}^{n+1} - 2\psi_i^{n+1} + \psi_{i+1}^{n+1})\\
&\qquad\qquad - \frac{\I\Delta t}{2}(V_i^n\psi_i^n + V_i^{n+1}\psi_i^{n+1})
}\end{equation}
令 $\alpha = \I\Delta t/(4\Delta x^2), \beta = \I\Delta t/2$, 整理可得
\begin{equation}\label{CraNic_eq2}
-\alpha\psi_{i-1}^{n+1} + (1+2\alpha + \beta V_i^{n+1})\psi_i^{n+1} - \alpha \psi_{i+1}^{n+1} = \alpha\psi_{i-1}^n + (1 - 2\alpha - \beta V_i^n)\psi_i^n + \alpha \psi_{i+1}^n
\end{equation}
其中 $\psi_i^n = \psi(x_i, t_n)$, $V_i^n = V(x_i, t_n)$。

我们把一个区间划分成 $N_x - 1$ 段等长的区间, 并令 $N_x$ 个格点为 $x_1\dots x_{N_x}$。 最简单的边界条件是取 $\psi(x_1) = \psi(x_{N_x}) = 0$。 这样\autoref{CraNic_eq2} 中的 $i$ 可以取 $i = 2\dots N_x - 1$, 得到 $N_x - 2$ 条式子, 其中只有 $\psi_2^{n+1}\dots \psi_{N_x-1}^{n+1}$ 这 $N_x - 2$ 个未知量, 每条式子最多包含连续 3 个未知量。 将线性方程用矩阵表示, 就可以得到一个三对角矩阵(第一行和最后一行只有两个系数)。
