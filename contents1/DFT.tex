% 离散傅里叶变换

\bb{离散傅里叶变换(Discrete Fourier Transform)(DFT)}是一个复数域的线性变换\upref{LTrans}. 对两组有序数列 $f_0, f_1, \dots, f_{N-1}$ 和 $g_0,g_2,\dots, g_{N-1}$,正变换和逆变换分别为\footnote{工程上的定义常常是正变换没有 $1/\sqrt{N}$ 因子,逆变换的 $1/\sqrt{N}$ 因子变为 $1/N$. 这样的好处是节省运算量.本书中使用的定义好处是变换为幺正变换,有保持归一化的特点.}
\begin{align}
g_p &= \frac{1}{\sqrt{N}}\sum_{q=0}^{N-1} \exp(-\frac{2\pi\I}{N} p q ) f_q \\
f_q &= \frac{1}{\sqrt{N}}\sum_{p=0}^{N-1} \exp(\frac{2\pi\I}{N} p q) g_p
\end{align}

一个更常见的名词是\bb{快速傅里叶变换(FFT)}, 其定义与离散傅里叶变换一样,只是优化了算法使程序运行更快\footnote{参考 Numerical Recipes 3ed}.

\subsection{离散傅里叶变换与傅里叶变换}
\pentry{傅里叶变换(指数函数)\upref{FTExp}}
在详细分析离散不理也变换的性质之前, 我们先看看它与解析的傅里叶变换如何对应。 函数 $f(x)$ 和 $g(k)$ 间的傅里叶变换为
\begin{align}
g(k) &= \frac{1}{\sqrt{2\pi}} \int_{-\infty}^{+\infty} f(x)\E^{-\I kx} \dd{x}\label{DFT_eq3}\\
f(x) &= \frac{1}{\sqrt{2\pi}} \int_{-\infty}^{+\infty} g(k)\E^{\I kx} \dd{k}\label{DFT_eq4}
\end{align}
如果 $f(x)$ 和 $g(k)$ 分别只在区间 $[-L_x/2, L_x, 2]$ 和 $[-L_k/2, L_k/2]$ 内不为零, 积分就可以只在这两个区间内进行。 我们再给两个区间划出 $N$ 个等间距的格点 $\dots, x_{-1}, x_0, x_1,\dots$ 和 $\dots, k_{-1}, k_0, k_1,\dots$ ($x_0 = k_0 = 0$), 并规定相邻格点的间距为
\begin{equation}
\Delta x = L_x/N \qquad \Delta k = L_k/N
\end{equation}
注意 $x$ 和 $k$ 的首尾格点分别相距 $L_x(N-1)/N$ 和 $L_k(N-1)/N$。 若 $N$ 是奇数, 我们令中间的格点为 $x_0$ 和 $k_0$, 如果 $N$ 是偶数, 我们令中间靠右的格点为 $x_0$ 和 $k_0$。

现在我们用求和近似\autoref{DFT_eq3} 的积分得\footnote{事实上, 学习了下文的采样定理会发现这里用求和代替积分是完全精确的。}
\begin{align}
g(k_p) &= \frac{1}{\sqrt{2\pi}} \sum_q f(x_q) \E^{-\I k_p x_q} \Delta x\label{DFT_eq6}\\
f(x_q) &= \frac{1}{\sqrt{2\pi}} \sum_p g(k_p) \E^{\I k_p x_q} \Delta k\label{DFT_eq7}
\end{align}
式中 $k_p x_q = (\Delta x \Delta k)pq$, 对比 DFT 中的指数项, 得
\begin{equation}\label{DFT_eq8}
\Delta x\Delta k = \frac{2\pi}{N}
\end{equation}
但我们会发现这里的 $p, q$ 可以是负整数, 而 DFT 中的 $p, q$ 都是非负整数。 但稍加计算就会发现当 $p$ 或 $q$ 是负值时, 把它们加上 $N$, 指数项并不改变, 例如 $k_{p+N} x_q = k_p x_q + 2\pi$, 并不影响指数项。 所以, 我们只要将所有小于零的格点编号加上 $N$ 并重新排列即可。 例如 $N = 4$ 时, $x$ 格点为 $x_{-2}, x_{-1}, x_0, x_1$, 令 $x_2 = x_{-2}, x_3 = x_{-1}$, 这四个格点的名字就变为 $x_0, x_1, x_2, x_3$。 现在再来对比 DFT 和 \autoref{DFT_eq6} \autoref{DFT_eq7}, 就只是相差两个常数因子而已了。

\autoref{DFT_eq8} 是 DFT 一个重要的性质。 稍加变换得
\begin{equation}
N\Delta x \Delta k = L_x \Delta k = L_k \Delta x = \frac{L_x L_k}{N} = 2\pi
\end{equation}
所以离散傅里叶变换只有两个自由度, 只要决定 $N, L_x, \Delta x, L_k, \Delta k$ 中的任意两个, 就可以完全决定变换公式。 注意 $L_x$ 与 $\Delta k$ 成反比, $L_k$ 与 $\Delta x$ 成反比。

总结起来, 要用 DFT 数值求解一个函数 $f(x)$ 的傅里叶变换, 就先用上述方法生成的格点将该函数等间距采样, 然后左半和右半调换得到 $f_i$, 代入 DFT 公式得到 $g_i$ 再左半和右半调换得到 $g(k)$ 的离散点。 反傅里叶变换同理。

\subsection{变换矩阵}
\pentry{酋矩阵}% 链接未完成
DFT 显然是一个线性变换, 我们来看变换矩阵的性质。 把变换和逆变换的系数矩阵用% 幺正矩阵链接未完成
 $\mat F$ 和 $\mat F^{-1}$ 来表示, 令列矢量 $\vec f = (f_0,f_1,\dots,f_N)\Tr$, $\vec g = (g_1,g_2,\dots,g_N)\Tr$, 变换和逆变换分别记为
\begin{equation}
\vec g = \mat F \vec f \qquad
\vec f = \mat F^{-1} \vec g
\end{equation}
其中
\begin{equation}
F_{pq} = \frac{1}{\sqrt{N}} \exp(-\frac{2\pi\I}{N} p q) \qquad
\mat F^{-1} = \mat F\Her
\end{equation}
可以证明, $\mat F$ 是一个幺正矩阵, 所以有
% 未完成:正交矩阵的逆矩阵
% 未完成:凡是证明以前,真的至少应该说明一下这玩意有什么用,从哪来吧!任何词条都是啊!

\subsection{证明 $\mat F$ 的正交性}
根据正交矩阵的定义,我们需要证明
\begin{equation}
\sum_{p=0}^{N-1} F^*_{pq_1} F_{pq_2} = 0 \quad (q_1 \ne q_2)
\end{equation}
而
\begin{equation}\label{DFT_eq1}
\sum_{p=0}^{N-1} F^*_{pq_1}F_{pq_2}
= \sum_{p=0}^{N-1} \exp[\frac{2\pi\I}{N} (q_2-q_1) p]
\end{equation}
注意到求和的每一项在复平面上都对应模长为 1, 幅角为 $(q_2-q_1)p/N$ 个圆周的矢量,% 未完成:这个几何意义要在复数的加法以及指数函数(复数)中介绍!
而 $N$ 条矢量恰好向不同方向均匀分布,所以相加为 $0$.证毕.

\subsection{逆矩阵}
先来看矩阵各列的模长平方,即\autoref{DFT_eq1} 中 $q_1 = q_2$ 的情况,易得任何一列的模长平方都为 $N$.所以矩阵 $\mat F$ 是一个幺正矩阵 $\mat U$ 乘以 $\sqrt{N}$.考虑到幺正矩阵的逆矩阵是其厄米共轭, %链接未完成
有
\begin{equation}
\mat F\Her \mat F = \sqrt{N}^2 \mat U\Her \mat U = N \mat U^{-1} \mat U = N
\end{equation}
两边除以 $N$,可得 $\mat F^{-1} = \mat F\Her /N$. 证毕.
% 未完成:在逆矩阵的介绍中应该说明用常数表示常数乘以单位矩阵

有时为了对称起见,把矩阵元定义为
\begin{equation}
F_{pq} = \frac{1}{\sqrt{N}}\exp(-\frac{2\pi\I}{N} p q)
\end{equation}
这样,$\mat F$ 的每一列模长为1,使 $\mat F$ 本身就是幺正矩阵,其逆矩阵等于厄米共轭.

\subsection{采样定理(Sampling Theorem)}
以上的离散傅里叶变换看似只是一种近似, 且有种种限制, 例如我们想既规定 $L_x$, 又减小 $\Delta k$。 使我们不禁想定义更广义的离散傅里叶变换为
\begin{equation}\ali{
g(k_j) = \frac{1}{\sqrt{2\pi}}\sum_i f(x_i) \E^{-\I k_j x_i}\Delta x\\
f(x_i) = \frac{1}{\sqrt{2\pi}}\sum_j g(k_j) \E^{\I k_j x_i}\Delta k
}\end{equation}
(其中相邻的 $x_i$ 间距为 $\Delta x$, 相邻的 $k_j$ 间距为 $\Delta k$)。 但事实上, 这样的定义并不比离散傅里叶变换好, 不仅不能用 FFT 提高运算速度, 而且可能更不精确。

接下来我们要介绍非常强大的\bb{采样定理}, 并得出一个惊人的结论, 即只要 $f(x)$ 和 $g(k)$ 不超出 $L_x$ 和 $L_k$ 的范围, DFT 就是精确的傅里叶变换。

采样定理说的是, 如果 $g(k)$ 不超出 $[-L_k/2, L_k/2]$, 那么我们只需要用 $\Delta x = 2\pi/L_k$ 来对  $f(x)$  采样就可以用以下插值公式精确还原出 $f(x)$\footnote{其实我并不确定是否需要 $x_0 = 0$ 或者 $x_i$ 的位置有什么其他要求, 现在姑且认为 $x_0 = 0$ 好了。}
\begin{equation}
f(x) = \sum_{n = -\infty}^{\infty} f(x_n)\sinc[\pi(x - x_n)/\Delta x]
\end{equation}
其中 $\sinc x = \sin x/x$ (且定义 $\sinc 0 = 1$)。

现在我们将插值公式代入傅里叶变换得
\begin{equation}
g(k) =  \frac{1}{\sqrt{2\pi}} \sum_{n = -\infty}^{\infty} f(x_n) \int_{-\infty}^{+\infty}\sinc[\pi(x - x_n)/\Delta x] \E^{-\I k x} \dd{x}
\end{equation}
结果是
\begin{equation}
g(k) = \frac{1}{\sqrt{2\pi}} \sum_{n = -\infty}^{\infty} f(x_n) \E^{-\I kx_n} \Delta x
\end{equation}
注意 $g(k)$ 是完全精确的。 另外, 既然傅里叶变化和反傅里叶变换是对称的, 我们同样可得频率步长只需取 $\Delta k = 2\pi/ L_x$。 而 DFT 恰好符合这两个条件。 最后, $g(k)$ 的插值公式同样为

\begin{equation}
g(k) = \sum_{n = -\infty}^{\infty} g(k_n)\sinc[\pi(k - k_n)/\Delta k]
\end{equation}

其实单纯的无损 DFT 效率还不是最高的,例如有时候 $k_{max}-k_{min}$ 很窄, 但 $L_k$ 却很大, 这时用 $\Delta x = 2\pi/L_k$ 的效率就很低。 同理, 如果 $x_{max} - x_{min} < L_x$ 时效率也会变低。 所以与其直接对 $f(x)$ 和 $g(k)$ 使用 DFT 不如先定义
\begin{equation}
x_0 = (x_{min} + x_{max})/2 \qquad k_0 = (k_{min} + k_{max})/2
\end{equation}
然后进行以下傅里叶变换
\begin{equation}
f(x+x_0)\E^{-\I k_0 x} \Leftrightarrow g(k+k_0)\E^{\I k x_0}
\end{equation}
完了再除以指数因子及平移。

