% 质心 质心系

\pentry{体积分\upref{IntN}, 质点系\upref{PSys}}

\subsection{质心的定义}
% 未完成: 光用公式不够形象, 质心, 就可以理解为质量的中心, 是位矢关于质量的加权平均值. 举个例子说明, 例如两个质点的质心.

对质点系,令第 $i$ 个质点质量为 $m_i$,位置为 $\vec r_i$,总质量为 $M = \sum\limits_i m_i$, 则该质点系的\bb{质心}定义为
\begin{equation}\label{CM_eq1}
\vec r_c = \frac{1}{M}\sum_i m_i \vec r_i
\end{equation}
对连续质量分布,令密度关于位置的函数为 $\rho (\vec r)$,总质量为密度的体积分 %(链接未完成)
\begin{equation}
M = \int \rho (\vec r) \dd{V}
\end{equation}
质心定义为
\begin{equation}
\vec r_c = \frac{1}{M}\int \vec r\rho (\vec r) \dd{V}
\end{equation}
以下的讨论都对质点系进行,连续质量分布可看做由许多体积微元组成,也可看做质点系.

\subsection{质心的唯一性}
既然质心的定义取决于参考系(因为 $\vec r_i$ 取决于参考系),那么不同参考系中计算出的质心是否是空间中的同一点呢?我们只需要证明,在 $A$ 坐标系中得到的质心 $\vec r_{Ac}$ 与 $B$ 坐标系中得到的质心 $\vec r_{Bc}$ 满足关系
\begin{equation}\label{CM_eq4}
\vec r_{Ac} = \vec r_{AB} + \vec r_{Bc}
\end{equation}
其中 $\vec r_{AB}$ 是 $A$ 系原点指向 $B$ 系原点的矢量. 首先根据定义
\begin{equation}
\vec r_{Ac} = \frac{1}{M}\sum_i m_i \vec r_{Ai}  \qquad \vec r_{Bc} = \frac{1}{M}\sum_i  m_i \vec r_{Bi} 
\end{equation}
由位矢的坐标系变换%(未完成: 确保有介绍)
,$\vec r_{Ai} = \vec r_{AB} + \vec r_{Bi}$, 所以
\begin{equation}
\vec r_{Ac} = \frac{1}{M}\sum_i m_i(\vec r_{AB} + \vec r_{Bi})  = \vec r_{AB} + \frac{1}{M} \sum_i m_i \vec r_{Bi}  = \vec r_{AB} + \vec r_{Bc}
\end{equation}
 
\subsection{质心系}
定义质点系的\bb{质心系}为原点固定在质心上且没有转动的参考系(平动参考系).%链接未完成: 平动是相对的, 转动是绝对的.
根据质心的唯一性(\autoref{CM_eq4}),在质心系中计算质心(\autoref{CM_eq1}) 仍然落在原点,即
\begin{equation}\label{CM_eq7}
\sum_i m_i \vec r_{ci} = \vec 0
\end{equation}
其中 $\vec r_{ci}$ 是质心系中质点 $i$ 的位矢.

注意质心系并不一定是惯性系,只有当合外力为零质心做匀速直线运动时,质心系才是惯性系.在非惯性系中,每个质点受惯性力.

\subsection{质心系中总动量}
把\autoref{CM_eq7} 两边对时间求导,得
\begin{equation}\label{CM_eq8}
\sum_i m_i \vec v_{ci} = \vec 0
\end{equation}
注意到等式左边恰好为质心系中质点系的总动量,所以我们得到质心系的一个重要特点,\bb{质心系中总动量为零}.
