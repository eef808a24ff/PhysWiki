% 库仑函数

在球坐标系中解库仑势场中的薛定谔方程, %链接未完成
会得到径向方程
\begin{equation}
\dv[2]{u}{\rho} + \qty[1 - \frac{2\eta}{\rho} - \frac{l(l+1)}{\rho^2}]u = 0
\end{equation}
其解为\bb{第一类库仑函数} $F_l(\eta; \rho)$ 和\bb{第一类库仑函数} $G_l(\eta; \rho)$. 后者在原点处为无穷大, 先不讨论. 第一类库仑函数的解析式为
\begin{equation}
F_l(\eta; \rho) = A_l(\eta) \rho^{l+1} \E^{-\I\rho} {_1F_1}(l+1-\I\eta; 2l+2; 2\I\rho)
\end{equation}
其中
\begin{equation}
A_l(\eta) = \frac{2^l \E^{-\pi\eta/2} \abs{\Gamma(l+1+\I\eta)}}{(2l+1)!}
\end{equation}
$_1 F_1(a;b;z)$ 是第一类合流超几何函数\upref{HypGeo}. 库仑函数也可以用\bb{惠特克 $M$ 函数(Whittaker M)}来表示得更紧凑
\begin{equation}
F_l(\eta; \rho) = A_l(\eta) \qty(-\frac{\I}{2})^{l+1} M\qty(\I\eta; l+\frac12; 2\I\rho)
\end{equation}

$F_l(\eta; \rho)$ 是一个实函数, 类似第一类球贝赛尔函数乘 $r$\footnote{当 $\eta = 0$ 时二者相等.}, 有
\begin{equation}
F_l(\eta; 0) = 0 \qquad \eval{\dv{F_l(\eta; \rho)}{\rho}}_0 = 
\leftgroup{&A_0(\eta) &\qquad &(l = 0)\\ & 0 &\qquad &(l > 0)}
\end{equation}
且渐进形式为
\begin{equation}
\lim_{\rho\to \infty} F_l(\eta; \rho) = \sin[\rho - \pi l/2 - \eta\ln(2\rho) + \phi_l(\eta)]
\end{equation}
其中\bb{库仑相移(Coulomb phase shift)}
\begin{equation}
\phi_l(\eta) = \arg[\Gamma(l+1+\I\eta)]
\end{equation}

我们以后将第一类库仑函数记为更直观的形式 $F_l(Z, k, r)$, 或简写为 $F_l(k, r)$. 由渐进形式可得径向归一化积分\footnote{积分时可忽略 $\sin$ 中的额外相位, 但我不会证.}与球贝赛尔函数的一样
\begin{equation}
\int_0^\infty F_l(k', r)F_l(k, r) \dd{r} = \int_0^\infty \sin(k'r)\sin(kr) \dd{r} = \frac{\pi}{2}
\end{equation}
