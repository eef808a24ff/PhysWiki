% 平面波的球谐展开

\pentry{球谐函数\upref{SphHar}}

复数形式的平面波可以展开为\footnote{易证这里的复共轭可以加在任意一个球谐函数上.}
\begin{equation}
\E^{\I \vec k \vdot \vec r} = 4\pi \sum_{l=0}^{\infty} \sum_{m=-l}^l \I^l j_l(kr) Y_{lm}(\uvec k) Y_{lm}^*(\uvec r)
\end{equation}
\begin{equation}
\E^{-\I \vec k \vdot \vec r} = 4\pi \sum_{l=0}^{\infty} \sum_{m=-l}^l \I^{-l} j_l(kr) Y_{lm}(\uvec k) Y_{lm}^*(\uvec r)
\end{equation}

\subsection{球谐展开函数的傅里叶变换}

任意函数可以表示为
\begin{equation}
f(\vec r) = \sum_{l,m} R_{lm}(r) Y_l^m(\uvec r)
\end{equation}
则傅里叶变换为
\begin{equation}\ali{
g(\vec k) &= \frac{1}{(2\pi)^{3/2}} \int f(\vec r) \E^{-\I \vec k \vec r} \dd[3]{r}\\
&= \sqrt{\frac{2}{\pi}} \sum_{l,m} \I^{-l} Y_l^m(\uvec k) \int_0^{+\infty} R_{lm}(r) j_l(kr) r^2 \dd{r}
}\end{equation}

\subsection{氢原子基态的动量谱}
氢原子基态的波函数(原子单位)为
\begin{equation}
\psi(\vec r) = \frac{1}{\sqrt\pi} \E^{-r}
\end{equation}
显然只有 $l = 0, m = 0$ 球谐项. 而 $Y_0^0 = 1/\sqrt{4\pi}$, 所以径向波函数为
\begin{equation}
R_{00}(r) = 2 \E^{-r}
\end{equation}
所以傅里叶变换为(注意 $j_0(x) = \sin x/x$)
\begin{equation}
g(\vec k) = \frac{\sqrt{2}}{k\pi} \int_0^\infty \E^{-r} \sin(kr) r \dd{r} = \frac{2\sqrt{2}}{\pi(k^2+1)^2}
\end{equation}

当然我们也可以将沿 $z$ 轴正方向的三维平面波用的复共轭球坐标表示, 再在球坐标中与波函数积分, 结果相同.
