%统计力学公式大全

%咳咳, 由于太多公式记不住, 有必要弄一篇公式大全
\subsection{微分关系}
\begin{equation}
H = E + PV
\end{equation}
\begin{equation}
G = E + PV - ST
\end{equation}
\begin{equation}
\mu\dd{N} + N\dd{\mu} = \dd{G} = V\dd{P} - S\dd{T} + \mu\dd{N}
\end{equation}
\subsection{微正则系综}
\begin{equation}
\dd{S} = \frac1T \dd{E} + \frac{P}{T} \dd{V} - \frac{\mu }{T} \dd{N}
\end{equation}
\subsection{正则系宗}
\begin{equation}
- kT\ln Q = F = E - ST
\end{equation}
\begin{equation}
\dd{F} = -S\dd{T} - P\dd{V} + \mu\dd{N}
\end{equation}
\begin{equation}
E = -\pdv{\beta} \ln Q
\end{equation}
\subsection{巨正则系综}
\begin{equation}
- PV = \Phi  = E - ST - \mu N
\end{equation}
\begin{equation}
\Phi  =  - kT\ln \Xi 
\end{equation}
\begin{equation}
\dd{\Phi} =  - P\dd{V} - S\dd{T} - N\dd{\mu}
\end{equation}
\begin{equation}
\ev{n_i} = \pdv{\Phi}{\varepsilon_i}
\end{equation}
\subsection{理想气体}
\begin{equation}
V_n = \frac{\pi^{n/2}}{R^n} \Gamma(1 + n/2)
\qquad
\text{( $N$ 维球体)}
\end{equation}
\begin{equation}
\Omega_0 = \frac{V^N}{N! h^3} \frac{(2\pi mE)^{3N/2}}{(3N/2)!}
\qquad \text{( $N$ 粒子能级密度)}
\end{equation}
\begin{equation}
a(\varepsilon) = \frac{2\pi V{(2m)^{3/2}}}{h^3} \varepsilon^{1/2}
\qquad \text{(单粒子能及密度)}
\end{equation}
\begin{equation}
S = Nk\qty(\ln \frac{V}{N\lambda^3} + \frac52)
\qquad
\text{(熵)}
\end{equation}
\begin{equation}
N = zQ_1 \Rightarrow \mu  = kT\ln \frac{N\lambda^3}{V}
\qquad
\text{(化学势)}
\end{equation}
\begin{equation}
\Xi  = \ln N
\qquad
\text{(巨势)}
\end{equation}
\subsection{量子气体}
\begin{equation}
N = Q_1 g_{3/2} (z) = \frac{V}{\lambda^3} g_{3/2} (z)
\qquad\text{($BE$)}
\end{equation}
\begin{equation}
N = Q_1 f_{3/2} (z)
\qquad\text{($FD$)}
\end{equation}
\begin{equation}
\frac{PV}{kT} = Q_1 g_{5/2} (z) = \frac{V}{\lambda^3} g_{5/2} (z)
\qquad\text{($BE$)}
\end{equation}
\begin{equation}
\frac{PV}{kT} = Q_1 f_{5/2} (z)
\qquad\text{($FD$)}
\end{equation}
\begin{equation}
PV = NkT\frac{g_{5/2}(z)}{g_{3/2}(z)}
\qquad\text{($BE$)}
\end{equation}
\begin{equation}
PV = NkT\frac{f_{5/2}(z)}{f_{3/2}(z)}
\qquad\text{($FD$)}
\end{equation}
\begin{equation}
E = \frac32 PV
\qquad\text{($BE$ 和 $FD$)}
\end{equation}
理论上可以通过三式中的任意两式消去 $z$,  但是不能写成解析形式.
\begin{equation}\ali{
g_n(z) = z + \frac{z^2}{2^n} + \frac{z^3}{3^n}\dots\\
f_n(z) = z - \frac{z^2}{2^n} + \frac{z^3}{3^n}\dots
}\end{equation}
\subsection{$BE$ 凝聚态}
\begin{equation}
N = \frac{V}{\lambda_c^3} g_{3/2} (1) \Rightarrow T_c = \frac{h^2}{2\pi mk} \qty(\frac{N}{2.612V})^{2/3}
\end{equation}
\begin{equation}
\frac{N_e}{N} = \frac{\lambda^3}{\lambda_c^3} \Rightarrow N_e = N\qty(\frac{T}{T_c})^{3/2} \Rightarrow {N_0} = N\qty[1 - \qty(\frac{T}{T_c})^{3/2}]
\end{equation}
\begin{equation}
N_0 = \frac{1}{\E^{(\varepsilon_0 - \mu )/kT} - 1} = \frac{kT}{\varepsilon_0 - \mu}
\end{equation}
\begin{equation}
\varepsilon_0 - \mu  \ll \varepsilon_1 - \varepsilon_0 \Rightarrow \varepsilon_0 - \mu  \ll \varepsilon_1 - \mu 
\end{equation}
\begin{equation}
N_1 = \frac{1}{\E^{(\varepsilon_1 - \mu )/kT} - 1} < \frac{kT}{\varepsilon_1 - \mu} \ll \frac{kT}{\varepsilon_0 - \mu } = N_0
\end{equation}
\subsection{范德瓦尔斯方程}
\begin{equation}
\qty(P + \frac{aN^2}{V^2}) (V - bN) = NkT
\end{equation}
\subsection{量子转子能级}
角量子数 $l$ 决定能级
\begin{equation}
E_l = l (l + 1)\frac{\hbar^2}{2IkT}
\end{equation}
$2l+1$ 重简并, 其中 $I = m_1 m_2 r_{12}^2/(m_1 + m_2)$ 为质心转动惯量. 当 $l$ 为偶数时, 两粒子的波函数具有交换对称, 奇数时反对称. 两原子核的自旋共有 $s^2 = (2I + 1)^2$ 种状态, 其中对称态占 $s(s + 1)/2$ 种, 反对称太占 $s(s - 1)/2$ 种. 若两粒子都是费米子($I$ 为半整数), 则总波函数反对称, 即 $l$ 为单数核自旋对称, 或 $l$ 为偶数核自旋反对称.

\subsection{弹簧振子能级}
\begin{equation}
E_n = \hbar \omega \qty(n + \frac12)
\end{equation}
非简并.

为什么书上说 $m = 0 $ (能级密度与 $\varepsilon^m$ 成正比)不能产生凝聚态, 然而我在模拟中做到了?