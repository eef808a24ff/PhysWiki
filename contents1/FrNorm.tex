% 径向波函数归一化

令一组 scaled 径向函数基底为 $u_{l,m}(k, r) = rR_{l,m}(k, r)$, 若满足
\begin{equation}
\int_0^\infty u^*_{l,m}(k',r) u_{l,m}(k, r) \dd{r} = \delta(k - k')
\end{equation}
则一组完备正交归一的基底为
\begin{equation}
s_{l,m}(k, \vec r) = \frac{1}{r} u_{l,m}(k, r) Y_{l,m}(\uvec r)
\end{equation}
正交归一条件为
\begin{equation}\ali{
&\quad \braket{s_{l,m}(k, \vec r)}{s_{l',m'}(k', \vec r)}\\
&= \int_0^\infty \frac{1}{r} u_{l,m}^*(k, r)  \frac{1}{r} u_{l',m'}^*(k', r)  r^2 \dd{r} \int Y_{l,m}^*(\uvec r) Y_{l',m'}(\uvec r) \dd{\Omega}\\
&= \delta_{l,l'}\delta_{m,m'}\delta(k-k')
}\end{equation}

任意复函数可以表示为
\begin{equation}\ali{
f(\vec r) &= \sum_{l,m} \int_0^\infty c_{l,m}(k) \ket{s_{l,m}(k, \vec r)} \dd{k}\\
&= \frac{1}{r}\sum_{l,m} \qty(\int_0^\infty c_{l,m}(k) u_{l,m}(k, r) \dd{k}) Y_{l,m}(\uvec r)
}\end{equation}
函数在任意基底上的投影为
\begin{equation}\ali{
\braket{s_{l,m}(k, \vec r)}{f(\vec r)} &= \sum_{l',m'} \int_0^\infty \dd{k'} c_{l',m'}(k') \braket{s_{l,m}(k, \vec r)}{s_{l',m'}(k', \vec r)}\\
& = \sum_{l',m'} \int_0^\infty \dd{k'} c_{l',m'}(k') \delta_{l,l'}\delta_{m,m'}\delta(k-k')\\
& = c_{l,m}(k)
}\end{equation}

若函数已经具有分波展开形式
\begin{equation}
f(\vec r) = 
\frac{1}{r}\sum_{l,m} g_{l,m}(r) Y_{l,m}(\uvec r)
\end{equation}
则 $g_{l,m}(r)$ 和 $c_{l,m}(k)$ 有类似傅里叶变换的关系
\begin{equation}
g_{l,m}(r) = \int_0^\infty c_{l,m}(k) u_{l,m}(k, r) \dd{k}
\end{equation}
\begin{equation}
c_{l,m}(k) = \int_0^\infty u_{l,m}^*(k, r) g_{l,m}(r) \dd{r}
\end{equation}

\subsection{归一化的球贝赛尔函数}
由于 $\E^{\I kx}$ 从负无穷到正无穷积分可以归一化为 $2\pi\delta(k-k')$, 易得 $\sin^{kx}$ 从负无穷到正无穷可以归一化为 $\pi\delta(k-k')$ (先表示成指数形式), 从 0 到正无穷归一化为 $\pi\delta(k-k')/2$.

注意归一化只需要渐进表达式即可(因为局部的不同相对于无穷积分来说可以忽略). 球贝赛尔函数的渐进形式为 $j_l(x) \to \sin(x - l\pi/2)/x$, 所以 $r j_l(kr)$ 从 0 到正无穷归一化为 $\pi\delta(k-k')/2$. 所以归一化的球贝赛尔函数为
\begin{equation}
u_{l,m}(k, r) = \sqrt{\frac{2}{\pi}} r j_l(kr)
\end{equation}
