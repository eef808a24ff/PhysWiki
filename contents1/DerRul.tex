%求导法则

\pentry{导数\upref{Der},基本初等函数的导数\upref{FunDer}}
\subsection{结论}
如果需要求导的函数可以看做若干个已知导函数的函数(如基本初等函数)经过四则运算或复合得到的, 那么我们可以直接使用一系列求导法则对其求导
\subsubsection{四则运算}
\begin{equation}
[ f(x) \pm g(x) ]' = f'(x) \pm g'(x)
\end{equation}
\begin{equation}
[ f(x)g(x) ]' = f'(x)g(x) + f(x)g'(x) 
\end{equation}
\begin{equation}
\qty[ \frac{f(x)}{g(x)} ]'  = \frac{f'(x)g(x) - g'(x)f(x)}{g(x)^2}
\end{equation}
\subsubsection{复合函数}
\begin{equation}
f[g(x)] = f'[g(x)]g'(x)
\end{equation}
详细见“一元复合函数求导(链式法则)\upref{ChainR}”

\subsection{线性}
对求导而言, \bb{线性}是指若干\bb{函数线性组合}(即把若干个函数分别乘以常数再相加)的求导等于对这些函数先分别求导再进行同样的线性组合. 由于函数加减法属于函数线性组合的两种简单情况, 这里只需要证明求导运算是线性的, 即求导是一种\bb{线性运算} 即可.  令若干常数为 $c_i$, 若干可导函数为 $f_i(x)$, 根据导数的定义, 这些函数线性组合的导数为
\begin{equation}\ali{
\dv{x}\sum_i c_i f_i(x) &= \lim_{h\to 0} \qty[\sum_i c_i f_i(x+h) - \sum_i c_i f_i(x) ]/h\\
& =  \sum_i c_i \lim_{h\to 0} [f_i(x+h) - f_i(x)]/h\\
& = \sum_i c_i f_i'(x)
}\end{equation}

\begin{exam}{对函数 $f(x) = 5\sin x + 3x^2$ 求导}
这里的 $f(x)$ 可以看做三角函数 $\sin x$ 函数和幂函数 $x^2$ 的线性组合, 二者都是基本初等函数, 导数分别为 $\cos x$ 和 $2x$, 由于求导是线性运算, 我们只需要对两个函数各自的导函数进行同样的线性组合即可
\begin{equation}
f'(x) = 5 \sin' x + 3(x^2)' = 5 \cos x + 3(2x) = 5\cos x + 6x
\end{equation}
\end{exam}

\subsection{两函数相乘的导数}
令两函数分别为 $f(x)$ 和 $g(x)$, 现在求 $f(x) g(x)$ 的导函数. 由导数的定义\autoref{Der_eq2}\upref{Der} 得
\begin{equation}
[f(x)g(x)]' = \lim_{h\to 0} [f(x+h)g(x+h) - f(x)g(x)]/h
\end{equation}
从几何上来看, 我们可以把 $f(x)g(x)$ 看做一个矩形的面积
% 图未完成


















