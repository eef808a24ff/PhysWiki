% 用于 \lstlisting 和 \lstinputlisting 显示 m 文件
\lstdefinelanguage{matlab}
{
    basicstyle= \ttfamily\zihao{5}, % 基本字体
    backgroundcolor=\color{background}, % 背景颜色
    breaklines=true, % 自动换行
    % 边框
    frame=single,
    framerule=0.2mm, % 粗细
    rulecolor=\color{gray}, % 线条颜色
    % keyword
    morekeywords={matlabTestKeyWord, break, case, catch, classdef, continue, else, elseif, end, for, function, global, if, otherwie, parfor, persistent, return, spmd, switch, try, while},
    keywordstyle=\color{blue}\textbf,
    % comment
    morecomment=[l]{\%},
    commentstyle=\color{comment},
    % string
    morestring=[m]',
    stringstyle=\color{string},
    showstringspaces=false,
    % line numbers
    numbers=left,
    numberstyle={\ttfamily\zihao{-5}\color{gray}}, % 行号大小
}

% 用于 \lstlisting 和 \lstinputlisting 显示 Matlab 控制行
\lstdefinelanguage{matlabC}
{
    basicstyle= \ttfamily\zihao{5}, % 基本字体
    backgroundcolor=\color{background}, % 背景颜色
    breaklines=true, % 自动换行
    % keyword
    morekeywords={matlabCTestKeyWord, break, case, catch, classdef, continue, else, elseif, end, for, function, global, if, otherwie, parfor, persistent, return, spmd, switch, try, while},
    keywordstyle=\color{blue}\textbf,
    % comment
    morecomment=[l]{\%},
    commentstyle=\color{comment},
    % string
    morestring=[m]',
    stringstyle=\color{string},
    showstringspaces=false,
}

% 用于 \lstlisting 和 \lstinputlisting 显示 m 文件
\lstdefinelanguage{cpp}
{
    basicstyle= \ttfamily\zihao{5}, % 基本字体
    backgroundcolor=\color{background}, % 背景颜色
    breaklines=true, % 自动换行
    % 边框
    frame=single,
    framerule=0.2mm, % 粗细
    rulecolor=\color{gray}, % 线条颜色
    % keyword
    morekeywords={if, else, inline, using, namespace, bool, char, int, long, double, Bool, Bool_I, Bool_O, Char, Char_I, Char_O, Int, Int_I, Int_O, Long, Long_I, Long_O, Doub, Doub_I, Doub_O, Comp, Comp_I, Comp_O, return, template, class, void, typename, define, const, constexpr, public, protected, private, explicit, default, typedef, sizeof},
    keywordstyle=\color{blue}\textbf,
   % 未完成: 可以将 keyword 分组, 用不同颜色高亮, 将 cpp 本身的 keyword 分为第一组, 用户定义的类型分为第二组
    % comment
    morecomment=[l]{//},
    morecomment=[s]{/*}{*/},
    commentstyle=\color{comment},
    % string
    morestring=[m]', % 未完成
    stringstyle=\color{string},
    showstringspaces=false,
    % line numbers
    numbers=left,
    numberstyle={\ttfamily\zihao{-5}\color{gray}}, % 行号大小
}

% 用于 \lstlisting 和 \lstinputlisting 显示 python 代码
\lstdefinelanguage{python}
{
    basicstyle= \ttfamily\zihao{5}, % 基本字体
    backgroundcolor=\color{background}, % 背景颜色
    breaklines=true, % 自动换行
    % 边框
    frame=single,
    framerule=0.2mm, % 粗细
    rulecolor=\color{gray}, % 线条颜色
    % keyword
    morekeywords={and, as, assert, break, class, continue, def, del, elif, else, except, False, finally, for, from, global, if, import, in, is, lambda, None, nonlocal, not, or, pass, raise, return, True, try, while, with, yield},
    keywordstyle=\color{blue}\textbf,
   % 未完成: 可以将 keyword 分组, 用不同颜色高亮, 将 cpp 本身的 keyword 分为第一组, 用户定义的类型分为第二组
    % comment
    morecomment=[l]{\#},
    morecomment=[s]{"""}{"""},
    commentstyle=\color{comment},
    % string
    morestring=[m]', % 未完成
    stringstyle=\color{string},
    showstringspaces=false,
    % line numbers
    numbers=left,
    numberstyle={\ttfamily\zihao{-5}\color{gray}}, % 行号大小
}

% 用于 \lstlisting 和 \lstinputlisting 显示 makefile 代码
\lstdefinelanguage{makefile}
{
    basicstyle= \ttfamily\zihao{5}, % 基本字体
    backgroundcolor=\color{background}, % 背景颜色
    breaklines=true, % 自动换行
    % 边框
    frame=single,
    framerule=0.2mm, % 粗细
    rulecolor=\color{gray}, % 线条颜色
    % keyword
    morekeywords={goal, include},
    keywordstyle=\color{blue}\textbf,
   % 未完成: 可以将 keyword 分组, 用不同颜色高亮, 将 cpp 本身的 keyword 分为第一组, 用户定义的类型分为第二组
    % comment
    morecomment=[l]{\#},
    commentstyle=\color{comment},
    % string
    morestring=[m]', % 未完成
    stringstyle=\color{string},
    showstringspaces=false,
    % line numbers
    numbers=left,
    numberstyle={\ttfamily\zihao{-5}\color{gray}}, % 行号大小
}

% 用于 \lstlisting 和 \lstinputlisting 显示 latex 代码
\lstdefinelanguage{latex}
{
    basicstyle= \ttfamily\zihao{5}, % 基本字体
    backgroundcolor=\color{background}, % 背景颜色
    breaklines=true, % 自动换行
    % 边框
    frame=single,
    framerule=0.2mm, % 粗细
    rulecolor=\color{gray}, % 线条颜色
    % keyword
    morekeywords={documentclass, usepackage, begin, end},
    keywordstyle=\color{blue}\textbf,
   % 未完成: 可以将 keyword 分组, 用不同颜色高亮, 将 cpp 本身的 keyword 分为第一组, 用户定义的类型分为第二组
    % comment
    morecomment=[l]{\#},
    commentstyle=\color{comment},
    % string
    morestring=[m]', % 未完成
    stringstyle=\color{string},
    showstringspaces=false,
    % line numbers
    numbers=left,
    numberstyle={\ttfamily\zihao{-5}\color{gray}}, % 行号大小
}

% 用于 \lstlisting 和 \lstinputlisting 显示 bash 代码
\lstdefinelanguage{bash}
{
    basicstyle= \ttfamily\zihao{5}, % 基本字体
    backgroundcolor=\color{background}, % 背景颜色
    breaklines=true, % 自动换行
    % 边框
    frame=single,
    framerule=0.2mm, % 粗细
    rulecolor=\color{gray}, % 线条颜色
    % keyword
    morekeywords={cd, pwe, echo, cat, git, ls, mount, umount, df, du},
    keywordstyle=\color{blue}\textbf,
   % 未完成: 可以将 keyword 分组, 用不同颜色高亮, 将 cpp 本身的 keyword 分为第一组, 用户定义的类型分为第二组
    % comment
    morecomment=[l]{\#},
    commentstyle=\color{comment},
    % string
    morestring=[m]', % 未完成
    stringstyle=\color{string},
    showstringspaces=false,
    % line numbers
    numbers=left,
    numberstyle={\ttfamily\zihao{-5}\color{gray}}, % 行号大小
}

% 修改 \verb|...| 的默认字体和颜色
\makeatletter 
  \renewcommand\verbatim@font{\normalfont\ttfamily\color{comment}}
\makeatother
